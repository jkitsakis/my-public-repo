\documentclass[11pt]{article}

    \usepackage[breakable]{tcolorbox}
    \usepackage{parskip} % Stop auto-indenting (to mimic markdown behaviour)
    

    % Basic figure setup, for now with no caption control since it's done
    % automatically by Pandoc (which extracts ![](path) syntax from Markdown).
    \usepackage{graphicx}
    % Maintain compatibility with old templates. Remove in nbconvert 6.0
    \let\Oldincludegraphics\includegraphics
    % Ensure that by default, figures have no caption (until we provide a
    % proper Figure object with a Caption API and a way to capture that
    % in the conversion process - todo).
    \usepackage{caption}
    \DeclareCaptionFormat{nocaption}{}
    \captionsetup{format=nocaption,aboveskip=0pt,belowskip=0pt}

    \usepackage{float}
    \floatplacement{figure}{H} % forces figures to be placed at the correct location
    \usepackage{xcolor} % Allow colors to be defined
    \usepackage{enumerate} % Needed for markdown enumerations to work
    \usepackage{geometry} % Used to adjust the document margins
    \usepackage{amsmath} % Equations
    \usepackage{amssymb} % Equations
    \usepackage{textcomp} % defines textquotesingle
    % Hack from http://tex.stackexchange.com/a/47451/13684:
    \AtBeginDocument{%
        \def\PYZsq{\textquotesingle}% Upright quotes in Pygmentized code
    }
    \usepackage{upquote} % Upright quotes for verbatim code
    \usepackage{eurosym} % defines \euro

    \usepackage{iftex}
    \ifPDFTeX
        \usepackage[T1]{fontenc}
        \IfFileExists{alphabeta.sty}{
              \usepackage{alphabeta}
          }{
              \usepackage[mathletters]{ucs}
              \usepackage[utf8x]{inputenc}
          }
    \else
        \usepackage{fontspec}
        \usepackage{unicode-math}
    \fi

    \usepackage{fancyvrb} % verbatim replacement that allows latex
    \usepackage[Export]{adjustbox} % Used to constrain images to a maximum size
    \adjustboxset{max size={0.9\linewidth}{0.9\paperheight}}

    % The hyperref package gives us a pdf with properly built
    % internal navigation ('pdf bookmarks' for the table of contents,
    % internal cross-reference links, web links for URLs, etc.)
    \usepackage{hyperref}
    % The default LaTeX title has an obnoxious amount of whitespace. By default,
    % titling removes some of it. It also provides customization options.
    \usepackage{titling}
    \usepackage{longtable} % longtable support required by pandoc >1.10
    \usepackage{booktabs}  % table support for pandoc > 1.12.2
    \usepackage{array}     % table support for pandoc >= 2.11.3
    \usepackage{calc}      % table minipage width calculation for pandoc >= 2.11.1
    \usepackage[inline]{enumitem} % IRkernel/repr support (it uses the enumerate* environment)
    \usepackage[normalem]{ulem} % ulem is needed to support strikethroughs (\sout)
                                % normalem makes italics be italics, not underlines
    \usepackage{mathrsfs}
    

    
    % Colors for the hyperref package
    \definecolor{urlcolor}{rgb}{0,.145,.698}
    \definecolor{linkcolor}{rgb}{.71,0.21,0.01}
    \definecolor{citecolor}{rgb}{.12,.54,.11}

    % ANSI colors
    \definecolor{ansi-black}{HTML}{3E424D}
    \definecolor{ansi-black-intense}{HTML}{282C36}
    \definecolor{ansi-red}{HTML}{E75C58}
    \definecolor{ansi-red-intense}{HTML}{B22B31}
    \definecolor{ansi-green}{HTML}{00A250}
    \definecolor{ansi-green-intense}{HTML}{007427}
    \definecolor{ansi-yellow}{HTML}{DDB62B}
    \definecolor{ansi-yellow-intense}{HTML}{B27D12}
    \definecolor{ansi-blue}{HTML}{208FFB}
    \definecolor{ansi-blue-intense}{HTML}{0065CA}
    \definecolor{ansi-magenta}{HTML}{D160C4}
    \definecolor{ansi-magenta-intense}{HTML}{A03196}
    \definecolor{ansi-cyan}{HTML}{60C6C8}
    \definecolor{ansi-cyan-intense}{HTML}{258F8F}
    \definecolor{ansi-white}{HTML}{C5C1B4}
    \definecolor{ansi-white-intense}{HTML}{A1A6B2}
    \definecolor{ansi-default-inverse-fg}{HTML}{FFFFFF}
    \definecolor{ansi-default-inverse-bg}{HTML}{000000}

    % common color for the border for error outputs.
    \definecolor{outerrorbackground}{HTML}{FFDFDF}

    % commands and environments needed by pandoc snippets
    % extracted from the output of `pandoc -s`
    \providecommand{\tightlist}{%
      \setlength{\itemsep}{0pt}\setlength{\parskip}{0pt}}
    \DefineVerbatimEnvironment{Highlighting}{Verbatim}{commandchars=\\\{\}}
    % Add ',fontsize=\small' for more characters per line
    \newenvironment{Shaded}{}{}
    \newcommand{\KeywordTok}[1]{\textcolor[rgb]{0.00,0.44,0.13}{\textbf{{#1}}}}
    \newcommand{\DataTypeTok}[1]{\textcolor[rgb]{0.56,0.13,0.00}{{#1}}}
    \newcommand{\DecValTok}[1]{\textcolor[rgb]{0.25,0.63,0.44}{{#1}}}
    \newcommand{\BaseNTok}[1]{\textcolor[rgb]{0.25,0.63,0.44}{{#1}}}
    \newcommand{\FloatTok}[1]{\textcolor[rgb]{0.25,0.63,0.44}{{#1}}}
    \newcommand{\CharTok}[1]{\textcolor[rgb]{0.25,0.44,0.63}{{#1}}}
    \newcommand{\StringTok}[1]{\textcolor[rgb]{0.25,0.44,0.63}{{#1}}}
    \newcommand{\CommentTok}[1]{\textcolor[rgb]{0.38,0.63,0.69}{\textit{{#1}}}}
    \newcommand{\OtherTok}[1]{\textcolor[rgb]{0.00,0.44,0.13}{{#1}}}
    \newcommand{\AlertTok}[1]{\textcolor[rgb]{1.00,0.00,0.00}{\textbf{{#1}}}}
    \newcommand{\FunctionTok}[1]{\textcolor[rgb]{0.02,0.16,0.49}{{#1}}}
    \newcommand{\RegionMarkerTok}[1]{{#1}}
    \newcommand{\ErrorTok}[1]{\textcolor[rgb]{1.00,0.00,0.00}{\textbf{{#1}}}}
    \newcommand{\NormalTok}[1]{{#1}}
    
    % Additional commands for more recent versions of Pandoc
    \newcommand{\ConstantTok}[1]{\textcolor[rgb]{0.53,0.00,0.00}{{#1}}}
    \newcommand{\SpecialCharTok}[1]{\textcolor[rgb]{0.25,0.44,0.63}{{#1}}}
    \newcommand{\VerbatimStringTok}[1]{\textcolor[rgb]{0.25,0.44,0.63}{{#1}}}
    \newcommand{\SpecialStringTok}[1]{\textcolor[rgb]{0.73,0.40,0.53}{{#1}}}
    \newcommand{\ImportTok}[1]{{#1}}
    \newcommand{\DocumentationTok}[1]{\textcolor[rgb]{0.73,0.13,0.13}{\textit{{#1}}}}
    \newcommand{\AnnotationTok}[1]{\textcolor[rgb]{0.38,0.63,0.69}{\textbf{\textit{{#1}}}}}
    \newcommand{\CommentVarTok}[1]{\textcolor[rgb]{0.38,0.63,0.69}{\textbf{\textit{{#1}}}}}
    \newcommand{\VariableTok}[1]{\textcolor[rgb]{0.10,0.09,0.49}{{#1}}}
    \newcommand{\ControlFlowTok}[1]{\textcolor[rgb]{0.00,0.44,0.13}{\textbf{{#1}}}}
    \newcommand{\OperatorTok}[1]{\textcolor[rgb]{0.40,0.40,0.40}{{#1}}}
    \newcommand{\BuiltInTok}[1]{{#1}}
    \newcommand{\ExtensionTok}[1]{{#1}}
    \newcommand{\PreprocessorTok}[1]{\textcolor[rgb]{0.74,0.48,0.00}{{#1}}}
    \newcommand{\AttributeTok}[1]{\textcolor[rgb]{0.49,0.56,0.16}{{#1}}}
    \newcommand{\InformationTok}[1]{\textcolor[rgb]{0.38,0.63,0.69}{\textbf{\textit{{#1}}}}}
    \newcommand{\WarningTok}[1]{\textcolor[rgb]{0.38,0.63,0.69}{\textbf{\textit{{#1}}}}}
    
    
    % Define a nice break command that doesn't care if a line doesn't already
    % exist.
    \def\br{\hspace*{\fill} \\* }
    % Math Jax compatibility definitions
    \def\gt{>}
    \def\lt{<}
    \let\Oldtex\TeX
    \let\Oldlatex\LaTeX
    \renewcommand{\TeX}{\textrm{\Oldtex}}
    \renewcommand{\LaTeX}{\textrm{\Oldlatex}}
    % Document parameters
    % Document title
    \title{Probability Statistics}
    
    
    
    
    
% Pygments definitions
\makeatletter
\def\PY@reset{\let\PY@it=\relax \let\PY@bf=\relax%
    \let\PY@ul=\relax \let\PY@tc=\relax%
    \let\PY@bc=\relax \let\PY@ff=\relax}
\def\PY@tok#1{\csname PY@tok@#1\endcsname}
\def\PY@toks#1+{\ifx\relax#1\empty\else%
    \PY@tok{#1}\expandafter\PY@toks\fi}
\def\PY@do#1{\PY@bc{\PY@tc{\PY@ul{%
    \PY@it{\PY@bf{\PY@ff{#1}}}}}}}
\def\PY#1#2{\PY@reset\PY@toks#1+\relax+\PY@do{#2}}

\@namedef{PY@tok@w}{\def\PY@tc##1{\textcolor[rgb]{0.73,0.73,0.73}{##1}}}
\@namedef{PY@tok@c}{\let\PY@it=\textit\def\PY@tc##1{\textcolor[rgb]{0.24,0.48,0.48}{##1}}}
\@namedef{PY@tok@cp}{\def\PY@tc##1{\textcolor[rgb]{0.61,0.40,0.00}{##1}}}
\@namedef{PY@tok@k}{\let\PY@bf=\textbf\def\PY@tc##1{\textcolor[rgb]{0.00,0.50,0.00}{##1}}}
\@namedef{PY@tok@kp}{\def\PY@tc##1{\textcolor[rgb]{0.00,0.50,0.00}{##1}}}
\@namedef{PY@tok@kt}{\def\PY@tc##1{\textcolor[rgb]{0.69,0.00,0.25}{##1}}}
\@namedef{PY@tok@o}{\def\PY@tc##1{\textcolor[rgb]{0.40,0.40,0.40}{##1}}}
\@namedef{PY@tok@ow}{\let\PY@bf=\textbf\def\PY@tc##1{\textcolor[rgb]{0.67,0.13,1.00}{##1}}}
\@namedef{PY@tok@nb}{\def\PY@tc##1{\textcolor[rgb]{0.00,0.50,0.00}{##1}}}
\@namedef{PY@tok@nf}{\def\PY@tc##1{\textcolor[rgb]{0.00,0.00,1.00}{##1}}}
\@namedef{PY@tok@nc}{\let\PY@bf=\textbf\def\PY@tc##1{\textcolor[rgb]{0.00,0.00,1.00}{##1}}}
\@namedef{PY@tok@nn}{\let\PY@bf=\textbf\def\PY@tc##1{\textcolor[rgb]{0.00,0.00,1.00}{##1}}}
\@namedef{PY@tok@ne}{\let\PY@bf=\textbf\def\PY@tc##1{\textcolor[rgb]{0.80,0.25,0.22}{##1}}}
\@namedef{PY@tok@nv}{\def\PY@tc##1{\textcolor[rgb]{0.10,0.09,0.49}{##1}}}
\@namedef{PY@tok@no}{\def\PY@tc##1{\textcolor[rgb]{0.53,0.00,0.00}{##1}}}
\@namedef{PY@tok@nl}{\def\PY@tc##1{\textcolor[rgb]{0.46,0.46,0.00}{##1}}}
\@namedef{PY@tok@ni}{\let\PY@bf=\textbf\def\PY@tc##1{\textcolor[rgb]{0.44,0.44,0.44}{##1}}}
\@namedef{PY@tok@na}{\def\PY@tc##1{\textcolor[rgb]{0.41,0.47,0.13}{##1}}}
\@namedef{PY@tok@nt}{\let\PY@bf=\textbf\def\PY@tc##1{\textcolor[rgb]{0.00,0.50,0.00}{##1}}}
\@namedef{PY@tok@nd}{\def\PY@tc##1{\textcolor[rgb]{0.67,0.13,1.00}{##1}}}
\@namedef{PY@tok@s}{\def\PY@tc##1{\textcolor[rgb]{0.73,0.13,0.13}{##1}}}
\@namedef{PY@tok@sd}{\let\PY@it=\textit\def\PY@tc##1{\textcolor[rgb]{0.73,0.13,0.13}{##1}}}
\@namedef{PY@tok@si}{\let\PY@bf=\textbf\def\PY@tc##1{\textcolor[rgb]{0.64,0.35,0.47}{##1}}}
\@namedef{PY@tok@se}{\let\PY@bf=\textbf\def\PY@tc##1{\textcolor[rgb]{0.67,0.36,0.12}{##1}}}
\@namedef{PY@tok@sr}{\def\PY@tc##1{\textcolor[rgb]{0.64,0.35,0.47}{##1}}}
\@namedef{PY@tok@ss}{\def\PY@tc##1{\textcolor[rgb]{0.10,0.09,0.49}{##1}}}
\@namedef{PY@tok@sx}{\def\PY@tc##1{\textcolor[rgb]{0.00,0.50,0.00}{##1}}}
\@namedef{PY@tok@m}{\def\PY@tc##1{\textcolor[rgb]{0.40,0.40,0.40}{##1}}}
\@namedef{PY@tok@gh}{\let\PY@bf=\textbf\def\PY@tc##1{\textcolor[rgb]{0.00,0.00,0.50}{##1}}}
\@namedef{PY@tok@gu}{\let\PY@bf=\textbf\def\PY@tc##1{\textcolor[rgb]{0.50,0.00,0.50}{##1}}}
\@namedef{PY@tok@gd}{\def\PY@tc##1{\textcolor[rgb]{0.63,0.00,0.00}{##1}}}
\@namedef{PY@tok@gi}{\def\PY@tc##1{\textcolor[rgb]{0.00,0.52,0.00}{##1}}}
\@namedef{PY@tok@gr}{\def\PY@tc##1{\textcolor[rgb]{0.89,0.00,0.00}{##1}}}
\@namedef{PY@tok@ge}{\let\PY@it=\textit}
\@namedef{PY@tok@gs}{\let\PY@bf=\textbf}
\@namedef{PY@tok@gp}{\let\PY@bf=\textbf\def\PY@tc##1{\textcolor[rgb]{0.00,0.00,0.50}{##1}}}
\@namedef{PY@tok@go}{\def\PY@tc##1{\textcolor[rgb]{0.44,0.44,0.44}{##1}}}
\@namedef{PY@tok@gt}{\def\PY@tc##1{\textcolor[rgb]{0.00,0.27,0.87}{##1}}}
\@namedef{PY@tok@err}{\def\PY@bc##1{{\setlength{\fboxsep}{\string -\fboxrule}\fcolorbox[rgb]{1.00,0.00,0.00}{1,1,1}{\strut ##1}}}}
\@namedef{PY@tok@kc}{\let\PY@bf=\textbf\def\PY@tc##1{\textcolor[rgb]{0.00,0.50,0.00}{##1}}}
\@namedef{PY@tok@kd}{\let\PY@bf=\textbf\def\PY@tc##1{\textcolor[rgb]{0.00,0.50,0.00}{##1}}}
\@namedef{PY@tok@kn}{\let\PY@bf=\textbf\def\PY@tc##1{\textcolor[rgb]{0.00,0.50,0.00}{##1}}}
\@namedef{PY@tok@kr}{\let\PY@bf=\textbf\def\PY@tc##1{\textcolor[rgb]{0.00,0.50,0.00}{##1}}}
\@namedef{PY@tok@bp}{\def\PY@tc##1{\textcolor[rgb]{0.00,0.50,0.00}{##1}}}
\@namedef{PY@tok@fm}{\def\PY@tc##1{\textcolor[rgb]{0.00,0.00,1.00}{##1}}}
\@namedef{PY@tok@vc}{\def\PY@tc##1{\textcolor[rgb]{0.10,0.09,0.49}{##1}}}
\@namedef{PY@tok@vg}{\def\PY@tc##1{\textcolor[rgb]{0.10,0.09,0.49}{##1}}}
\@namedef{PY@tok@vi}{\def\PY@tc##1{\textcolor[rgb]{0.10,0.09,0.49}{##1}}}
\@namedef{PY@tok@vm}{\def\PY@tc##1{\textcolor[rgb]{0.10,0.09,0.49}{##1}}}
\@namedef{PY@tok@sa}{\def\PY@tc##1{\textcolor[rgb]{0.73,0.13,0.13}{##1}}}
\@namedef{PY@tok@sb}{\def\PY@tc##1{\textcolor[rgb]{0.73,0.13,0.13}{##1}}}
\@namedef{PY@tok@sc}{\def\PY@tc##1{\textcolor[rgb]{0.73,0.13,0.13}{##1}}}
\@namedef{PY@tok@dl}{\def\PY@tc##1{\textcolor[rgb]{0.73,0.13,0.13}{##1}}}
\@namedef{PY@tok@s2}{\def\PY@tc##1{\textcolor[rgb]{0.73,0.13,0.13}{##1}}}
\@namedef{PY@tok@sh}{\def\PY@tc##1{\textcolor[rgb]{0.73,0.13,0.13}{##1}}}
\@namedef{PY@tok@s1}{\def\PY@tc##1{\textcolor[rgb]{0.73,0.13,0.13}{##1}}}
\@namedef{PY@tok@mb}{\def\PY@tc##1{\textcolor[rgb]{0.40,0.40,0.40}{##1}}}
\@namedef{PY@tok@mf}{\def\PY@tc##1{\textcolor[rgb]{0.40,0.40,0.40}{##1}}}
\@namedef{PY@tok@mh}{\def\PY@tc##1{\textcolor[rgb]{0.40,0.40,0.40}{##1}}}
\@namedef{PY@tok@mi}{\def\PY@tc##1{\textcolor[rgb]{0.40,0.40,0.40}{##1}}}
\@namedef{PY@tok@il}{\def\PY@tc##1{\textcolor[rgb]{0.40,0.40,0.40}{##1}}}
\@namedef{PY@tok@mo}{\def\PY@tc##1{\textcolor[rgb]{0.40,0.40,0.40}{##1}}}
\@namedef{PY@tok@ch}{\let\PY@it=\textit\def\PY@tc##1{\textcolor[rgb]{0.24,0.48,0.48}{##1}}}
\@namedef{PY@tok@cm}{\let\PY@it=\textit\def\PY@tc##1{\textcolor[rgb]{0.24,0.48,0.48}{##1}}}
\@namedef{PY@tok@cpf}{\let\PY@it=\textit\def\PY@tc##1{\textcolor[rgb]{0.24,0.48,0.48}{##1}}}
\@namedef{PY@tok@c1}{\let\PY@it=\textit\def\PY@tc##1{\textcolor[rgb]{0.24,0.48,0.48}{##1}}}
\@namedef{PY@tok@cs}{\let\PY@it=\textit\def\PY@tc##1{\textcolor[rgb]{0.24,0.48,0.48}{##1}}}

\def\PYZbs{\char`\\}
\def\PYZus{\char`\_}
\def\PYZob{\char`\{}
\def\PYZcb{\char`\}}
\def\PYZca{\char`\^}
\def\PYZam{\char`\&}
\def\PYZlt{\char`\<}
\def\PYZgt{\char`\>}
\def\PYZsh{\char`\#}
\def\PYZpc{\char`\%}
\def\PYZdl{\char`$}
\def\PYZhy{\char`\-}
\def\PYZsq{\char`\'}
\def\PYZdq{\char`\"}
\def\PYZti{\char`\~}
% for compatibility with earlier versions
\def\PYZat{@}
\def\PYZlb{[}
\def\PYZrb{]}
\makeatother


    % For linebreaks inside Verbatim environment from package fancyvrb. 
    \makeatletter
        \newbox\Wrappedcontinuationbox 
        \newbox\Wrappedvisiblespacebox 
        \newcommand*\Wrappedvisiblespace {\textcolor{red}{\textvisiblespace}} 
        \newcommand*\Wrappedcontinuationsymbol {\textcolor{red}{\llap{\tiny$\m@th\hookrightarrow$}}} 
        \newcommand*\Wrappedcontinuationindent {3ex } 
        \newcommand*\Wrappedafterbreak {\kern\Wrappedcontinuationindent\copy\Wrappedcontinuationbox} 
        % Take advantage of the already applied Pygments mark-up to insert 
        % potential linebreaks for TeX processing. 
        %        {, <, #, %, $, ' and ": go to next line. 
        %        _, }, ^, &, >, - and ~: stay at end of broken line. 
        % Use of \textquotesingle for straight quote. 
        \newcommand*\Wrappedbreaksatspecials {% 
            \def\PYGZus{\discretionary{\char`\_}{\Wrappedafterbreak}{\char`\_}}% 
            \def\PYGZob{\discretionary{}{\Wrappedafterbreak\char`\{}{\char`\{}}% 
            \def\PYGZcb{\discretionary{\char`\}}{\Wrappedafterbreak}{\char`\}}}% 
            \def\PYGZca{\discretionary{\char`\^}{\Wrappedafterbreak}{\char`\^}}% 
            \def\PYGZam{\discretionary{\char`\&}{\Wrappedafterbreak}{\char`\&}}% 
            \def\PYGZlt{\discretionary{}{\Wrappedafterbreak\char`\<}{\char`\<}}% 
            \def\PYGZgt{\discretionary{\char`\>}{\Wrappedafterbreak}{\char`\>}}% 
            \def\PYGZsh{\discretionary{}{\Wrappedafterbreak\char`\#}{\char`\#}}% 
            \def\PYGZpc{\discretionary{}{\Wrappedafterbreak\char`\%}{\char`\%}}% 
            \def\PYGZdl{\discretionary{}{\Wrappedafterbreak\char`$}{\char`$}}% 
            \def\PYGZhy{\discretionary{\char`\-}{\Wrappedafterbreak}{\char`\-}}% 
            \def\PYGZsq{\discretionary{}{\Wrappedafterbreak\textquotesingle}{\textquotesingle}}% 
            \def\PYGZdq{\discretionary{}{\Wrappedafterbreak\char`\"}{\char`\"}}% 
            \def\PYGZti{\discretionary{\char`\~}{\Wrappedafterbreak}{\char`\~}}% 
        } 
        % Some characters . , ; ? ! / are not pygmentized. 
        % This macro makes them "active" and they will insert potential linebreaks 
        \newcommand*\Wrappedbreaksatpunct {% 
            \lccode`\~`\.\lowercase{\def~}{\discretionary{\hbox{\char`\.}}{\Wrappedafterbreak}{\hbox{\char`\.}}}% 
            \lccode`\~`\,\lowercase{\def~}{\discretionary{\hbox{\char`\,}}{\Wrappedafterbreak}{\hbox{\char`\,}}}% 
            \lccode`\~`\;\lowercase{\def~}{\discretionary{\hbox{\char`\;}}{\Wrappedafterbreak}{\hbox{\char`\;}}}% 
            \lccode`\~`\:\lowercase{\def~}{\discretionary{\hbox{\char`\:}}{\Wrappedafterbreak}{\hbox{\char`\:}}}% 
            \lccode`\~`\?\lowercase{\def~}{\discretionary{\hbox{\char`\?}}{\Wrappedafterbreak}{\hbox{\char`\?}}}% 
            \lccode`\~`\!\lowercase{\def~}{\discretionary{\hbox{\char`\!}}{\Wrappedafterbreak}{\hbox{\char`\!}}}% 
            \lccode`\~`\/\lowercase{\def~}{\discretionary{\hbox{\char`\/}}{\Wrappedafterbreak}{\hbox{\char`\/}}}% 
            \catcode`\.\active
            \catcode`\,\active 
            \catcode`\;\active
            \catcode`\:\active
            \catcode`\?\active
            \catcode`\!\active
            \catcode`\/\active 
            \lccode`\~`\~ 	
        }
    \makeatother

    \let\OriginalVerbatim=\Verbatim
    \makeatletter
    \renewcommand{\Verbatim}[1][1]{%
        %\parskip\z@skip
        \sbox\Wrappedcontinuationbox {\Wrappedcontinuationsymbol}%
        \sbox\Wrappedvisiblespacebox {\FV@SetupFont\Wrappedvisiblespace}%
        \def\FancyVerbFormatLine ##1{\hsize\linewidth
            \vtop{\raggedright\hyphenpenalty\z@\exhyphenpenalty\z@
                \doublehyphendemerits\z@\finalhyphendemerits\z@
                \strut ##1\strut}%
        }%
        % If the linebreak is at a space, the latter will be displayed as visible
        % space at end of first line, and a continuation symbol starts next line.
        % Stretch/shrink are however usually zero for typewriter font.
        \def\FV@Space {%
            \nobreak\hskip\z@ plus\fontdimen3\font minus\fontdimen4\font
            \discretionary{\copy\Wrappedvisiblespacebox}{\Wrappedafterbreak}
            {\kern\fontdimen2\font}%
        }%
        
        % Allow breaks at special characters using \PYG... macros.
        \Wrappedbreaksatspecials
        % Breaks at punctuation characters . , ; ? ! and / need catcode=\active 	
        \OriginalVerbatim[#1,codes*=\Wrappedbreaksatpunct]%
    }
    \makeatother

    % Exact colors from NB
    \definecolor{incolor}{HTML}{303F9F}
    \definecolor{outcolor}{HTML}{D84315}
    \definecolor{cellborder}{HTML}{CFCFCF}
    \definecolor{cellbackground}{HTML}{F7F7F7}
    
    % prompt
    \makeatletter
    \newcommand{\boxspacing}{\kern\kvtcb@left@rule\kern\kvtcb@boxsep}
    \makeatother
    \newcommand{\prompt}[4]{
        {\ttfamily\llap{{\color{#2}[#3]:\hspace{3pt}#4}}\vspace{-\baselineskip}}
    }
    

    
    % Prevent overflowing lines due to hard-to-break entities
    \sloppy 
    % Setup hyperref package
    \hypersetup{
      breaklinks=true,  % so long urls are correctly broken across lines
      colorlinks=true,
      urlcolor=urlcolor,
      linkcolor=linkcolor,
      citecolor=citecolor,
      }
    % Slightly bigger margins than the latex defaults
    
    \geometry{verbose,tmargin=1in,bmargin=1in,lmargin=1in,rmargin=1in}
    
    

\begin{document}
    
    \maketitle
    
    

    
    \(\textbf{The Monty Hall problem}\)

    Το πρόβλημα του Μόντι Χολ είναι ένα πασίγνωστο πρόβλημα πιθανοτήτων που
ονομάστηκε έτσι από τον ονοματοδότη του, τον παρουσιαστή του τηλεοπτικού
παιχνιδιού ``Let's Make a Deal'', Μόντι Χολ.

Το πρόβλημα αφορά έναν διαγωνιζόμενο που επιλέγει μεταξύ τριών πιθανών
προσφορών. Πίσω από μία από αυτές τις προσφορές κρύβεται ένα αξιοθέατο
έπαθλο, ενώ πίσω από τις άλλες δύο βρίσκονται απλά αντικείμενα που δεν
ενδιαφέρουν τον διαγωνιζόμενο.

Αρχικά, ο διαγωνιζόμενος επιλέγει μία από τις τρεις προσφορές.\\
Στη συνέχεια, ο παρουσιαστής ανοίγει μία από τις δύο προσφορές που δεν
επέλεξε ο διαγωνιζόμενος, αποκαλύπτοντας ένα αντικείμενο που δεν είναι
το έπαθλο.\\
Στη συνέχεια, ο παρουσιαστής δίνει στον διαγωνιζόμενο την ευκαιρία να
αλλάξει την αρχική του επιλογή και να επιλέξει την άλλη προσφορά που
παραμένει ανοιχτή, ή να διατηρήσει την αρχική του επιλογή.\\
Το ερώτημα είναι αν είναι καλύτερο για τον διαγωνιζόμενο να αλλάξει την
αρχική του επιλογή ή να διατηρήσει αυτήν που είχε αρχικά κάνει.

Η σωστή απάντηση στο πρόβλημα είναι ότι είναι καλύτερο για τον
διαγωνιζόμενο να αλλάξει την αρχική του επιλογή, καθώς αυξάνει τις
πιθανότητες του να κερδίσει το έπαθλο.

    \begin{tcolorbox}[breakable, size=fbox, boxrule=1pt, pad at break*=1mm,colback=cellbackground, colframe=cellborder]
\prompt{In}{incolor}{ }{\boxspacing}
\begin{Verbatim}[commandchars=\\\{\}]

\end{Verbatim}
\end{tcolorbox}

    \(\textbf{Sets -Venn Diagrams}\)

    Τα σύνολα είναι μια έννοια στη θεωρία συνόλων που αναφέρεται σε ομάδες
αντικειμένων ή στοιχείων που έχουν κάποια κοινή χαρακτηριστικά ή
ιδιότητες. Ένα σύνολο μπορεί να περιλαμβάνει οποιοδήποτε αριθμό
στοιχείων, συμβολίζεται συνήθως με μεγάλα κεφαλαία γράμματα (όπως Α, Β,
Γ) και τα στοιχεία του συνόλου μπορεί να παρουσιάζονται με μικρά
γράμματα.

Οι διαγράμματα Venn είναι γραφικές αναπαραστάσεις των συνόλων και των
σχέσεών τους μεταξύ τους. Αυτά τα διαγράμματα χρησιμοποιούν κυκλικές ή
οβάλ περιοχές που επικαλύπτονται για να αναπαραστήσουν την κοινή ή τη
διαφορετική περιοχή των συνόλων. Κάθε σύνολο αντιπροσωπεύεται από μία
περιοχή και οι σχέσεις μεταξύ των συνόλων αναπαρίστανται από την
επικάλυψη των περιοχών αυτών.

Για παράδειγμα, έστω ότι έχουμε τα σύνολα Α και Β. Ένα διάγραμμα Venn με
δύο σύνολα θα περιλαμβάνει δύο κυκλικές περιοχές, μία για το σύνολο Α
και μία για το σύνολο Β. Η επικάλυψη των περιοχών αντιπροσωπεύει τα
στοιχεία που ανήκουν τόσο στο σύνολο Α όσο και στο σύνολο Β.

Τα διαγράμματα Venn είναι χρήσιμα για την εικόνιση των σχέσεων μεταξύ
των συνόλων και την ανάλυση των συνόλων σε υποσύνολα.

    \begin{tcolorbox}[breakable, size=fbox, boxrule=1pt, pad at break*=1mm,colback=cellbackground, colframe=cellborder]
\prompt{In}{incolor}{ }{\boxspacing}
\begin{Verbatim}[commandchars=\\\{\}]

\end{Verbatim}
\end{tcolorbox}

    \(\textbf{The Sample Space and the Event Space}\)

    Το δειγματικό χώρο (Sample Space) και το χώρο των συμβάντων (Event
Space) είναι σημαντικά έννοιες στη θεωρία πιθανοτήτων. Ας τις εξηγήσουμε
στα ελληνικά:

\begin{enumerate}
\def\labelenumi{\arabic{enumi}.}
\item
  Δειγματικός Χώρος (Sample Space): Ο δειγματικός χώρος είναι ο σύνολο
  όλων των πιθανών αποτελεσμάτων ενός πειράματος. Συμβολίζεται συνήθως
  με το γράμμα Ω (omega) και περιλαμβάνει όλα τα δυνατά αποτελέσματα που
  μπορούν να προκύψουν από το πείραμα.
\item
  Χώρος των Συμβάντων (Event Space): Ο χώρος των συμβάντων είναι το
  σύνολο όλων των πιθανών συμβάντων ή υποσυνόλων του δειγματικού χώρου.
  Ένα συμβάν είναι οποιαδήποτε υποσυνένολο του δειγματικού χώρου και
  αντιπροσωπεύει ένα συγκεκριμένο αποτέλεσμα ή σύνολο αποτελεσμάτων του
  πειράματος.
\end{enumerate}

Ο δειγματικός χώρος παρέχει το πλαίσιο για την ανάλυση όλων των πιθανών
αποτελεσμάτων, ενώ ο χώρος των συμβάντων επιτρέπει την αναπαράσταση και
τον υπολογισμό πιθανών συμβάντων μέσα σε αυτό το πλαίσιο.

    \begin{tcolorbox}[breakable, size=fbox, boxrule=1pt, pad at break*=1mm,colback=cellbackground, colframe=cellborder]
\prompt{In}{incolor}{ }{\boxspacing}
\begin{Verbatim}[commandchars=\\\{\}]

\end{Verbatim}
\end{tcolorbox}

    \(\textbf{Probability and set operations with examples}\)

    Η πιθανότητα και οι λειτουργίες συνόλων είναι σημαντικές έννοιες στον
τομέα της πιθανοτικής. Ας δούμε πώς μπορούν να χρησιμοποιηθούν σε
παραδείγματα στα ελληνικά:

\begin{enumerate}
\def\labelenumi{\arabic{enumi}.}
\tightlist
\item
  Πιθανότητα (Probability): Η πιθανότητα ενός συμβάντος αναφέρεται στη
  μέτρηση της συχνότητας με την οποία αυτό το συμβάν συμβαίνει σε ένα
  πείραμα. Υπολογίζεται ως τον αριθμό των επιθυμητών αποτελεσμάτων προς
  τον συνολικό αριθμό των πιθανών αποτελεσμάτων.
\end{enumerate}

Παράδειγμα: Ένα φαράγγι έχει τρία δρομολόγια: ένα εύκολο, ένα μεσαίο και
ένα δύσκολο. Η πιθανότητα ενός τυχαίου πεζοπορικού να επιλέξει το εύκολο
δρομολόγιο είναι 1/3.

\begin{enumerate}
\def\labelenumi{\arabic{enumi}.}
\setcounter{enumi}{1}
\tightlist
\item
  Λειτουργίες Συνόλων (Set Operations): Οι βασικές λειτουργίες συνόλων
  περιλαμβάνουν την ένωση, τη διαφορά, το τομήμα και το συμπλήρωμα.
\end{enumerate}

\begin{itemize}
\tightlist
\item
  Ένωση: Η ένωση δύο συνόλων περιλαμβάνει όλα τα μοναδικά στοιχεία από
  κάθε σύνολο.
\item
  Διαφορά: Η διαφορά δύο συνόλων περιλαμβάνει τα στοιχεία που υπάρχουν
  στο πρώτο σύνολο αλλά όχι στο δεύτερο.
\item
  Τομή: Η τομή δύο συνόλων περιλαμβάνει τα στοιχεία που υπάρχουν και στα
  δύο σύνολα.
\item
  Συμπλήρωμα: Το συμπλήρωμα ενός συνόλου αναφέρεται στα στοιχεία που δεν
  ανήκουν σε αυτό το σύνολο, σε σχέση με ένα άλλο σύνολο.
\end{itemize}

Παραδείγματα: Έστω δύο σύνολα: Α = \{1, 2, 3, 4\} και Β = \{3, 4, 5,
6\}.

\begin{itemize}
\tightlist
\item
  Ένωση: Α Ένωση Β = \{1, 2, 3, 4, 5, 6\}.
\item
  Διαφορά: Α Διαφορά Β = \{1, 2\}.
\item
  Τομή: Α Τομή Β = \{3, 4\}.
\item
  Συμπλήρωμα: Το συμπλήρωμα του συνόλου Α σε σχέση με το σύνολο Β είναι
  \{1, 2\}.
\end{itemize}

    \begin{tcolorbox}[breakable, size=fbox, boxrule=1pt, pad at break*=1mm,colback=cellbackground, colframe=cellborder]
\prompt{In}{incolor}{ }{\boxspacing}
\begin{Verbatim}[commandchars=\\\{\}]

\end{Verbatim}
\end{tcolorbox}

    \(\textbf{Conditional probability}\)

    Η συνθήκη πιθανότητας είναι μια έννοια στην πιθανοτική που αναφέρεται
στην πιθανότητα ενός συμβάντος υπό συγκεκριμένες συνθήκες ή
προϋποθέσεις. Συνήθως συμβολίζεται ως P(A\textbar B), που διαβάζεται ``η
πιθανότητα του συμβάντος Α δεδομένου του συμβάντος Β''.

Για να υπολογίσουμε τη συνθήκη πιθανότητας, χρησιμοποιούμε τον τύπο:

$ P(A\textbar B) = \frac{P(A \cap B)}{P(B)} $

Παράδειγμα:

Ένα τυχαίο πείραμα αφορά τη ρίψη ενός ζαριού. Έστω ότι: - Συμβάν Α: Το
ζάρι εμφανίζει έναν περιττό αριθμό (1, 3, 5). - Συμβάν Β: Το ζάρι
εμφανίζει έναν αριθμό μεγαλύτερο από 2.

Τώρα, ας υπολογίσουμε τη συνθήκη πιθανότητας του συμβάντος Α δεδομένου
του συμβάντος Β.

Έχουμε: - Πιθανότητα του συμβάντος Α και του συμβάντος Β ( $ P(A
\cap B) $ ): Η πιθανότητα του ζαριού να είναι περιττός αριθμός και
ταυτόχρονα να είναι μεγαλύτερο από 2. Αυτό συμβαίνει μόνο όταν το ζάρι
εμφανίζει τους αριθμούς 3 ή 5, οπότε $ P(A \cap B) = \frac{2}{6} =
\frac{1}{3} $. - Πιθανότητα του συμβάντος Β ( $ P(B) $ ): Η
πιθανότητα του ζαριού να είναι μεγαλύτερο από 2, που είναι $ P(B) =
\frac{4}{6} = \frac{2}{3} $.

Εφαρμόζοντας τον τύπο της συνθήκης πιθανότητας, έχουμε:

$ P(A\textbar B) = \frac{P(A \cap B)}{P(B)} =
\frac{\frac{1}{3}}{\frac{2}{3}} = \frac{1}{2} $

Άρα, η συνθήκη πιθανότητας του συμβάντος Α δεδομένου του συμβάντος Β
είναι 1/2.

    \begin{tcolorbox}[breakable, size=fbox, boxrule=1pt, pad at break*=1mm,colback=cellbackground, colframe=cellborder]
\prompt{In}{incolor}{ }{\boxspacing}
\begin{Verbatim}[commandchars=\\\{\}]

\end{Verbatim}
\end{tcolorbox}

    \(\textbf{Multiplication rule}\)

    Ο κανόνας του πολλαπλασιασμού είναι ένας βασικός κανόνας στη θεωρία
πιθανοτήτων που χρησιμοποιείται για τον υπολογισμό της συνολικής
πιθανότητας δύο ή περισσότερων συμβάντων που συμβαίνουν μεταξύ τους.

Ο κανόνας του πολλαπλασιασμού μπορεί να εκφραστεί ως εξής:

$ P(A \cap B) = P(A) \cdot P(B\textbar A) $

Όπου: - $ P(A \cap B) $ είναι η πιθανότητα του συμβάντος που
συμβαίνουν τα συμβάντα A και B ταυτόχρονα. - $ P(A) $ είναι η
πιθανότητα του συμβάντος A. - $ P(B\textbar A) $ είναι η πιθανότητα
του συμβάντος B δεδομένου του συμβάντος A.

Παράδειγμα:

Ας θεωρήσουμε ένα τυχαίο πείραμα με ένα κανονικό τράπουλα χαρτιών.

Έστω ότι Α είναι το συμβάν που επιλέγετε ένα άσσο, και Β είναι το συμβάν
που επιλέγετε ένα χρυσό χρώμα.

Η πιθανότητα να επιλεγεί ένας άσσος είναι $ P(A) = \frac{4}{52} =
\frac{1}{13} $, καθώς ένα τράπουλο έχει συνολικά 52 κάρτες και 4 από
αυτές είναι άσσοι.

Η πιθανότητα να επιλεγεί ένας άσσος με χρυσό χρώμα είναι $
P(B\textbar A) = \frac{2}{4} = \frac{1}{2} $, καθώς υπάρχουν 2 χρυσοί
άσσοι ανάμεσα στους συνολικούς 4 άσσους.

Έτσι, χρησιμοποιώντας τον κανόνα του πολλαπλασιασμού:

$ P(A \cap B) = P(A) \cdot P(B\textbar A) = \frac{1}{13}
\cdot \frac{1}{2} = \frac{1}{26} $

Άρα, η πιθανότητα να επιλεχθεί ένας χρυσός άσσος είναι $ \frac{1}{26}
$.

    \begin{tcolorbox}[breakable, size=fbox, boxrule=1pt, pad at break*=1mm,colback=cellbackground, colframe=cellborder]
\prompt{In}{incolor}{ }{\boxspacing}
\begin{Verbatim}[commandchars=\\\{\}]

\end{Verbatim}
\end{tcolorbox}

    \(\textbf{Independence}\)

    Η ανεξαρτησία είναι ένας σημαντικός έννοια στη θεωρία πιθανοτήτων και
αφορά τη σχέση μεταξύ δύο συμβάντων. Δύο συμβάντα θεωρούνται ανεξάρτητα
όταν η εμφάνιση ή μη εμφάνιση του ενός δεν επηρεάζει την πιθανότητα
εμφάνισης του άλλου.

Για να δείξουμε ότι δύο συμβάντα είναι ανεξάρτητα, πρέπει να ισχύει η
εξίσωση:

$ P(A \cap B) = P(A) \cdot P(B) $

Εάν η παραπάνω σχέση ισχύει, τότε τα συμβάντα A και B θεωρούνται
ανεξάρτητα.

Παράδειγμα:

Ας θεωρήσουμε ένα κέρμα. Ας ορίσουμε τα εξής συμβάντα: - Συμβάν A:
Εμφάνιση κεφαλής. - Συμβάν B: Εμφάνιση γραμμής.

Ένα κέρμα είναι ένα συμβάν με δύο πιθανά αποτελέσματα: κεφαλή ή γραμμή.
Εάν το κέρμα είναι τίμιο, η πιθανότητα να εμφανίσει κεφαλή είναι $ P(A)
= \frac{1}{2} $, και η πιθανότητα να εμφανίσει γραμμή είναι επίσης $
P(B) = \frac{1}{2} $.

Τώρα, ας θεωρήσουμε το συμβάν C: Εμφάνιση κεφαλής και γραμμής
ταυτόχρονα. Αν το κέρμα είναι τίμιο, τότε η πιθανότητα να εμφανίσει
ταυτόχρονα κεφαλή και γραμμή είναι $ P(C) = \frac{1}{4} $, καθώς αυτό
συμβαίνει μόνο σε ένα από τα τέσσερα πιθανά αποτελέσματα.

Τώρα, αν οι συμβάντα A και B είναι ανεξάρτητα, τότε:

$ P(A \cap B) = P(A) \cdot P(B) $ $ P(C) = P(A) \cdot P(B) $ $
\frac{1}{4} = \frac{1}{2} \cdot \frac{1}{2} $

Επομένως, τα συμβάντα A και B είναι ανεξάρτητα.

    \begin{tcolorbox}[breakable, size=fbox, boxrule=1pt, pad at break*=1mm,colback=cellbackground, colframe=cellborder]
\prompt{In}{incolor}{ }{\boxspacing}
\begin{Verbatim}[commandchars=\\\{\}]

\end{Verbatim}
\end{tcolorbox}

    \(\textbf{The Bayes’ Theorem}\)

    Η θεώρηση του Bayes είναι ένας σημαντικός κανόνας στη θεωρία πιθανοτήτων
που χρησιμοποιείται για τον υπολογισμό της πιθανότητας ενός συμβάντος
υπό την προϋπόθεση ότι ένα άλλο συμβάν έχει συμβεί.

Bayes' Theorem (Θεώρηση του Bayes):

Ο Θεώρηση του Bayes δίνεται από τον ακόλουθο τύπο:

$ P(A\textbar B) = \frac{P(B|A) \cdot P(A)}{P(B)} $

Όπου: - $ P(A\textbar B) $ είναι η πιθανότητα του συμβάντος A
δεδομένου του συμβάντος B. - $ P(B\textbar A) $ είναι η πιθανότητα του
συμβάντος B δεδομένου του συμβάντος A. - $ P(A) $ και $ P(B) $ είναι
οι ανεξάρτητες πιθανότητες των συμβάντων A και B αντίστοιχα.

Παράδειγμα:

Έστω ότι σε μια πόλη, το 2\% του πληθυσμού έχει μία συγκεκριμένη
ασθένεια. Ένα τεστ για αυτή την ασθένεια είναι 90\% ακριβές όταν κάποιος
πράγματι έχει την ασθένεια, και 95\% ακριβές όταν κάποιος δεν έχει την
ασθένεια.

Έστω ότι ένας άνθρωπος πέρασε το τεστ και βρέθηκε θετικός. Ποια είναι η
πιθανότητα που πραγματικά να έχει την ασθένεια;

Ας ορίσουμε τα εξής: - Συμβάν A: Ένα άτομο έχει την ασθένεια. - Συμβάν
B: Το τεστ είναι θετικό.

Με βάση τα δεδομένα:

\begin{itemize}
\tightlist
\item
  $ P(A) = 0.02 $ (2\% του πληθυσμού έχει την ασθένεια).
\item
  $ P(B\textbar A) = 0.90 $ (το τεστ είναι 90\% ακριβές όταν κάποιος
  έχει την ασθένεια).
\item
  $ P(B\textbar{}\neg A) = 0.05 $ (το τεστ είναι 95\% ακριβές όταν
  κάποιος δεν έχει την ασθένεια).
\end{itemize}

Πρώτα υπολογίζουμε την $ P(B) $ με βάση τον νόμο του συνολικού
πιθανοτικού διασταυρώματος (Law of Total Probability):

$ P(B) = P(B\textbar A) \cdot P(A) + P(B\textbar{}\neg A)
\cdot P(\neg A) $

$ P(B) = 0.90 \times 0.02 + 0.05 \times (1-0.02) $ $ P(B) = 0.018 +
0.941 = 0.959 $

Τώρα, χρησιμοποιώντας τον θεώρηση του Bayes:

$ P(A\textbar B) = \frac{P(B|A) \cdot P(A)}{P(B)} $ 
$ P(A\textbar B)= \frac{0.90 \times 0.02}{0.959} $ 
$ P(A\textbar B) \approx \frac{0.018}{0.959} \approx 0.0188 $

Επομένως, η πιθανότητα ότι ο άνθρωπος πραγματικά έχει την ασθένεια όταν
το τεστ είναι θετικό είναι περίπου 0.0188, ή 1.88\%.

    \begin{tcolorbox}[breakable, size=fbox, boxrule=1pt, pad at break*=1mm,colback=cellbackground, colframe=cellborder]
\prompt{In}{incolor}{ }{\boxspacing}
\begin{Verbatim}[commandchars=\\\{\}]

\end{Verbatim}
\end{tcolorbox}

    \(\textbf{The base rate fallacy}\)

    Η βασική λάθος τιμή (base rate fallacy) αναφέρεται σε ένα συγκεκριμένο
είδος σφάλματος που μπορεί να συμβεί όταν οι άνθρωποι αξιολογούν την
πιθανότητα ενός συμβάντος χωρίς να λαμβάνουν υπόψη τη βασική συχνότητα ή
πιθανότητα του συμβάντος στην αρχή.

Ένα συνηθισμένο παράδειγμα είναι η εκτίμηση της πιθανότητας κάποιου να
έχει μια ασθένεια με βάση το θετικό αποτέλεσμα ενός δοκιμαστικού τεστ.
Για παράδειγμα, ας θεωρήσουμε μια ασθένεια που εμφανίζεται σε 1 στους
1000 ανθρώπους (βασική συχνότητα 0.001 ή 0.1\%). Ένα δοκιμαστικό τεστ
για αυτή την ασθένεια είναι 99\% ακριβές όταν κάποιος πράγματι την έχει,
και 95\% ακριβές όταν κάποιος δεν την έχει.

Ένας άνθρωπος κάνει το δοκιμαστικό τεστ και παίρνει θετικό αποτέλεσμα.
Τώρα, ποια είναι η πιθανότητα πραγματικά να έχει αυτή την ασθένεια;

Για να αντιμετωπιστεί αυτή η κατάσταση με βάση το θεώρημα του Bayes,
πρέπει να λάβουμε υπόψη την βασική συχνότητα και την ακρίβεια του τεστ.

Ας υπολογίσουμε την πιθανότητα ότι ο άνθρωπος πραγματικά έχει την
ασθένεια, δεδομένου ότι το τεστ είναι θετικό:

\begin{itemize}
\tightlist
\item
  $ P(A) = 0.001 $ (βασική συχνότητα της ασθένειας).
\item
  $ P(B\textbar A) = 0.99 $ (το τεστ είναι 99\% ακριβές όταν κάποιος
  πράγματι έχει την ασθένεια).
\end{itemize}

Αφού υπολογίσουμε τη συνολική πιθανότητα του θετικού τεστ
(χρησιμοποιώντας τον νόμο του συνολικού πιθανοτικού διασταυρώματος),
μπορούμε να χρησιμοποιήσουμε το θεώρημα του Bayes για να υπολογίσουμε
την επιθυμητή πιθανότητα:

$ P(B) = P(B\textbar A) \cdot P(A) + P(B\textbar{}\neg A)
\cdot P(\neg A) $ $ P(B) = 0.99 \times 0.001 + 0.05 \times (1-0.001)
$ $ P(B) = 0.001 \times 0.99 + 0.999 \times 0.05 = 0.00199 + 0.04995 =
0.05194 $

$ P(A\textbar B) = \frac{P(B|A) \cdot P(A)}{P(B)} $ $ P(A\textbar B)
= \frac{0.99 \times 0.001}{0.05194} $ $ P(A\textbar B) \approx
\frac{0.00099}{0.05194} \approx 0.01905 $

Επομένως, η πιθανότητα ότι ο άνθρωπος

πραγματικά έχει την ασθένεια όταν το τεστ είναι θετικό είναι περίπου
0.019 ή 1.9\%. Αυτό είναι ένα παράδειγμα της βασικής λάθος τιμής, όπου
οι άνθρωποι τείνουν να υποεκτιμούν την πιθανότητα του συμβάντος εξαιτίας
της αγνοούμενης βασικής συχνότητας.

    \begin{tcolorbox}[breakable, size=fbox, boxrule=1pt, pad at break*=1mm,colback=cellbackground, colframe=cellborder]
\prompt{In}{incolor}{ }{\boxspacing}
\begin{Verbatim}[commandchars=\\\{\}]

\end{Verbatim}
\end{tcolorbox}

    \(\textbf{Discrete Random Variables}\)

    Οι διακριτές τυχαίες μεταβλητές αποτελούν ένα σημαντικό μέρος της
θεωρίας πιθανοτήτων και της στατιστικής. Μια διακριτή τυχαία μεταβλητή
αντιπροσωπεύει έναν αριθμό ο οποίος προκύπτει από ένα τυχαίο πείραμα και
μπορεί να πάρει μια συγκεκριμένη τιμή από ένα πεπερασμένο σύνολο δυνατών
τιμών.

Για παράδειγμα, αν ρίξουμε ένα ζάρι, η διακριτή τυχαία μεταβλητή είναι
το αποτέλεσμα της ρίψης του ζαριού, το οποίο μπορεί να είναι ένας
αριθμός από 1 έως 6.

Ορισμένες σημαντικές έννοιες που σχετίζονται με τις διακριτές τυχαίες
μεταβλητές περιλαμβάνουν:

\begin{enumerate}
\def\labelenumi{\arabic{enumi}.}
\item
  \textbf{Συνάρτηση πιθανότητας (Probability Mass Function - PMF)}: Η
  συνάρτηση πιθανότητας μιας διακριτής τυχαίας μεταβλητής δίνει την
  πιθανότητα να πάρει κάθε πιθανή τιμή.
\item
  \textbf{Αναμενόμενη τιμή (Expected Value)}: Είναι η μέση τιμή μιας
  διακριτής τυχαίας μεταβλητής και υπολογίζεται ως η συνολική αθροιστική
  των πιθανοτήτων επί των αντίστοιχων τιμών.
\item
  \textbf{Διασπορά (Variance)}: Η διασπορά μιας διακριτής τυχαίας
  μεταβλητής μετρά την απόκλιση των τιμών από την αναμενόμενη τιμή.
\end{enumerate}

Ένα παράδειγμα διακριτής τυχαίας μεταβλητής μπορεί να είναι το
αποτέλεσμα ενός πειράματος με ένα νόμισμα, όπου μπορεί να πάρει τις
τιμές 0 (κορώνα) ή 1 (γράμμα). Η συνάρτηση πιθανότητας θα έδινε την
πιθανότητα του κάθε αποτελέσματος, ενώ η αναμενόμενη τιμή θα ήταν η
πιθανότητα εμφάνισης κάθε αποτελέσματος πολλαπλασιασμένη με την
αντίστοιχη τιμή, δηλαδή (0 * πιθανότητα της κορώνας) + (1 * πιθανότητα
της γραμμής).

    \begin{tcolorbox}[breakable, size=fbox, boxrule=1pt, pad at break*=1mm,colback=cellbackground, colframe=cellborder]
\prompt{In}{incolor}{ }{\boxspacing}
\begin{Verbatim}[commandchars=\\\{\}]

\end{Verbatim}
\end{tcolorbox}

    \(\textbf{Expected value}\)

    Η αναμενόμενη τιμή μιας διακριτής τυχαίας μεταβλητής είναι ένας
αριθμητικός δείκτης που δίνει την μέση τιμή ή τον μέσο όρο των
αποτελεσμάτων αυτής της μεταβλητής, λαμβάνοντας υπόψη τις πιθανότητες
των αντίστοιχων αποτελεσμάτων. Η αναμενόμενη τιμή συχνά αναφέρεται ως
μέση τιμή.

Η αναμενόμενη τιμή $ E{[}X{]} $ μιας διακριτής τυχαίας μεταβλητής $ X
$ υπολογίζεται ως η συνολική αθροιστική των πιθανοτήτων επί των
αντίστοιχων τιμών:

$ E{[}X{]} = \sum\_\{i\} x\_i \cdot P(X = x\_i) $

Όπου: - $ x\_i $ είναι μια από τις πιθανές τιμές της μεταβλητής $ X
$, - $ P(X = x\_i) $ είναι η πιθανότητα να λάβουμε την τιμή $ x\_i
$.

Για παράδειγμα, αν έχουμε μια διακριτή τυχαία μεταβλητή που
αντιπροσωπεύει τα αποτελέσματα ενός ρίψου κέρματος, όπου μπορεί να πάρει
τις τιμές 0 (κορώνα) ή 1 (γράμμα), και η πιθανότητα να λάβουμε κορώνα
είναι 0.6 και η πιθανότητα να λάβουμε γράμμα είναι 0.4, τότε η
αναμενόμενη τιμή θα υπολογιζόταν ως εξής:

$ E{[}X{]} = 0 \cdot 0.6 + 1 \cdot 0.4 = 0.4 $

Συνεπώς, η αναμενόμενη τιμή της διακριτής τυχαίας μεταβλητής σε αυτό το
παράδειγμα είναι 0.4.

    \begin{tcolorbox}[breakable, size=fbox, boxrule=1pt, pad at break*=1mm,colback=cellbackground, colframe=cellborder]
\prompt{In}{incolor}{ }{\boxspacing}
\begin{Verbatim}[commandchars=\\\{\}]

\end{Verbatim}
\end{tcolorbox}

    \(\textbf{Random variables}\)

    Ένα παράδειγμα διακριτής τυχαίας μεταβλητής μπορεί να αναφέρεται στο
αποτέλεσμα μιας ρίψης ζαριού. Ένα ζάρι έχει έξι πλευρές, και κάθε πλευρά
έχει έναν αριθμό από 1 έως 6. Ας δούμε δύο παραδείγματα:

\begin{enumerate}
\def\labelenumi{\arabic{enumi}.}
\tightlist
\item
  \textbf{Τυχαία μεταβλητή που αντιπροσωπεύει το αποτέλεσμα μιας μονής
  ρίψης ζαριού:}

  \begin{itemize}
  \tightlist
  \item
    Η τυχαία μεταβλητή $ X $ μπορεί να λάβει τις τιμές 1, 2, 3, 4, 5 ή
    6.
  \item
    Η πιθανότητα $ P(X = x) $ για κάθε $ x $ είναι $ \frac{1}{6}
    $, διότι κάθε πλευρά έχει ίση πιθανότητα εμφάνισης.
  \item
    Η αναμενόμενη τιμή $ E{[}X{]} $ υπολογίζεται ως εξής: $ E{[}X{]}
    = 1 \times \frac{1}{6} + 2 \times \frac{1}{6} + 3 \times \frac{1}{6}
    + 4 \times \frac{1}{6} + 5 \times \frac{1}{6} + 6 \times \frac{1}{6}
    = \frac{21}{6} = 3.5 $
  \item
    Συνεπώς, η αναμενόμενη τιμή του αποτελέσματος μιας μονής ρίψης
    ζαριού είναι 3.5.
  \end{itemize}
\item
  \textbf{Τυχαία μεταβλητή που αντιπροσωπεύει το άθροισμα δύο ρίψεων
  ζαριού:}

  \begin{itemize}
  \tightlist
  \item
    Έστω $ Y $ η τυχαία μεταβλητή που αντιπροσωπεύει το άθροισμα δύο
    ρίψεων ζαριού.
  \item
    Η τυχαία μεταβλητή $ Y $ μπορεί να πάρει τις τιμές από 2 έως 12
    (καθώς το άθροισμα δύο ρίψεων ζαριού είναι 2 έως 12).
  \item
    Οι πιθανότητες για κάθε τιμή του $ Y $ μπορούν να υπολογιστούν από
    τον τρόπο με τον οποίο είναι δυνατό να επιτευχθεί αυτή η τιμή
    συνδυάζοντας τις πιθανότητες του κάθε αποτελέσματος του πρώτου και
    του δεύτερου ρίψη.
  \item
    Η αναμενόμενη τιμή $ E{[}Y{]} $ μπορεί να υπολογιστεί
    χρησιμοποιώντας την ίδια μέθοδο που χρησιμοποιήθηκε για το $ X $,
    με τη διαφορά ότι θα αντικαταστήσουμε τις τιμές του $ x $ με τις
    τιμές του $ Y $.
  \item
    Για αυτό το παράδειγμα, η αναμενόμενη τιμή του αθροίσματος δύο
    ρίψεων ζαριού είναι επίσης 7.
  \end{itemize}
\end{enumerate}

    \begin{tcolorbox}[breakable, size=fbox, boxrule=1pt, pad at break*=1mm,colback=cellbackground, colframe=cellborder]
\prompt{In}{incolor}{ }{\boxspacing}
\begin{Verbatim}[commandchars=\\\{\}]

\end{Verbatim}
\end{tcolorbox}

    \(\textbf{Variance}\)

    Η διασπορά μιας διακριτής τυχαίας μεταβλητής μετρά το βαθμό που οι τιμές
της μεταβλητής αποκλίνουν από την αναμενόμενη τιμή της. Για μια διακριτή
τυχαία μεταβλητή $ X $ με αναμενόμενη τιμή $ E{[}X{]} $, η διασπορά
$ Var(X) $ υπολογίζεται ως η συνολική αθροιστική διαφορά των
τετραγώνων της απόκλισης των τιμών της από τη μέση τιμή, καθώς
πολλαπλασιάζονται με την αντίστοιχη πιθανότητα, δηλαδή:

$ Var(X) = E{[}(X - E{[}X{]})^2{]} = \sum\_\{i\} (x\_i - E{[}X{]})^2 \cdot P(X = x\_i) $

Όπου: - $ x\_i $ είναι μια από τις πιθανές τιμές της μεταβλητής $ X
$, - $ P(X = x\_i) $ είναι η πιθανότητα να λάβουμε την τιμή $ x\_i
$.

Για παράδειγμα, αν έχουμε μια διακριτή τυχαία μεταβλητή που
αντιπροσωπεύει το αποτέλεσμα μιας μονής ρίψης ζαριού, όπως προηγουμένως,
η διασπορά μπορεί να υπολογιστεί ως εξής:

$ Var(X) = \sum\_\{i\} (x\_i - E{[}X{]})^2 \cdot P(X = x\_i) $

Όπου $ E{[}X{]} = 3.5 $ είναι η αναμενόμενη τιμή της μεταβλητής $ X
$.

Ας υποθέσουμε ότι η πιθανότητα να πάρουμε κάθε πιθανή τιμή του ζαριού
είναι ίση (κάθε πλευρά του ζαριού έχει πιθανότητα 1/6). Τότε η διασπορά
θα είναι:

$ Var(X) = (1 - 3.5)^2 \cdot \frac{1}{6} + (2 - 3.5)^2
\cdot \frac{1}{6} + \ldots + (6 - 3.5)^2 \cdot \frac{1}{6} $

Και μετά μπορούμε να υπολογίσουμε το αποτέλεσμα.

\begin{center}\rule{0.5\linewidth}{0.5pt}\end{center}

Ας υποθέσουμε ότι έχουμε ένα σύνολο δεδομένων που αντιπροσωπεύει τα
ηλικιακά έτη των μαθητών σε ένα σχολείο:

$ \{12, 13, 14, 15, 16\} $

Για να υπολογίσουμε τη διασπορά των ηλικιών αυτών των μαθητών, πρέπει να
ακολουθήσουμε τα παρακάτω βήματα:

\begin{enumerate}
\def\labelenumi{\arabic{enumi}.}
\item
  \textbf{Υπολογισμός της μέσης τιμής:} $ \mu =
  \frac{12 + 13 + 14 + 15 + 16}{5} = \frac{70}{5} = 14 $
\item
  \textbf{Υπολογισμός της απόκλισης από τη μέση τιμή για κάθε τιμή:} $
  (12 - 14)^2 = 4, ~(13 - 14)^2 = 1, ~(14 - 14)^2 = 0, ~(15 -
  14)^2 = 1, ~(16 - 14)^2 = 4 $
\item
  \textbf{Υπολογισμός του μέσου όρου των τετραγώνων των αποκλίσεων:} $
  \text{Διασπορά} = \frac{4 + 1 + 0 + 1 + 4}{5} = \frac{10}{5} = 2 $
\end{enumerate}

Άρα η διασπορά των ηλικιών είναι 2.

Η τυπική απόκλιση είναι η τετραγωνική ρίζα της διασποράς, οπότε: $
\text{Τυπική Απόκλιση} = \sqrt{2} \approx 1.41 $

Άρα η τυπική απόκλιση των ηλικιών είναι περίπου 1.41 έτη.

    \begin{tcolorbox}[breakable, size=fbox, boxrule=1pt, pad at break*=1mm,colback=cellbackground, colframe=cellborder]
\prompt{In}{incolor}{ }{\boxspacing}
\begin{Verbatim}[commandchars=\\\{\}]

\end{Verbatim}
\end{tcolorbox}

    \(\textbf{Variance and Standard Deviation}\)

    Η διασπορά (Variance) και η τυπική απόκλιση (Standard Deviation) είναι
μέτρα που χρησιμοποιούνται για να περιγράψουν το βαθμό διασποράς ή
διακύμανσης μιας σύνολης δεδομένων ή μιας τυχαίας μεταβλητής.

\begin{enumerate}
\def\labelenumi{\arabic{enumi}.}
\tightlist
\item
  \textbf{Διασπορά (Variance):} Η διασπορά είναι η μέση του τετραγώνου
  της απόκλισης των τιμών από την μέση τιμή. Υπολογίζεται ως η
  αναμενόμενη τιμή του τετραγώνου της απόκλισης:
\end{enumerate}

$ \text{Var}(X) = E{[}(X - \mu)^2{]} $

Όπου: - $ X $ είναι η τυχαία μεταβλητή, - $ \mu $ είναι η μέση τιμή
της τυχαίας μεταβλητής, - $ E{[}\cdot{]} $ συμβολίζει την αναμενόμενη
τιμή.

\begin{enumerate}
\def\labelenumi{\arabic{enumi}.}
\setcounter{enumi}{1}
\tightlist
\item
  \textbf{Τυπική Απόκλιση (Standard Deviation):} Η τυπική απόκλιση είναι
  η τετραγωνική ρίζα της διασποράς και παρέχει μια πιο ερμηνεύσιμη
  μέτρηση της διασποράς. Υπολογίζεται ως εξής:
\end{enumerate}

$ \sigma = \sqrt{\text{Var}(X)} $

Όπου $ \sigma $ συμβολίζει την τυπική απόκλιση.

Η τυπική απόκλιση μας δίνει μια εικόνα για το πόσο διαφορετικές είναι οι
τιμές από τη μέση τιμή, ενώ η διασπορά μας δίνει το μέγεθος αυτής της
διαφοράς σε μια μονάδα που έχει την ίδια μονάδα με την αρχική μεταβλητή.

Σε πολλές περιπτώσεις, η τυπική απόκλιση είναι προτιμότερη γιατί είναι
πιο εύκολη στην ερμηνεία, καθώς έχει την ίδια κλίμακα με την αρχική
μεταβλητή.

Ας δούμε ένα παράδειγμα για τον υπολογισμό της διασποράς και της τυπικής
απόκλισης για ένα σύνολο δεδομένων.

Έστω ότι έχουμε το παρακάτω σύνολο δεδομένων:

$ \{4, 7, 8, 10, 12\} $

Πρώτα, υπολογίζουμε τη μέση τιμή του συνόλου:

$ \text{Μέση Τιμή} (\mu) = \frac{4 + 7 + 8 + 10 + 12}{5} = \frac{41}{5}
= 8.2 $

Τώρα, για να υπολογίσουμε τη διασπορά, πρέπει να βρούμε την απόκλιση
κάθε τιμής από τη μέση τιμή, να την τετραγωνίσουμε, και μετά να
υπολογίσουμε τον μέσο όρο αυτών των τετραγώνων.

Έτσι, έχουμε:

$ \text{Διασπορά} =
\frac{(4-8.2)^2 + (7-8.2)^2 + (8-8.2)^2 + (10-8.2)^2 + (12-8.2)^2}{5} $

$ = \frac{(-4.2)^2 + (-1.2)^2 + (-0.2)^2 + (1.8)^2 + (3.8)^2}{5} $

$ = \frac{17.64 + 1.44 + 0.04 + 3.24 + 14.44}{5} $

$ = \frac{36.8}{5} $

$ = 7.36 $

Άρα η διασπορά είναι \(7.36\).

Τώρα, η τυπική απόκλιση είναι απλώς η τετραγωνική ρίζα της διασποράς:

$ \text{Τυπική Απόκλιση} (\sigma) = \sqrt{7.36} = 2.71 $

Συνεπώς, η τυπική απόκλιση είναι \(2.71\).

    \begin{tcolorbox}[breakable, size=fbox, boxrule=1pt, pad at break*=1mm,colback=cellbackground, colframe=cellborder]
\prompt{In}{incolor}{ }{\boxspacing}
\begin{Verbatim}[commandchars=\\\{\}]

\end{Verbatim}
\end{tcolorbox}

    \(\textbf{Bernoulli distribution}\)

    Η κατανομή Bernoulli είναι μια διακριτή πιθανοτική κατανομή που
περιγράφει ένα πείραμα με δύο δυνατά αποτελέσματα, όπου το κάθε
αποτέλεσμα έχει μια σταθερή πιθανότητα επιτυχίας $ p $ και αποτυχίας
$ 1-p $, εκφρασμένη με τις τιμές 1 και 0 αντίστοιχα.

Συνήθως, συμβολίζουμε το αποτέλεσμα της επιτυχίας με το 1 και το
αποτέλεσμα της αποτυχίας με το 0. Έτσι, μια τυχαία μεταβλητή $ X $ που
ακολουθεί μια κατανομή Bernoulli έχει την ακόλουθη πιθανοτική μάζα:

$ P(X = x) =\begin{cases} p, & \text{εάν } x = 1 \\ 1-p, & \text{εάν } x = 0 \end{cases}$

Όπου $ p $ είναι η πιθανότητα επιτυχίας του πειράματος.

Η αναμενόμενη τιμή $ E{[}X{]} $ μιας τυχαίας μεταβλητής Bernoulli
υπολογίζεται ως η πιθανότητα επιτυχίας $ p $, δηλαδή $ E{[}X{]} = p
$. Η διασπορά $ \text{Var}(X) $ είναι επίσης εύκολο να υπολογιστεί,
ως:

$ \text{Var}(X) = E{[}X^2{]} - (E{[}X{]})^2 = p - p^2 = p(1-p)$

Ένα παράδειγμα χρήσης της κατανομής Bernoulli είναι ένα νόμισμα που
ρίχνεται, όπου το αποτέλεσμα μπορεί να είναι κορώνα (επιτυχία) με
πιθανότητα $ p $ ή γράμμα (αποτυχία) με πιθανότητα $ 1-p $.

    \begin{tcolorbox}[breakable, size=fbox, boxrule=1pt, pad at break*=1mm,colback=cellbackground, colframe=cellborder]
\prompt{In}{incolor}{ }{\boxspacing}
\begin{Verbatim}[commandchars=\\\{\}]

\end{Verbatim}
\end{tcolorbox}

    \(\textbf{Binomial Distribution}\)

    Ένα παράδειγμα χρήσης της διακριτής κατανομής binomial είναι όταν έχουμε
ένα σετ από ανεξάρτητα πειράματα, κάθε ένα από τα οποία έχει δύο πιθανά
αποτελέσματα (επιτυχία ή αποτυχία), και θέλουμε να υπολογίσουμε την
πιθανότητα εμφάνισης μιας συγκεκριμένης αριθμητικής ποσότητας επιτυχιών
σε ένα συγκεκριμένο αριθμό πειραμάτων.

Για παράδειγμα, ας υποθέσουμε ότι έχουμε ένα ένα νόμισμα που το ρίχνουμε
10 φορές και θέλουμε να υπολογίσουμε την πιθανότητα να πάρουμε 7 φορές
κορώνα (επιτυχίες) αν γνωρίζουμε ότι η πιθανότητα να πάρουμε κορώνα σε
κάθε ρίψη είναι 0.5.

Σε αυτή την περίπτωση, η τυχαία μεταβλητή $ X $, που αντιπροσωπεύει
τον αριθμό των επιτυχιών (δηλαδή τον αριθμό των φορών που πέφτει κορώνα
σε αυτό το πείραμα), ακολουθεί μια κατανομή binomial με παραμέτρους $ n
= 10 $ (τον συνολικό αριθμό των πειραμάτων) και $ p = 0.5 $ (η
πιθανότητα επιτυχίας σε κάθε πείραμα).

Για να υπολογίσουμε αυτή την πιθανότητα, μπορούμε να χρησιμοποιήσουμε
την τύπο της κατανομής binomial:

$ P(X = k) = \binom{n}{k} \cdot p^k \cdot (1-p)^\{n-k\} $

όπου $ k $ είναι ο αριθμός των επιτυχιών που θέλουμε να πάρουμε (σε
αυτή την περίπτωση, $ k = 7 $), $ n $ είναι ο συνολικός αριθμός των
πειραμάτων (σε αυτή την περίπτωση, $ n = 10 $), και $ p $ είναι η
πιθανότητα επιτυχίας σε κάθε πείραμα (σε αυτή την περίπτωση, $ p = 0.5
$).

Έτσι, η πιθανότητα να πάρουμε 7 φορές κορώνα από τις 10 φορές που
ρίχνουμε το νόμισμα είναι:

$ P(X = 7) = \binom{10}{7} \cdot 0.5^7 \cdot (1-0.5)^\{10-7\} $

Μπορούμε να υπολογίσουμε αυτή την τιμή και να βρούμε την απάντηση.

    \begin{tcolorbox}[breakable, size=fbox, boxrule=1pt, pad at break*=1mm,colback=cellbackground, colframe=cellborder]
\prompt{In}{incolor}{ }{\boxspacing}
\begin{Verbatim}[commandchars=\\\{\}]

\end{Verbatim}
\end{tcolorbox}

    \(\textbf{Continuous Random Variables }\)

    Οι συνεχείς τυχαίες μεταβλητές αντιπροσωπεύουν μεγέθη που μπορούν να
πάρουν οποιαδήποτε τιμή σε ένα διάστημα, συνήθως ένα διάστημα
πραγματικών αριθμών. Διαφορετικά από τις διακριτές τυχαίες μεταβλητές
που μπορούν να πάρουν μόνο συγκεκριμένες τιμές, οι συνεχείς τυχαίες
μεταβλητές μπορούν να πάρουν οποιαδήποτε τιμή εντός ενός διαστήματος των
πραγματικών αριθμών.

Για παράδειγμα, η χρονική διάρκεια που απαιτείται για να ολοκληρωθεί ένα
έργο, η ταχύτητα ενός αυτοκινήτου ή η ύψωση ενός βάρους είναι
παραδείγματα συνεχών τυχαίων μεταβλητών.

Για να περιγραφεί μια συνεχής τυχαία μεταβλητή, χρειάζεται μια συνεχής
συνάρτηση πυκνότητας πιθανότητας (probability density function - PDF). Η
PDF δεν δίνει ακριβώς την πιθανότητα να λάβει μια συγκεκριμένη τιμή,
αλλά τη συνάρτηση πυκνότητας της πιθανότητας σε μια περιοχή των τιμών. Η
πιθανότητα να πέσει η τιμή μέσα σε μια συγκεκριμένη περιοχή είναι η
έκθεση της περιοχής αυτής υπό την καμπύλη της PDF.

Για παράδειγμα, μια συνεχής τυχαία μεταβλητή μπορεί να είναι η ύψωση
ενός βάρους από έναν αθλητή. Η PDF μπορεί να δείχνει πόσο πιθανό είναι
να είναι η ύψωση ενός συγκεκριμένου βάρους εντός ενός διαστήματος (π.χ.
80 κιλά έως 90 κιλά). Η πιθανότητα να είναι η ύψωση ακριβώς 85 κιλά δεν
είναι ίση με την τιμή της PDF στο σημείο 85 κιλών, αλλά είναι η έκθεση
της περιοχής 85 κιλών

\begin{center}\rule{0.5\linewidth}{0.5pt}\end{center}

Ένα παράδειγμα συνεχούς τυχαίας μεταβλητής είναι η ύψωση ενός βάρους από
έναν αθλητή. Έστω ότι ο αθλητής μπορεί να υψώσει ένα βάρος ανάμεσα σε 0
και 100 κιλά, και η κατανομή του ύψους που μπορεί να υψώσει αυτός ο
αθλητής είναι γνωστή.

Έστω ότι η πιθανότητα να υψώσει το βάρος Χ κιλά από τον αθλητή δίνεται
από την παρακάτω πυκνότητα πιθανότητας (PDF):

$ f(x) = \frac{1}{100} $

Αυτό σημαίνει ότι η πιθανότητα να υψώσει το βάρος Χ κιλά είναι ίση με το
$ \frac{1}{100} $ για κάθε κιλό στο διάστημα {[}0, 100{]}.

Για παράδειγμα, ας υπολογίσουμε την πιθανότητα να υψώσει το βάρος
μικρότερο από 50 κιλά. Αυτή η πιθανότητα δίνεται από τη συναρτηση
συσσώρευσης κατανομής (CDF):

$ F(x) = \int\_\{-\infty\}^\{x\} f(t) , dt $

Στην περίπτωσή μας, επειδή η πυκνότητα πιθανότητας είναι σταθερή,
μπορούμε να απλοποιήσουμε τον υπολογισμό:

$ F(x) = \int\_\{0\}^\{x\} \frac{1}{100} , dt $

$ F(x) = \left. \frac{t}{100} \right|_{0}^{x} $

$ F(x) = \frac{x}{100} $

Επομένως, η πιθανότητα να υψώσει το βάρος μικρότερο από 50 κιλά είναι:

$ F(50) = \frac{50}{100} = 0.5 $

Άρα η πιθανότητα είναι 0.5 ή 50\%.

    \begin{tcolorbox}[breakable, size=fbox, boxrule=1pt, pad at break*=1mm,colback=cellbackground, colframe=cellborder]
\prompt{In}{incolor}{ }{\boxspacing}
\begin{Verbatim}[commandchars=\\\{\}]

\end{Verbatim}
\end{tcolorbox}

    \(\textbf{Cumulative Distribution Function (CDF) and most important properties}\)

    Η Συναρτηση Συσσωρευτικής Κατανομής (Cumulative Distribution Function -
CDF) είναι μια συνάρτηση που παρέχει την πιθανότητα να είναι μια τυχαία
μεταβλητή μικρότερη ή ίση με ένα συγκεκριμένο όριο. Συγκεκριμένα, η CDF
της τυχαίας μεταβλητής $ X $, που συμβολίζεται με $ F(x) $, ορίζεται
ως εξής:

$ F(x) = P(X \leq x) $

Με άλλα λόγια, η CDF μετράει τη συγκεντρωτική πιθανότητα ότι η τυχαία
μεταβλητή θα πάρει μια τιμή μικρότερη ή ίση με ένα δεδομένο όριο $ x
$.

Οι πιο σημαντικές ιδιότητες της CDF είναι οι εξής:

\begin{enumerate}
\def\labelenumi{\arabic{enumi}.}
\item
  \textbf{Μονοτονία:} Η CDF είναι πάντα μη φθίνουσα, δηλαδή όσο
  αυξάνεται η τιμή της τυχαίας μεταβλητής, τόσο αυξάνεται ή παραμένει
  σταθερή η τιμή της CDF.
\item
  \textbf{Πεπερασμένοι Ορίζοντες:} Η τιμή της CDF βρίσκεται πάντα μεταξύ
  του 0 και του 1, δηλαδή $ 0 \leq F(x) \leq 1 $ για κάθε $ x $.
\item
  \textbf{Σύγκλιση στα Άκρα:} Καθώς η τιμή της τυχαίας μεταβλητής
  πλησιάζει το άπειρο, η τιμή της CDF σύγκλινει προς το 1.
\item
  \textbf{Σύγκλιση στο 0:} Καθώς η τιμή της τυχαίας μεταβλητής πλησιάζει
  τον αρνητικό άπειρο, η τιμή της CDF σύγκλινει προς το 0.
\end{enumerate}

Αυτές οι ιδιότητες καθιστούν την CDF ένα ισχυρό εργαλείο για την ανάλυση
των συνεχών τυχαίων μεταβλητών.

    \begin{tcolorbox}[breakable, size=fbox, boxrule=1pt, pad at break*=1mm,colback=cellbackground, colframe=cellborder]
\prompt{In}{incolor}{ }{\boxspacing}
\begin{Verbatim}[commandchars=\\\{\}]

\end{Verbatim}
\end{tcolorbox}

    \(\textbf{The uniform distribution}\)

    Η ομοιόμορφη κατανομή είναι μια συνεχής πιθανοτική κατανομή όπου κάθε
τιμή εντός ενός συγκεκριμένου διαστήματος έχει την ίδια πιθανότητα
εμφάνισης. Αυτό σημαίνει ότι η πυκνότητα πιθανότητας είναι σταθερή σε
ολόκληρο το διάστημα και μηδενίζεται έξω από αυτό.

Η ομοιόμορφη κατανομή συνήθως συμβολίζεται ως $ U(a, b) $, όπου $ a
$ και $ b $ είναι τα άκρα του διαστήματος. Η πυκνότητα πιθανότητας $
f(x) $ για μια τυχαία μεταβλητή που ακολουθεί μια ομοιόμορφη κατανομή
στο διάστημα {[}a, b{]} ορίζεται ως:

$ f(x) = \frac{1}{b - a} $

για $ a \leq x \leq b $ και $ 0 $ για $ x \textless{} a $ ή $ x
\textgreater{} b $.

Με άλλα λόγια, κάθε τιμή μέσα στο διάστημα {[}a, b{]} έχει μια ίση
πιθανότητα εμφάνισης, ενώ εκτός αυτού το πιθανότερο είναι μηδέν.

Ένα παράδειγμα ομοιόμορφης κατανομής είναι όταν ρίχνουμε ένα ζάρι. Αν
έχουμε μια δίκαιη έξιπλή ζαριά, κάθε πλευρά του ζαριού έχει την ίδια
πιθανότητα εμφάνισης (1/6) κατά τη ρίψη του ζαριού. Άρα, αυτή η κατανομή
είναι ένα παράδειγμα ομοιόμορφης κατανομής.

    \begin{tcolorbox}[breakable, size=fbox, boxrule=1pt, pad at break*=1mm,colback=cellbackground, colframe=cellborder]
\prompt{In}{incolor}{ }{\boxspacing}
\begin{Verbatim}[commandchars=\\\{\}]

\end{Verbatim}
\end{tcolorbox}

    \(\textbf{The normal distribution}\)

    Η κανονική κατανομή, γνωστή επίσης ως κατανομή Gauss ή καμπάνας, είναι
μια συνεχής πιθανοτική κατανομή που είναι πολύ σημαντική στη στατιστική
λόγω των ευρείας εφαρμογών της. Η κανονική κατανομή χαρακτηρίζεται από
τη μορφή μιας καμπάνας με συμμετρική κατανομή γύρω από τη μέση τιμή της.

Η πυκνότητα πιθανότητας της κανονικής κατανομής δίνεται από την εξίσωση:

$ f(x) = \frac{1}{\sqrt{2\pi\sigma^2}}
e^\{-\frac{(x-\mu)^2}{2\sigma^2}\} $

όπου: - $ \mu $ είναι η μέση τιμή της κατανομής (συμβολίζει το κέντρο
της καμπάνας), - $ \sigma $ είναι η τυπική απόκλιση της κατανομής (που
ελέγχει το πλάτος της καμπάνας).

Η κανονική κατανομή είναι συμμετρική γύρω από τη μέση τιμή και έχει τη
μορφή μιας καμπάνας. Το 68\% των παρατηρήσεων βρίσκονται μέσα σε μία
τυπική απόκλιση από τη μέση τιμή, το 95\% βρίσκεται μέσα σε δύο τυπικές
αποκλίσεις, ενώ το 99.7\% βρίσκεται μέσα σε τρεις τυπικές αποκλίσεις.

Η κανονική κατανομή έχει εφαρμογές σε πολλούς τομείς, όπως στη φυσική,
στην οικονομική, στην επιστήμη των υπολογιστών και στην επεξεργασία
σημάτων, λόγω της ιδιότητας της να περιγράφει φυσικά φαινόμενα που
εμφανίζονται συχνά στη φύση.

    \begin{tcolorbox}[breakable, size=fbox, boxrule=1pt, pad at break*=1mm,colback=cellbackground, colframe=cellborder]
\prompt{In}{incolor}{ }{\boxspacing}
\begin{Verbatim}[commandchars=\\\{\}]

\end{Verbatim}
\end{tcolorbox}

    \(\textbf{Standardizing the normal distribution}\)

    Η τυποποίηση της κανονικής κατανομής είναι ένα σημαντικό βήμα που μας
επιτρέπει να μετατρέψουμε τυχαίες μεταβλητές που ακολουθούν μια κανονική
κατανομή σε μια τυπική κανονική κατανομή με μέση τιμή $ \mu = 0 $ και
τυπική απόκλιση $ \sigma = 1 $.

Για να τυποποιήσουμε μια τυχαία μεταβλητή $ X $ που ακολουθεί μια
κανονική κατανομή με μέση τιμή $ \mu $ και τυπική απόκλιση $
\sigma $, ακολουθούμε τα παρακάτω βήματα:

\begin{enumerate}
\def\labelenumi{\arabic{enumi}.}
\item
  Υπολογίζουμε το απόκλισμα από τη μέση τιμή για την τυχαία μεταβλητή $
  X $, δηλαδή $ Z = \frac{X - \mu}{\sigma} $.
\item
  Το απόκλισμα $ Z $ αντιστοιχεί στην τυπική κανονική κατανομή με μέση
  τιμή $ \mu = 0 $ και τυπική απόκλιση $ \sigma = 1 $.
\end{enumerate}

Με άλλα λόγια, μετατρέπουμε τις τιμές της αρχικής κανονικής κατανομής σε
αντίστοιχες τιμές στην τυπική κανονική κατανομή με τη χρήση της τύπου $
Z = \frac{X - \mu}{\sigma} $.

Αυτή η διαδικασία επιτρέπει να κανονικοποιήσουμε τις κατανομές,
ευκολαίνοντας τη σύγκριση των αποτελεσμάτων από διαφορετικές κατανομές
και την εφαρμογή στατιστικών τεχνικών.

    \begin{tcolorbox}[breakable, size=fbox, boxrule=1pt, pad at break*=1mm,colback=cellbackground, colframe=cellborder]
\prompt{In}{incolor}{ }{\boxspacing}
\begin{Verbatim}[commandchars=\\\{\}]

\end{Verbatim}
\end{tcolorbox}

    \(\textbf{Covariance and Correlation}\)

    Η κοιναρχία (covariance) και η συσχέτιση (correlation) είναι δύο μέτρα
που χρησιμοποιούνται για να μετρήσουν τη σχέση μεταξύ δύο τυχαίων
μεταβλητών.

Η κοιναρχία (covariance) μετράει τη συσχέτιση των μεταβολών μεταξύ δύο
μεταβλητών. Αν οι τιμές των μεταβλητών σχετίζονται θετικά, τότε η
κοιναρχία είναι θετική. Αν οι τιμές των μεταβλητών σχετίζονται αρνητικά,
τότε η κοιναρχία είναι αρνητική. Αν δεν υπάρχει σχέση μεταξύ των
μεταβλητών, τότε η κοιναρχία είναι κοντά στο μηδέν. Η κοιναρχία
υπολογίζεται με τον ακόλουθο τύπο:

$ \text{cov}(X,Y) =
\frac{\sum_{i=1}^{n}(X_i - \bar{X})(Y_i - \bar{Y})}{n} $

όπου $ X $ και $ Y $ είναι οι μεταβλητές, $ \bar\{X\} $ και $
\bar\{Y\} $ είναι οι μέσες τιμές των μεταβλητών αντίστοιχα, και $ n $
είναι το πλήθος των παρατηρήσεων.

Η συσχέτιση (correlation) είναι μια κανονικοποιημένη μορφή της
κοιναρχίας και μετράει τη δύναμη και τη φορά της γραμμικής σχέσης μεταξύ
δύο μεταβλητών. Η τιμή της συσχέτισης βρίσκεται στο διάστημα {[}-1,1{]}.
Όταν η τιμή της συσχέτισης είναι προς το 1, υπάρχει θετική γραμμική
σχέση, ενώ όταν είναι προς το -1, υπάρχει αρνητική γραμμική σχέση. Όταν
η τιμή της συσχέτισης είναι κοντά στο 0, δεν υπάρχει γραμμική σχέση. Η
συσχέτιση υπολογίζεται ως εξής:

$ \text{corr}(X,Y) = \frac{\text{cov}(X,Y)}{\sigma_X \cdot \sigma_Y} $

όπου $ \sigma\_X $ και $ \sigma\_Y $ είναι οι τυπικές αποκλίσεις των
μεταβλητών $ X $ και $ Y $ αντίστοιχα.

    \begin{tcolorbox}[breakable, size=fbox, boxrule=1pt, pad at break*=1mm,colback=cellbackground, colframe=cellborder]
\prompt{In}{incolor}{ }{\boxspacing}
\begin{Verbatim}[commandchars=\\\{\}]

\end{Verbatim}
\end{tcolorbox}

    \(\textbf{The Central Limit Theorem}\)

    Το Κεντρικό Όριο Θεώρημα (Central Limit Theorem - CLT) είναι ένα από τα
πιο σημαντικά θεωρήματα στη στατιστική και έχει ευρείες εφαρμογές σε
πολλούς τομείς της επιστήμης και της τεχνολογίας. Το θεώρημα αυτό
δηλώνει ότι όταν προσθέτουμε ή αθροίζουμε μια μεγάλη αριθμητική ποσότητα
ανεξάρτητων τυχαίων μεταβλητών, η κατανομή του αθροίσματος πλησιάζει σε
μια κανονική κατανομή, ανεξάρτητα από την αρχική κατανομή των
μεταβλητών.

Με άλλα λόγια, αν λάβουμε ένα μεγάλο δείγμα τυχαίων μεταβλητών από
οποιαδήποτε κατανομή με μέση τιμή $ \mu $ και τυπική απόκλιση $
\sigma $, η κατανομή του μέσου του δείγματος θα πλησιάσει μια κανονική
κατανομή με μέση τιμή $ \mu $ και τυπική απόκλιση $
\frac{\sigma}{\sqrt{n}} $, όπου $ n $ είναι το μέγεθος του δείγματος.

Το Κεντρικό Όριο Θεώρημα έχει πολλές εφαρμογές, συμπεριλαμβανομένης της
εκτίμησης παραμέτρων, του τεστ υποθέσεων και της σχεδίασης πειραμάτων.
Επίσης, εξηγεί γιατί οι κανονικές κατανομές είναι τόσο συχνά
χρησιμοποιούμενες σε στατιστικές αναλύσεις, αφού πολλές φυσικές,
κοινωνικές και οικονομικές μεταβλητές ακολουθούν κατά κάποιο τρόπο
κανονική κατανομή όταν αθροίζονται μεγάλοι αριθμοί ανεξάρτητων επιμέρους
παρατηρήσεων.

    \begin{tcolorbox}[breakable, size=fbox, boxrule=1pt, pad at break*=1mm,colback=cellbackground, colframe=cellborder]
\prompt{In}{incolor}{ }{\boxspacing}
\begin{Verbatim}[commandchars=\\\{\}]

\end{Verbatim}
\end{tcolorbox}

    \(\textbf{Simulations}\)

    Οι προσομοιώσεις είναι μια διαδικασία που χρησιμοποιείται για να
μοντελοποιήσει τη συμπεριφορά συστημάτων ή διεργασιών μέσω της
προσομοίωσης των τυχαίων μεταβλητών. Η διαδικασία αυτή μπορεί να
εφαρμοστεί σε ποικίλους τομείς, όπως η επιστήμη των υπολογιστών, η
οικονομία, η μηχανική, η φυσική, η υγεία και άλλοι.

Οι προσομοιώσεις συνήθως απαιτούν τη χρήση υπολογιστικών προγραμμάτων
και αλγορίθμων για τη δημιουργία τυχαίων συμβάντων, την εκτέλεση των
προσομοιώσεων και την ανάλυση των αποτελεσμάτων. Στις προσομοιώσεις,
μπορούν να προσομοιώνονται διάφορες συνθήκες, σενάρια ή πειράματα, και
να μελετώνται η συμπεριφορά του συστήματος υπό αυτές τις συνθήκες.

Για παράδειγμα, στην επιστήμη των υπολογιστών, μπορεί να προσομοιώνεται
η απόδοση ενός νέου αλγορίθμου αναζήτησης, ενώ στην οικονομία μπορεί να
προσομοιώνεται η επίδραση μιας νέας φορολογικής πολιτικής στην αγορά
εργασίας.

Οι προσομοιώσεις είναι χρήσιμες όταν δεν είναι δυνατή η ανάλυση ενός
συστήματος με αναλυτικές μεθόδους ή όταν η πραγματική εφαρμογή είναι
ακριβή ή αδύνατη. Επίσης, μπορούν να χρησιμοποιηθούν για την εκτίμηση
της αβεβαιότητας ή για την εξαγωγή στατιστικών δεδομένων σε συνθήκες που
είναι δύσκολο να μελετηθούν αναλυτικά.

    \begin{tcolorbox}[breakable, size=fbox, boxrule=1pt, pad at break*=1mm,colback=cellbackground, colframe=cellborder]
\prompt{In}{incolor}{ }{\boxspacing}
\begin{Verbatim}[commandchars=\\\{\}]

\end{Verbatim}
\end{tcolorbox}


    % Add a bibliography block to the postdoc
    
    
    
\end{document}

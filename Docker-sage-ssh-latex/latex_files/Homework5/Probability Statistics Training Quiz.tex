\documentclass[11pt]{article}

    \usepackage[breakable]{tcolorbox}
    \usepackage{parskip} % Stop auto-indenting (to mimic markdown behaviour)
    

    % Basic figure setup, for now with no caption control since it's done
    % automatically by Pandoc (which extracts ![](path) syntax from Markdown).
    \usepackage{graphicx}
    % Maintain compatibility with old templates. Remove in nbconvert 6.0
    \let\Oldincludegraphics\includegraphics
    % Ensure that by default, figures have no caption (until we provide a
    % proper Figure object with a Caption API and a way to capture that
    % in the conversion process - todo).
    \usepackage{caption}
    \DeclareCaptionFormat{nocaption}{}
    \captionsetup{format=nocaption,aboveskip=0pt,belowskip=0pt}

    \usepackage{float}
    \floatplacement{figure}{H} % forces figures to be placed at the correct location
    \usepackage{xcolor} % Allow colors to be defined
    \usepackage{enumerate} % Needed for markdown enumerations to work
    \usepackage{geometry} % Used to adjust the document margins
    \usepackage{amsmath} % Equations
    \usepackage{amssymb} % Equations
    \usepackage{textcomp} % defines textquotesingle
    % Hack from http://tex.stackexchange.com/a/47451/13684:
    \AtBeginDocument{%
        \def\PYZsq{\textquotesingle}% Upright quotes in Pygmentized code
    }
    \usepackage{upquote} % Upright quotes for verbatim code
    \usepackage{eurosym} % defines \euro

    \usepackage{iftex}
    \ifPDFTeX
        \usepackage[T1]{fontenc}
        \IfFileExists{alphabeta.sty}{
              \usepackage{alphabeta}
          }{
              \usepackage[mathletters]{ucs}
              \usepackage[utf8x]{inputenc}
          }
    \else
        \usepackage{fontspec}
        \usepackage{unicode-math}
    \fi

    \usepackage{fancyvrb} % verbatim replacement that allows latex
    \usepackage[Export]{adjustbox} % Used to constrain images to a maximum size
    \adjustboxset{max size={0.9\linewidth}{0.9\paperheight}}

    % The hyperref package gives us a pdf with properly built
    % internal navigation ('pdf bookmarks' for the table of contents,
    % internal cross-reference links, web links for URLs, etc.)
    \usepackage{hyperref}
    % The default LaTeX title has an obnoxious amount of whitespace. By default,
    % titling removes some of it. It also provides customization options.
    \usepackage{titling}
    \usepackage{longtable} % longtable support required by pandoc >1.10
    \usepackage{booktabs}  % table support for pandoc > 1.12.2
    \usepackage{array}     % table support for pandoc >= 2.11.3
    \usepackage{calc}      % table minipage width calculation for pandoc >= 2.11.1
    \usepackage[inline]{enumitem} % IRkernel/repr support (it uses the enumerate* environment)
    \usepackage[normalem]{ulem} % ulem is needed to support strikethroughs (\sout)
                                % normalem makes italics be italics, not underlines
    \usepackage{mathrsfs}
    

    
    % Colors for the hyperref package
    \definecolor{urlcolor}{rgb}{0,.145,.698}
    \definecolor{linkcolor}{rgb}{.71,0.21,0.01}
    \definecolor{citecolor}{rgb}{.12,.54,.11}

    % ANSI colors
    \definecolor{ansi-black}{HTML}{3E424D}
    \definecolor{ansi-black-intense}{HTML}{282C36}
    \definecolor{ansi-red}{HTML}{E75C58}
    \definecolor{ansi-red-intense}{HTML}{B22B31}
    \definecolor{ansi-green}{HTML}{00A250}
    \definecolor{ansi-green-intense}{HTML}{007427}
    \definecolor{ansi-yellow}{HTML}{DDB62B}
    \definecolor{ansi-yellow-intense}{HTML}{B27D12}
    \definecolor{ansi-blue}{HTML}{208FFB}
    \definecolor{ansi-blue-intense}{HTML}{0065CA}
    \definecolor{ansi-magenta}{HTML}{D160C4}
    \definecolor{ansi-magenta-intense}{HTML}{A03196}
    \definecolor{ansi-cyan}{HTML}{60C6C8}
    \definecolor{ansi-cyan-intense}{HTML}{258F8F}
    \definecolor{ansi-white}{HTML}{C5C1B4}
    \definecolor{ansi-white-intense}{HTML}{A1A6B2}
    \definecolor{ansi-default-inverse-fg}{HTML}{FFFFFF}
    \definecolor{ansi-default-inverse-bg}{HTML}{000000}

    % common color for the border for error outputs.
    \definecolor{outerrorbackground}{HTML}{FFDFDF}

    % commands and environments needed by pandoc snippets
    % extracted from the output of `pandoc -s`
    \providecommand{\tightlist}{%
      \setlength{\itemsep}{0pt}\setlength{\parskip}{0pt}}
    \DefineVerbatimEnvironment{Highlighting}{Verbatim}{commandchars=\\\{\}}
    % Add ',fontsize=\small' for more characters per line
    \newenvironment{Shaded}{}{}
    \newcommand{\KeywordTok}[1]{\textcolor[rgb]{0.00,0.44,0.13}{\textbf{{#1}}}}
    \newcommand{\DataTypeTok}[1]{\textcolor[rgb]{0.56,0.13,0.00}{{#1}}}
    \newcommand{\DecValTok}[1]{\textcolor[rgb]{0.25,0.63,0.44}{{#1}}}
    \newcommand{\BaseNTok}[1]{\textcolor[rgb]{0.25,0.63,0.44}{{#1}}}
    \newcommand{\FloatTok}[1]{\textcolor[rgb]{0.25,0.63,0.44}{{#1}}}
    \newcommand{\CharTok}[1]{\textcolor[rgb]{0.25,0.44,0.63}{{#1}}}
    \newcommand{\StringTok}[1]{\textcolor[rgb]{0.25,0.44,0.63}{{#1}}}
    \newcommand{\CommentTok}[1]{\textcolor[rgb]{0.38,0.63,0.69}{\textit{{#1}}}}
    \newcommand{\OtherTok}[1]{\textcolor[rgb]{0.00,0.44,0.13}{{#1}}}
    \newcommand{\AlertTok}[1]{\textcolor[rgb]{1.00,0.00,0.00}{\textbf{{#1}}}}
    \newcommand{\FunctionTok}[1]{\textcolor[rgb]{0.02,0.16,0.49}{{#1}}}
    \newcommand{\RegionMarkerTok}[1]{{#1}}
    \newcommand{\ErrorTok}[1]{\textcolor[rgb]{1.00,0.00,0.00}{\textbf{{#1}}}}
    \newcommand{\NormalTok}[1]{{#1}}
    
    % Additional commands for more recent versions of Pandoc
    \newcommand{\ConstantTok}[1]{\textcolor[rgb]{0.53,0.00,0.00}{{#1}}}
    \newcommand{\SpecialCharTok}[1]{\textcolor[rgb]{0.25,0.44,0.63}{{#1}}}
    \newcommand{\VerbatimStringTok}[1]{\textcolor[rgb]{0.25,0.44,0.63}{{#1}}}
    \newcommand{\SpecialStringTok}[1]{\textcolor[rgb]{0.73,0.40,0.53}{{#1}}}
    \newcommand{\ImportTok}[1]{{#1}}
    \newcommand{\DocumentationTok}[1]{\textcolor[rgb]{0.73,0.13,0.13}{\textit{{#1}}}}
    \newcommand{\AnnotationTok}[1]{\textcolor[rgb]{0.38,0.63,0.69}{\textbf{\textit{{#1}}}}}
    \newcommand{\CommentVarTok}[1]{\textcolor[rgb]{0.38,0.63,0.69}{\textbf{\textit{{#1}}}}}
    \newcommand{\VariableTok}[1]{\textcolor[rgb]{0.10,0.09,0.49}{{#1}}}
    \newcommand{\ControlFlowTok}[1]{\textcolor[rgb]{0.00,0.44,0.13}{\textbf{{#1}}}}
    \newcommand{\OperatorTok}[1]{\textcolor[rgb]{0.40,0.40,0.40}{{#1}}}
    \newcommand{\BuiltInTok}[1]{{#1}}
    \newcommand{\ExtensionTok}[1]{{#1}}
    \newcommand{\PreprocessorTok}[1]{\textcolor[rgb]{0.74,0.48,0.00}{{#1}}}
    \newcommand{\AttributeTok}[1]{\textcolor[rgb]{0.49,0.56,0.16}{{#1}}}
    \newcommand{\InformationTok}[1]{\textcolor[rgb]{0.38,0.63,0.69}{\textbf{\textit{{#1}}}}}
    \newcommand{\WarningTok}[1]{\textcolor[rgb]{0.38,0.63,0.69}{\textbf{\textit{{#1}}}}}
    
    
    % Define a nice break command that doesn't care if a line doesn't already
    % exist.
    \def\br{\hspace*{\fill} \\* }
    % Math Jax compatibility definitions
    \def\gt{>}
    \def\lt{<}
    \let\Oldtex\TeX
    \let\Oldlatex\LaTeX
    \renewcommand{\TeX}{\textrm{\Oldtex}}
    \renewcommand{\LaTeX}{\textrm{\Oldlatex}}
    % Document parameters
    % Document title
    \title{Probability Statistics Training Quiz}
    
    
    
    
    
% Pygments definitions
\makeatletter
\def\PY@reset{\let\PY@it=\relax \let\PY@bf=\relax%
    \let\PY@ul=\relax \let\PY@tc=\relax%
    \let\PY@bc=\relax \let\PY@ff=\relax}
\def\PY@tok#1{\csname PY@tok@#1\endcsname}
\def\PY@toks#1+{\ifx\relax#1\empty\else%
    \PY@tok{#1}\expandafter\PY@toks\fi}
\def\PY@do#1{\PY@bc{\PY@tc{\PY@ul{%
    \PY@it{\PY@bf{\PY@ff{#1}}}}}}}
\def\PY#1#2{\PY@reset\PY@toks#1+\relax+\PY@do{#2}}

\@namedef{PY@tok@w}{\def\PY@tc##1{\textcolor[rgb]{0.73,0.73,0.73}{##1}}}
\@namedef{PY@tok@c}{\let\PY@it=\textit\def\PY@tc##1{\textcolor[rgb]{0.24,0.48,0.48}{##1}}}
\@namedef{PY@tok@cp}{\def\PY@tc##1{\textcolor[rgb]{0.61,0.40,0.00}{##1}}}
\@namedef{PY@tok@k}{\let\PY@bf=\textbf\def\PY@tc##1{\textcolor[rgb]{0.00,0.50,0.00}{##1}}}
\@namedef{PY@tok@kp}{\def\PY@tc##1{\textcolor[rgb]{0.00,0.50,0.00}{##1}}}
\@namedef{PY@tok@kt}{\def\PY@tc##1{\textcolor[rgb]{0.69,0.00,0.25}{##1}}}
\@namedef{PY@tok@o}{\def\PY@tc##1{\textcolor[rgb]{0.40,0.40,0.40}{##1}}}
\@namedef{PY@tok@ow}{\let\PY@bf=\textbf\def\PY@tc##1{\textcolor[rgb]{0.67,0.13,1.00}{##1}}}
\@namedef{PY@tok@nb}{\def\PY@tc##1{\textcolor[rgb]{0.00,0.50,0.00}{##1}}}
\@namedef{PY@tok@nf}{\def\PY@tc##1{\textcolor[rgb]{0.00,0.00,1.00}{##1}}}
\@namedef{PY@tok@nc}{\let\PY@bf=\textbf\def\PY@tc##1{\textcolor[rgb]{0.00,0.00,1.00}{##1}}}
\@namedef{PY@tok@nn}{\let\PY@bf=\textbf\def\PY@tc##1{\textcolor[rgb]{0.00,0.00,1.00}{##1}}}
\@namedef{PY@tok@ne}{\let\PY@bf=\textbf\def\PY@tc##1{\textcolor[rgb]{0.80,0.25,0.22}{##1}}}
\@namedef{PY@tok@nv}{\def\PY@tc##1{\textcolor[rgb]{0.10,0.09,0.49}{##1}}}
\@namedef{PY@tok@no}{\def\PY@tc##1{\textcolor[rgb]{0.53,0.00,0.00}{##1}}}
\@namedef{PY@tok@nl}{\def\PY@tc##1{\textcolor[rgb]{0.46,0.46,0.00}{##1}}}
\@namedef{PY@tok@ni}{\let\PY@bf=\textbf\def\PY@tc##1{\textcolor[rgb]{0.44,0.44,0.44}{##1}}}
\@namedef{PY@tok@na}{\def\PY@tc##1{\textcolor[rgb]{0.41,0.47,0.13}{##1}}}
\@namedef{PY@tok@nt}{\let\PY@bf=\textbf\def\PY@tc##1{\textcolor[rgb]{0.00,0.50,0.00}{##1}}}
\@namedef{PY@tok@nd}{\def\PY@tc##1{\textcolor[rgb]{0.67,0.13,1.00}{##1}}}
\@namedef{PY@tok@s}{\def\PY@tc##1{\textcolor[rgb]{0.73,0.13,0.13}{##1}}}
\@namedef{PY@tok@sd}{\let\PY@it=\textit\def\PY@tc##1{\textcolor[rgb]{0.73,0.13,0.13}{##1}}}
\@namedef{PY@tok@si}{\let\PY@bf=\textbf\def\PY@tc##1{\textcolor[rgb]{0.64,0.35,0.47}{##1}}}
\@namedef{PY@tok@se}{\let\PY@bf=\textbf\def\PY@tc##1{\textcolor[rgb]{0.67,0.36,0.12}{##1}}}
\@namedef{PY@tok@sr}{\def\PY@tc##1{\textcolor[rgb]{0.64,0.35,0.47}{##1}}}
\@namedef{PY@tok@ss}{\def\PY@tc##1{\textcolor[rgb]{0.10,0.09,0.49}{##1}}}
\@namedef{PY@tok@sx}{\def\PY@tc##1{\textcolor[rgb]{0.00,0.50,0.00}{##1}}}
\@namedef{PY@tok@m}{\def\PY@tc##1{\textcolor[rgb]{0.40,0.40,0.40}{##1}}}
\@namedef{PY@tok@gh}{\let\PY@bf=\textbf\def\PY@tc##1{\textcolor[rgb]{0.00,0.00,0.50}{##1}}}
\@namedef{PY@tok@gu}{\let\PY@bf=\textbf\def\PY@tc##1{\textcolor[rgb]{0.50,0.00,0.50}{##1}}}
\@namedef{PY@tok@gd}{\def\PY@tc##1{\textcolor[rgb]{0.63,0.00,0.00}{##1}}}
\@namedef{PY@tok@gi}{\def\PY@tc##1{\textcolor[rgb]{0.00,0.52,0.00}{##1}}}
\@namedef{PY@tok@gr}{\def\PY@tc##1{\textcolor[rgb]{0.89,0.00,0.00}{##1}}}
\@namedef{PY@tok@ge}{\let\PY@it=\textit}
\@namedef{PY@tok@gs}{\let\PY@bf=\textbf}
\@namedef{PY@tok@gp}{\let\PY@bf=\textbf\def\PY@tc##1{\textcolor[rgb]{0.00,0.00,0.50}{##1}}}
\@namedef{PY@tok@go}{\def\PY@tc##1{\textcolor[rgb]{0.44,0.44,0.44}{##1}}}
\@namedef{PY@tok@gt}{\def\PY@tc##1{\textcolor[rgb]{0.00,0.27,0.87}{##1}}}
\@namedef{PY@tok@err}{\def\PY@bc##1{{\setlength{\fboxsep}{\string -\fboxrule}\fcolorbox[rgb]{1.00,0.00,0.00}{1,1,1}{\strut ##1}}}}
\@namedef{PY@tok@kc}{\let\PY@bf=\textbf\def\PY@tc##1{\textcolor[rgb]{0.00,0.50,0.00}{##1}}}
\@namedef{PY@tok@kd}{\let\PY@bf=\textbf\def\PY@tc##1{\textcolor[rgb]{0.00,0.50,0.00}{##1}}}
\@namedef{PY@tok@kn}{\let\PY@bf=\textbf\def\PY@tc##1{\textcolor[rgb]{0.00,0.50,0.00}{##1}}}
\@namedef{PY@tok@kr}{\let\PY@bf=\textbf\def\PY@tc##1{\textcolor[rgb]{0.00,0.50,0.00}{##1}}}
\@namedef{PY@tok@bp}{\def\PY@tc##1{\textcolor[rgb]{0.00,0.50,0.00}{##1}}}
\@namedef{PY@tok@fm}{\def\PY@tc##1{\textcolor[rgb]{0.00,0.00,1.00}{##1}}}
\@namedef{PY@tok@vc}{\def\PY@tc##1{\textcolor[rgb]{0.10,0.09,0.49}{##1}}}
\@namedef{PY@tok@vg}{\def\PY@tc##1{\textcolor[rgb]{0.10,0.09,0.49}{##1}}}
\@namedef{PY@tok@vi}{\def\PY@tc##1{\textcolor[rgb]{0.10,0.09,0.49}{##1}}}
\@namedef{PY@tok@vm}{\def\PY@tc##1{\textcolor[rgb]{0.10,0.09,0.49}{##1}}}
\@namedef{PY@tok@sa}{\def\PY@tc##1{\textcolor[rgb]{0.73,0.13,0.13}{##1}}}
\@namedef{PY@tok@sb}{\def\PY@tc##1{\textcolor[rgb]{0.73,0.13,0.13}{##1}}}
\@namedef{PY@tok@sc}{\def\PY@tc##1{\textcolor[rgb]{0.73,0.13,0.13}{##1}}}
\@namedef{PY@tok@dl}{\def\PY@tc##1{\textcolor[rgb]{0.73,0.13,0.13}{##1}}}
\@namedef{PY@tok@s2}{\def\PY@tc##1{\textcolor[rgb]{0.73,0.13,0.13}{##1}}}
\@namedef{PY@tok@sh}{\def\PY@tc##1{\textcolor[rgb]{0.73,0.13,0.13}{##1}}}
\@namedef{PY@tok@s1}{\def\PY@tc##1{\textcolor[rgb]{0.73,0.13,0.13}{##1}}}
\@namedef{PY@tok@mb}{\def\PY@tc##1{\textcolor[rgb]{0.40,0.40,0.40}{##1}}}
\@namedef{PY@tok@mf}{\def\PY@tc##1{\textcolor[rgb]{0.40,0.40,0.40}{##1}}}
\@namedef{PY@tok@mh}{\def\PY@tc##1{\textcolor[rgb]{0.40,0.40,0.40}{##1}}}
\@namedef{PY@tok@mi}{\def\PY@tc##1{\textcolor[rgb]{0.40,0.40,0.40}{##1}}}
\@namedef{PY@tok@il}{\def\PY@tc##1{\textcolor[rgb]{0.40,0.40,0.40}{##1}}}
\@namedef{PY@tok@mo}{\def\PY@tc##1{\textcolor[rgb]{0.40,0.40,0.40}{##1}}}
\@namedef{PY@tok@ch}{\let\PY@it=\textit\def\PY@tc##1{\textcolor[rgb]{0.24,0.48,0.48}{##1}}}
\@namedef{PY@tok@cm}{\let\PY@it=\textit\def\PY@tc##1{\textcolor[rgb]{0.24,0.48,0.48}{##1}}}
\@namedef{PY@tok@cpf}{\let\PY@it=\textit\def\PY@tc##1{\textcolor[rgb]{0.24,0.48,0.48}{##1}}}
\@namedef{PY@tok@c1}{\let\PY@it=\textit\def\PY@tc##1{\textcolor[rgb]{0.24,0.48,0.48}{##1}}}
\@namedef{PY@tok@cs}{\let\PY@it=\textit\def\PY@tc##1{\textcolor[rgb]{0.24,0.48,0.48}{##1}}}

\def\PYZbs{\char`\\}
\def\PYZus{\char`\_}
\def\PYZob{\char`\{}
\def\PYZcb{\char`\}}
\def\PYZca{\char`\^}
\def\PYZam{\char`\&}
\def\PYZlt{\char`\<}
\def\PYZgt{\char`\>}
\def\PYZsh{\char`\#}
\def\PYZpc{\char`\%}
\def\PYZdl{\char`$}
\def\PYZhy{\char`\-}
\def\PYZsq{\char`\'}
\def\PYZdq{\char`\"}
\def\PYZti{\char`\~}
% for compatibility with earlier versions
\def\PYZat{@}
\def\PYZlb{[}
\def\PYZrb{]}
\makeatother


    % For linebreaks inside Verbatim environment from package fancyvrb. 
    \makeatletter
        \newbox\Wrappedcontinuationbox 
        \newbox\Wrappedvisiblespacebox 
        \newcommand*\Wrappedvisiblespace {\textcolor{red}{\textvisiblespace}} 
        \newcommand*\Wrappedcontinuationsymbol {\textcolor{red}{\llap{\tiny$\m@th\hookrightarrow$}}} 
        \newcommand*\Wrappedcontinuationindent {3ex } 
        \newcommand*\Wrappedafterbreak {\kern\Wrappedcontinuationindent\copy\Wrappedcontinuationbox} 
        % Take advantage of the already applied Pygments mark-up to insert 
        % potential linebreaks for TeX processing. 
        %        {, <, #, %, $, ' and ": go to next line. 
        %        _, }, ^, &, >, - and ~: stay at end of broken line. 
        % Use of \textquotesingle for straight quote. 
        \newcommand*\Wrappedbreaksatspecials {% 
            \def\PYGZus{\discretionary{\char`\_}{\Wrappedafterbreak}{\char`\_}}% 
            \def\PYGZob{\discretionary{}{\Wrappedafterbreak\char`\{}{\char`\{}}% 
            \def\PYGZcb{\discretionary{\char`\}}{\Wrappedafterbreak}{\char`\}}}% 
            \def\PYGZca{\discretionary{\char`\^}{\Wrappedafterbreak}{\char`\^}}% 
            \def\PYGZam{\discretionary{\char`\&}{\Wrappedafterbreak}{\char`\&}}% 
            \def\PYGZlt{\discretionary{}{\Wrappedafterbreak\char`\<}{\char`\<}}% 
            \def\PYGZgt{\discretionary{\char`\>}{\Wrappedafterbreak}{\char`\>}}% 
            \def\PYGZsh{\discretionary{}{\Wrappedafterbreak\char`\#}{\char`\#}}% 
            \def\PYGZpc{\discretionary{}{\Wrappedafterbreak\char`\%}{\char`\%}}% 
            \def\PYGZdl{\discretionary{}{\Wrappedafterbreak\char`$}{\char`$}}% 
            \def\PYGZhy{\discretionary{\char`\-}{\Wrappedafterbreak}{\char`\-}}% 
            \def\PYGZsq{\discretionary{}{\Wrappedafterbreak\textquotesingle}{\textquotesingle}}% 
            \def\PYGZdq{\discretionary{}{\Wrappedafterbreak\char`\"}{\char`\"}}% 
            \def\PYGZti{\discretionary{\char`\~}{\Wrappedafterbreak}{\char`\~}}% 
        } 
        % Some characters . , ; ? ! / are not pygmentized. 
        % This macro makes them "active" and they will insert potential linebreaks 
        \newcommand*\Wrappedbreaksatpunct {% 
            \lccode`\~`\.\lowercase{\def~}{\discretionary{\hbox{\char`\.}}{\Wrappedafterbreak}{\hbox{\char`\.}}}% 
            \lccode`\~`\,\lowercase{\def~}{\discretionary{\hbox{\char`\,}}{\Wrappedafterbreak}{\hbox{\char`\,}}}% 
            \lccode`\~`\;\lowercase{\def~}{\discretionary{\hbox{\char`\;}}{\Wrappedafterbreak}{\hbox{\char`\;}}}% 
            \lccode`\~`\:\lowercase{\def~}{\discretionary{\hbox{\char`\:}}{\Wrappedafterbreak}{\hbox{\char`\:}}}% 
            \lccode`\~`\?\lowercase{\def~}{\discretionary{\hbox{\char`\?}}{\Wrappedafterbreak}{\hbox{\char`\?}}}% 
            \lccode`\~`\!\lowercase{\def~}{\discretionary{\hbox{\char`\!}}{\Wrappedafterbreak}{\hbox{\char`\!}}}% 
            \lccode`\~`\/\lowercase{\def~}{\discretionary{\hbox{\char`\/}}{\Wrappedafterbreak}{\hbox{\char`\/}}}% 
            \catcode`\.\active
            \catcode`\,\active 
            \catcode`\;\active
            \catcode`\:\active
            \catcode`\?\active
            \catcode`\!\active
            \catcode`\/\active 
            \lccode`\~`\~ 	
        }
    \makeatother

    \let\OriginalVerbatim=\Verbatim
    \makeatletter
    \renewcommand{\Verbatim}[1][1]{%
        %\parskip\z@skip
        \sbox\Wrappedcontinuationbox {\Wrappedcontinuationsymbol}%
        \sbox\Wrappedvisiblespacebox {\FV@SetupFont\Wrappedvisiblespace}%
        \def\FancyVerbFormatLine ##1{\hsize\linewidth
            \vtop{\raggedright\hyphenpenalty\z@\exhyphenpenalty\z@
                \doublehyphendemerits\z@\finalhyphendemerits\z@
                \strut ##1\strut}%
        }%
        % If the linebreak is at a space, the latter will be displayed as visible
        % space at end of first line, and a continuation symbol starts next line.
        % Stretch/shrink are however usually zero for typewriter font.
        \def\FV@Space {%
            \nobreak\hskip\z@ plus\fontdimen3\font minus\fontdimen4\font
            \discretionary{\copy\Wrappedvisiblespacebox}{\Wrappedafterbreak}
            {\kern\fontdimen2\font}%
        }%
        
        % Allow breaks at special characters using \PYG... macros.
        \Wrappedbreaksatspecials
        % Breaks at punctuation characters . , ; ? ! and / need catcode=\active 	
        \OriginalVerbatim[#1,codes*=\Wrappedbreaksatpunct]%
    }
    \makeatother

    % Exact colors from NB
    \definecolor{incolor}{HTML}{303F9F}
    \definecolor{outcolor}{HTML}{D84315}
    \definecolor{cellborder}{HTML}{CFCFCF}
    \definecolor{cellbackground}{HTML}{F7F7F7}
    
    % prompt
    \makeatletter
    \newcommand{\boxspacing}{\kern\kvtcb@left@rule\kern\kvtcb@boxsep}
    \makeatother
    \newcommand{\prompt}[4]{
        {\ttfamily\llap{{\color{#2}[#3]:\hspace{3pt}#4}}\vspace{-\baselineskip}}
    }
    

    
    % Prevent overflowing lines due to hard-to-break entities
    \sloppy 
    % Setup hyperref package
    \hypersetup{
      breaklinks=true,  % so long urls are correctly broken across lines
      colorlinks=true,
      urlcolor=urlcolor,
      linkcolor=linkcolor,
      citecolor=citecolor,
      }
    % Slightly bigger margins than the latex defaults
    
    \geometry{verbose,tmargin=1in,bmargin=1in,lmargin=1in,rmargin=1in}
    
    

\begin{document}
    
    \maketitle
    
    

    
    $\textbf{Question1}$

For the given sets A=\{1,2,3,4\} and B=\{2,4,6\} find : A $\cup$ B, A $\cap$ B, A - B, B - A

$\textbf{Answer}$

    Let's break down each operation using the given sets A and B:

\begin{enumerate}
\def\labelenumi{\arabic{enumi}.}
\item
  Union (A $\cup$ B):\\
  The union of two sets contains all the elements that are present in
  either set A or set B, or in both sets.\\
  A $\cup$ B = \{1, 2, 3, 4\} $\cup$ \{2, 4, 6\} = \{1, 2, 3, 4, 6\}
\item
  Intersection (A $\cap$ B):\\
  The intersection of two sets contains all the elements that are common
  to both set A and set B.\\
  A $\cap$ B = \{1, 2, 3, 4\} $\cap$ \{2, 4, 6\} = \{2, 4\}
\item
  Set Difference (A - B):\\
  The set difference A - B contains all the elements that are in set A
  but not in set B.\\
  A - B = \{1, 2, 3, 4\} - \{2, 4, 6\} = \{1, 3\}
\item
  Set Difference (B - A):\\
  The set difference B - A contains all the elements that are in set B
  but not in set A.\\
  B - A = \{2, 4, 6\} - \{1, 2, 3, 4\} = \{6\}
\end{enumerate}

    \begin{tcolorbox}[breakable, size=fbox, boxrule=1pt, pad at break*=1mm,colback=cellbackground, colframe=cellborder]
\prompt{In}{incolor}{ }{\boxspacing}
\begin{Verbatim}[commandchars=\\\{\}]

\end{Verbatim}
\end{tcolorbox}

    $\textbf{Question2}$

Let A and B two sets which are subsets of a set Ω. Which of the
following expressions is not correct?

\begin{enumerate}
\def\labelenumi{\alph{enumi}.}
\item
  (A$\cup$B)$\cap$(A$\cup$C)=A$\cup$(B$\cap$C)
\item
  (A$\cup$B)=(A$\cap$Bc)$\cup$B
\item
  (A$\cup$B)c$\cap$C=(Ac$\cap$Bc)$\cap$C
\item
  Ac$\cap$B=A$\cup$B
\item
  (A$\cap$B)$\cap$(Bc$\cap$C)=$\emptyset$
\end{enumerate}

$\textbf{Answer}$

    Let's analyze each option:

\begin{enumerate}
\def\labelenumi{\alph{enumi}.}
\tightlist
\item
  $(A \cup B) \cap (A \cup C)=A \cup (B \cap C)$
\end{enumerate}

This is an application of the distributive property of set operations
and is indeed correct.

\begin{enumerate}
\def\labelenumi{\alph{enumi}.}
\setcounter{enumi}{1}
\tightlist
\item
  $(A \cup B)=(A \cap B^c) \cup B$
\end{enumerate}

This is the De Morgan's law and is correct.

\begin{enumerate}
\def\labelenumi{\alph{enumi}.}
\setcounter{enumi}{2}
\tightlist
\item
  $(A \cup B)^c \cap C=(A^c \cap B^c) \cap C$
\end{enumerate}

This also seems to be an application of De Morgan's law and is correct.

\begin{enumerate}
\def\labelenumi{\alph{enumi}.}
\setcounter{enumi}{3}
\tightlist
\item
  $A^c \cap B=A \cup B$
\end{enumerate}

This expression is incorrect. The correct expression should be
$A^c \cap B=(A^c \cup B^c)^c$. So, this is the incorrect option.

\begin{enumerate}
\def\labelenumi{\alph{enumi}.}
\setcounter{enumi}{4}
\tightlist
\item
  $(A \cap B) \cap (B^c \cap C)= \emptyset $
\end{enumerate}

This is correct since the intersection of $B$ and its complement is an
empty set.

So, the incorrect option is d.~$A^c \cap B=A \cup B$.

    \begin{tcolorbox}[breakable, size=fbox, boxrule=1pt, pad at break*=1mm,colback=cellbackground, colframe=cellborder]
\prompt{In}{incolor}{ }{\boxspacing}
\begin{Verbatim}[commandchars=\\\{\}]

\end{Verbatim}
\end{tcolorbox}

    $\textbf{Question3}$

In how many ways can 5 people be seated in a row of 5 chairs:

$\textbf{Answer}$

    When seating people in a row of chairs, the order matters. So, it's a
permutation problem.

For the first seat, you have 5 choices. For the second seat, you have 4
choices remaining, for the third seat, you have 3 choices remaining, and
so on.

Therefore, the total number of ways to seat 5 people in a row of 5
chairs is:

$5 \times 4 \times 3 \times 2 \times 1 = 5!$

$5! = 5 \times 4 \times 3 \times 2 \times 1 = 120$

So, there are 120 ways to seat 5 people in a row of 5 chairs.

    \begin{tcolorbox}[breakable, size=fbox, boxrule=1pt, pad at break*=1mm,colback=cellbackground, colframe=cellborder]
\prompt{In}{incolor}{ }{\boxspacing}
\begin{Verbatim}[commandchars=\\\{\}]

\end{Verbatim}
\end{tcolorbox}

    $\textbf{Question4}$

Eleven people meet each other and handshake. How many handshakes will
occur?

$\textbf{Answer}$

    In a group of $ n $ people, each person can shake hands with every
other person except themselves.

So, for 11 people, each person shakes hands with 10 other people.

However, this would count each handshake twice (once for each person
involved), so we must divide by 2 to avoid double-counting.

The total number of handshakes $ H $ can be calculated as:

$ H = \frac{n \times (n - 1)}{2} $

Substituting $ n = 11 $:

$ H = \frac{11 \times (11 - 1)}{2} $ $ H = \frac{11 \times 10}{2} $
$ H = \frac{110}{2} $ $ H = 55 $

So, there will be 55 handshakes occurring.

    \begin{tcolorbox}[breakable, size=fbox, boxrule=1pt, pad at break*=1mm,colback=cellbackground, colframe=cellborder]
\prompt{In}{incolor}{ }{\boxspacing}
\begin{Verbatim}[commandchars=\\\{\}]

\end{Verbatim}
\end{tcolorbox}

    $\textbf{Question5}$

What is the probability of randomly choosing, without replacement, three
red balls out of a bag of 10 red and 5 green balls?

$\textbf{Answer}$

    To find the probability of randomly choosing three red balls out of a
bag of 10 red and 5 green balls, we need to consider the total number of
ways to choose 3 balls out of the 15 balls in the bag and the number of
ways to choose 3 red balls out of the 10 red balls.

The total number of ways to choose 3 balls out of 15 is given by the
combination formula:

$ \text{Total number of ways} = \binom{15}{3} = \frac{15!}{3!(15-3)!}
$

Similarly, the number of ways to choose 3 red balls out of 10 is:

$ \text{Number of ways to choose 3 red balls} = \binom{10}{3} =
\frac{10!}{3!(10-3)!} $

So, the probability of choosing three red balls is:

$ P(\text{3 red balls}) =
\frac{\text{Number of ways to choose 3 red balls}}{\text{Total number of ways}}
$

$ P(\text{3 red balls}) = \frac{\binom{10}{3}}{\binom{15}{3}} $

$ P(\text{3 red balls}) =
\frac{\frac{10!}{3!(10-3)!}}{\frac{15!}{3!(15-3)!}} $

$ P(\text{3 red balls}) = \frac{\frac{10!}{3!7!}}{\frac{15!}{3!12!}} $

$ P(\text{3 red balls}) =
\frac{\frac{10 \times 9 \times 8}{3 \times 2 \times 1}}{\frac{15 \times 14 \times 13}{3 \times 2 \times 1}}
$

$ P(\text{3 red balls}) =
\frac{10 \times 9 \times 8}{15 \times 14 \times 13} $

$ P(\text{3 red balls}) = \frac{720}{2730} $

$ P(\text{3 red balls}) \approx 0.263 $

So, the probability of randomly choosing, without replacement, three red
balls out of the bag is approximately 0.263.

    \begin{tcolorbox}[breakable, size=fbox, boxrule=1pt, pad at break*=1mm,colback=cellbackground, colframe=cellborder]
\prompt{In}{incolor}{ }{\boxspacing}
\begin{Verbatim}[commandchars=\\\{\}]

\end{Verbatim}
\end{tcolorbox}

    $\textbf{Question6}$

You are given an unfair die where an odd outcome is four times as likely
as an even outcome. All odd outcomes are equally likely to each other,
and all even outcomes are equally likely to each other. What's the
probability of rolling a 5 given that you have rolled an odd number?

$\textbf{Answer}$

    Let's denote the probability of rolling an odd number as $
P(\text{odd}) $ and the probability of rolling an even number as $
P(\text{even}) $.

Given that an odd outcome is four times as likely as an even outcome, we
can set up the following relationships:

\begin{enumerate}
\def\labelenumi{\arabic{enumi}.}
\tightlist
\item
  $ P(\text{odd}) = 4 \times P(\text{even}) $
\item
  $ P(\text{odd}) + P(\text{even}) = 1 $ (since the sum of
  probabilities of all possible outcomes is 1)
\end{enumerate}

From equation 1, we can express $ P(\text{even}) $ in terms of $
P(\text{odd}) $: $ P(\text{even}) = \frac{1}{4} \times P(\text{odd})
$

Now, let's find the values of $ P(\text{odd}) $ and $ P(\text{even})
$.

Using equation 2: $ P(\text{odd}) + \frac{1}{4} \times P(\text{odd}) =
1 $ $ \frac{5}{4} \times P(\text{odd}) = 1 $ $ P(\text{odd}) =
\frac{4}{5} $

And using equation 1: $ P(\text{even}) = \frac{1}{4} \times \frac{4}{5}
= \frac{1}{5} $

Now, to find the probability of rolling a 5 given that you have rolled
an odd number ($ P(\text{5}\textbar{}\text{odd}) $), we use the
conditional probability formula:

$ P(\text{5}\textbar{}\text{odd}) =
\frac{P(\text{5 and odd})}{P(\text{odd})} $

Since all odd outcomes are equally likely to each other, the probability
of rolling a 5 given that you have rolled an odd number is simply the
probability of rolling a 5, which is $ \frac{1}{5} $.

So, the probability of rolling a 5 given that you have rolled an odd
number is $ \frac{1}{5} $.

    \begin{tcolorbox}[breakable, size=fbox, boxrule=1pt, pad at break*=1mm,colback=cellbackground, colframe=cellborder]
\prompt{In}{incolor}{ }{\boxspacing}
\begin{Verbatim}[commandchars=\\\{\}]

\end{Verbatim}
\end{tcolorbox}

    $\textbf{Question7}$

A fair die is thrown twice. Let the events A= \{sum of the throws equals
4\}, B=\{at least one of the throws is a 3\}. What is P(A\textbar B) ?

$\textbf{Answer}$

    To find the conditional probability $ P(A\textbar B) $, which
represents the probability of event A given event B, we'll use the
formula for conditional probability:

$ P(A\textbar B) = \frac{P(A \cap B)}{P(B)} $

First, let's find the probability of event B, which is the probability
of getting at least one 3 in two throws of a fair die.

$ P(B) = 1 - P(\text{no 3 in two throws}) $

To find the probability of no 3 in two throws, we can think of each
throw as independent events. The probability of not getting a 3 on a
single throw is $ \frac{5}{6} $ (since there are 5 outcomes that are
not 3 out of 6 possible outcomes). So, the probability of not getting a
3 in two throws is $ \left(\frac{5}{6}\right)^2 $.

$ P(\text{no 3 in two throws}) = \left(\frac{5}{6}\right)^2 =
\frac{25}{36} $

Therefore,

$ P(B) = 1 - \frac{25}{36} = \frac{11}{36} $

Next, let's find the probability of event A intersected with B, which is
the probability of both A and B occurring. Event A is the sum of throws
equaling 4, and event B is at least one of the throws being a 3. The
only way both events can happen simultaneously is if one throw is a 3
and the other is a 1, or vice versa (since 3 + 1 = 4 and 1 + 3 = 4).

$ P(A \cap B) = P(\text{one throw is 3 and the other is 1}) +
P(\text{one throw is 1 and the other is 3}) $

Each of these scenarios has a probability of $ \frac{1}{6}
\times \frac{1}{6} = \frac{1}{36} $.

$ P(A \cap B) = 2 \times \frac{1}{36} = \frac{2}{36} = \frac{1}{18} $

Now we can plug these values into the formula for conditional
probability:

$ P(A\textbar B) = \frac{P(A \cap B)}{P(B)} =
\frac{\frac{1}{18}}{\frac{11}{36}} = \frac{1}{18} \times \frac{36}{11} =
\frac{2}{11} $

So, $ P(A\textbar B) = \frac{2}{11} $.

    \begin{tcolorbox}[breakable, size=fbox, boxrule=1pt, pad at break*=1mm,colback=cellbackground, colframe=cellborder]
\prompt{In}{incolor}{ }{\boxspacing}
\begin{Verbatim}[commandchars=\\\{\}]

\end{Verbatim}
\end{tcolorbox}

    $\textbf{Question8}$

There is a 30 \% chance of rain today, and a 40 \% chance your umbrella
order will arrive on time. However, you found out that if it rains,
there is only a 20 \% chance your umbrella will arrive on time. What is
the probability it will rain and your umbrella will arrive on time?

$\textbf{Answer}$

    To find the probability that it will rain and your umbrella will arrive
on time, we can use the conditional probability formula:

$ P(\text{rain and umbrella on time}) = P(\text{rain})
\times P(\text{umbrella on time}\textbar{}\text{rain}) $

Given:

\begin{itemize}
\tightlist
\item
  Probability of rain: $ P(\text{rain}) = 0.30 $
\item
  Probability of umbrella arriving on time: $
  P(\text{umbrella on time}) = 0.40 $
\item
  Probability of umbrella arriving on time given that it rains: $
  P(\text{umbrella on time}\textbar{}\text{rain}) = 0.20 $
\end{itemize}

Now, we can plug these values into the formula:

$ P(\text{rain and umbrella on time}) = 0.30 \times 0.20 = 0.06 $

So, the probability that it will rain and your umbrella will arrive on
time is $ 0.06 $, or 6\%.

    \begin{tcolorbox}[breakable, size=fbox, boxrule=1pt, pad at break*=1mm,colback=cellbackground, colframe=cellborder]
\prompt{In}{incolor}{ }{\boxspacing}
\begin{Verbatim}[commandchars=\\\{\}]

\end{Verbatim}
\end{tcolorbox}

    $\textbf{Question9}$

For the events A and B, we know that the probabilities P(A)=0.3,
P(B)=0.2 and P(B\textbar A)=0.5. The probability P(A\textbar B) is:

$\textbf{Answer}$

    To find $ P(A\textbar B) $, the probability of event A given event B,
we can use Bayes' theorem:

$ P(A\textbar B) = \frac{P(B|A) \times P(A)}{P(B)} $

Given:

\begin{itemize}
\tightlist
\item
  Probability of event A: $ P(A) = 0.3 $
\item
  Probability of event B: $ P(B) = 0.2 $
\item
  Probability of B given A: $ P(B\textbar A) = 0.5 $
\end{itemize}

Now, we can plug these values into Bayes' theorem:

$ P(A\textbar B) = \frac{0.5 \times 0.3}{0.2} $

$ P(A\textbar B) = \frac{0.15}{0.2} $

$ P(A\textbar B) = 0.75 $

So, the probability $ P(A\textbar B) $ is $ 0.75 $ or 75\%.

    \begin{tcolorbox}[breakable, size=fbox, boxrule=1pt, pad at break*=1mm,colback=cellbackground, colframe=cellborder]
\prompt{In}{incolor}{ }{\boxspacing}
\begin{Verbatim}[commandchars=\\\{\}]

\end{Verbatim}
\end{tcolorbox}

    $\textbf{Question12}$

In an experiment into extrasensory perception, a subject guesses the
symbol on a card. There are five equally likely and different symbols,
so the subject has a 15 chance of guessing correct by chance. If the
subject makes 6 sequential card guesses, what is the probability that
more than half are correct?

$\textbf{Answer}$

    To solve this problem, we can use the binomial probability formula, as
the subject's guesses can be considered independent Bernoulli trials.

The probability of getting exactly $ k $ successes (correct guesses)
in $ n $ trials is given by the binomial probability formula:

$ P(X = k) = \binom{n}{k} \times p^k \times (1 - p)^\{n - k\} $

Where: - $ n $ is the number of trials (in this case, the number of
guesses) - $ k $ is the number of successes (in this case, the number
of correct guesses) - $ p $ is the probability of success on each
trial (the chance of guessing correctly by chance)

Given that the subject has a $ 1/5 $ chance of guessing correctly by
chance (so $ p = 1/5 $), we can calculate the probability of guessing
more than half of the guesses correctly. Since there are 6 guesses and
we want more than half, we need to calculate the probability of getting
4 or more correct guesses.

$ P(X \textgreater{} 3) = P(X = 4) + P(X = 5) + P(X = 6) $

$ P(X \textgreater{} 3) = \binom{6}{4}
\times \left(\frac{1}{5}\right)^4
\times \left(\frac{4}{5}\right)^2 + \binom{6}{5}
\times \left(\frac{1}{5}\right)^5
\times \left(\frac{4}{5}\right)^1 + \binom{6}{6}
\times \left(\frac{1}{5}\right)^6
\times \left(\frac{4}{5}\right)^0 $

Let's calculate each term:

$ \binom{6}{4} = \frac{6!}{4!(6-4)!} = 15 $

$ \binom{6}{5} = \frac{6!}{5!(6-5)!} = 6 $

$ \binom{6}{6} = \frac{6!}{6!(6-6)!} = 1 $

Now, plug these values into the formula:

$ P(X \textgreater{} 3) = (15 \times \frac{1}{5^4}
\times \frac{4^2}{5^2}) + (6 \times \frac{1}{5^5} \times \frac{4^1}{5})
+ (1 \times \frac{1}{5^6}) $

$ P(X \textgreater{} 3) = (\frac{15 \times 16}{5^6}) +
(\frac{6 \times 4}{5^6}) + \frac{1}{5^6} $

$ P(X \textgreater{} 3) = \frac{240 + 24 + 1}{5^6} $

$ P(X \textgreater{} 3) = \frac{265}{15625} $

$ P(X \textgreater{} 3) \approx 0.01696 $

So, the probability that more than half of the guesses are correct is
approximately $ 0.01696 $, or $ 1.696\% $.

    \begin{tcolorbox}[breakable, size=fbox, boxrule=1pt, pad at break*=1mm,colback=cellbackground, colframe=cellborder]
\prompt{In}{incolor}{ }{\boxspacing}
\begin{Verbatim}[commandchars=\\\{\}]

\end{Verbatim}
\end{tcolorbox}

    $\textbf{Question13}$

The marks of ten students in Course A and Course B are as follows:

Course A 7 3 4 10 8 2.5 4 6.5 7 9

Course B 8 3 5 5.5 6 4 7 8.5 9 7

Which of the following statements is true?

\begin{enumerate}
\def\labelenumi{\alph{enumi}.}
\tightlist
\item
  The mean performance of the students in Course A is lower than that in
  Course B. CorrectWell done!\\
\item
  The mean performance of students in Course A is higher than that in
  Course B.\\
\item
  The standard deviation that is recorded in the grades of Course A is
  lower than the corresponding value of Course B.\\
\item
  The standard deviation that is recorded in the grades of Course B is
  equal to 3.5.\\
\item
  None of these.
\end{enumerate}

$\textbf{Answer}$

    To find the correct statement, let's calculate the mean and standard
deviation for both Course A and Course B.

For Course A: Mean (average) = $
\frac{7 + 3 + 4 + 10 + 8 + 2.5 + 4 + 6.5 + 7 + 9}{10} = \frac{61}{10} =
6.1 $ Standard Deviation = $ \sqrt{\frac{\sum{(x_i - \mu)^2}}{N}} $
Where: - $ x\_i $ are the individual marks, - $ \mu $ is the mean, -
$ N $ is the number of data points.

$ \sum{(x_i - \mu)^2} = (7 - 6.1)^2 + (3 - 6.1)^2 + \dots + (9 -
6.1)^2 $ $ = 0.81 + 9.61 + 5.29 + 13.69 + 3.61 + 13.69 + 5.29 +
0.09 + 0.81 + 7.29 $ $ = 60.29 $ $ \text{Standard Deviation} =
\sqrt{\frac{60.29}{10}} = \sqrt{6.029} \approx 2.454 $

For Course B: Mean = $
\frac{8 + 3 + 5 + 5.5 + 6 + 4 + 7 + 8.5 + 9 + 7}{10} = \frac{63}{10} =
6.3 $ $ \sum{(x_i - \mu)^2} = (8 - 6.3)^2 + (3 - 6.3)^2 +
\dots + (7 - 6.3)^2 $ $ = 3.61 + 10.89 + 2.89 + 0.09 + 0.09 + 4.41
+ 0.49 + 6.25 + 8.41 + 0.49 $ $ = 37.63 $ $
\text{Standard Deviation} = \sqrt{\frac{37.63}{10}} = \sqrt{3.763}
\approx 1.941 $

Comparing the mean performances: - Course A: $ \mu\_A = 6.1 $ - Course
B: $ \mu\_B = 6.3 $

So, statement (a) is correct: The mean performance of the students in
Course A is lower than that in Course B.

Comparing the standard deviations: - Course A: $ \sigma\_A
\approx 2.454 $ - Course B: $ \sigma\_B \approx 1.941 $

So, statement (c) is incorrect: The standard deviation in the grades of
Course A is not lower than that of Course B.

Therefore, the correct answer is (a).

    \begin{tcolorbox}[breakable, size=fbox, boxrule=1pt, pad at break*=1mm,colback=cellbackground, colframe=cellborder]
\prompt{In}{incolor}{ }{\boxspacing}
\begin{Verbatim}[commandchars=\\\{\}]

\end{Verbatim}
\end{tcolorbox}

    $\textbf{Question14}$

A mathematics test is given to a class of 24 students. Two students
score 0 and all other students score 12. What is the standard deviation
of the exam score?

$\textbf{Answer}$

    To find the standard deviation of the exam scores, we first need to
calculate the mean of the scores. Then we'll use the formula for
standard deviation:

$ \text{Standard Deviation} = \sqrt{\frac{\sum{(x_i - \mu)^2}}{N}} $

Where: - $ x\_i $ are the individual scores, - $ \mu $ is the mean,
- $ N $ is the number of students.

Given that 2 students score 0 and all other students score 12:

Mean = $ \frac{2(0) + 22(12)}{24} = \frac{264}{24} = 11 $

Now, let's calculate the sum of the squares of the differences between
each score and the mean:

$ \sum{(x_i - \mu)^2} = (0 - 11)^2 + (0 - 11)^2 + (12 - 11)^2
+ \dots + (12 - 11)^2 $

$ = 2(11^2) + 22(1^2) $

$ = 242 + 22 $

$ = 264 $

Now, plug these values into the formula for standard deviation:

$ \text{Standard Deviation} = \sqrt{\frac{264}{24}} $

$ \text{Standard Deviation} = \sqrt{11} $

Therefore, the standard deviation of the exam scores is $ \sqrt{11} $
or approximately $ 3.317 $.

    \begin{tcolorbox}[breakable, size=fbox, boxrule=1pt, pad at break*=1mm,colback=cellbackground, colframe=cellborder]
\prompt{In}{incolor}{ }{\boxspacing}
\begin{Verbatim}[commandchars=\\\{\}]

\end{Verbatim}
\end{tcolorbox}

    $\textbf{Question15}$

Consider the independent random variables X and Y have variances 0.2 and
0.5, respectively. Let Z=5X-2Y. The variance of the random variable Z
equals:

$\textbf{Answer}$

    Given that $ X $ and $ Y $ are independent random variables, the
variance of the sum or difference of independent random variables is the
sum of their variances. However, when constants are involved in the
linear combinations, variances get scaled by the square of the
constants.

Given: - Variance of $ X $: $ \text{Var}(X) = 0.2 $ - Variance of $
Y $: $ \text{Var}(Y) = 0.5 $ - $ Z = 5X - 2Y $

We know that for linear combinations of random variables: $
\text{Var}(aX + bY) = a^2 \text{Var}(X) + b^2 \text{Var}(Y) $

So, for $ Z = 5X - 2Y $: $ \text{Var}(Z) = (5)^2 \text{Var}(X) +
(-2)^2 \text{Var}(Y) $ $ = 25 \times 0.2 + 4 \times 0.5 $ $ = 5 +
2 $ $ = 7 $

Therefore, the variance of the random variable $ Z $ is $ 7 $.

    \begin{tcolorbox}[breakable, size=fbox, boxrule=1pt, pad at break*=1mm,colback=cellbackground, colframe=cellborder]
\prompt{In}{incolor}{ }{\boxspacing}
\begin{Verbatim}[commandchars=\\\{\}]

\end{Verbatim}
\end{tcolorbox}

    $\textbf{Question16}$

Let X be a continuous random variable with probability density function
(pdf): $f(x)=\{9/(4x^3) , 1 \leq x \leq 3 \&\& 0$, otherwise.

The probability P(X\textgreater2) equals:

$\textbf{Answer}$

    To find the probability $ P(X \textgreater{} 2) $, we need to
integrate the probability density function (pdf) from $ x = 2 $ to $
x = 3 $.

Given the pdf: $ f(x) =\begin{cases} \frac{9}{4x^3} & 1 \leq x \leq 3 \\ 0 & \text{otherwise} \end{cases}$

We can write the integral for $ P(X \textgreater{} 2) $ as follows:

$ P(X \textgreater{} 2) = \int\_\{2\}^\{3\} f(x) , dx $

$ P(X \textgreater{} 2) = \int\_\{2\}^\{3\} \frac{9}{4x^3} , dx $

Now, let's integrate the function:

$ P(X \textgreater{} 2) =
\left[ -\frac{9}{4(2x^2)} \right]\_\{2\}^\{3\} $

$ P(X \textgreater{} 2) =
\left[ -\frac{9}{8x^2} \right]\_\{2\}^\{3\} $

$ P(X \textgreater{} 2) = -\frac{9}{8(3^2)} + \frac{9}{8(2^2)} $

$ P(X \textgreater{} 2) = -\frac{9}{72} + \frac{9}{32} $

$ P(X \textgreater{} 2) = -\frac{1}{8} + \frac{9}{32} $

$ P(X \textgreater{} 2) = -\frac{4}{32} + \frac{9}{32} $

$ P(X \textgreater{} 2) = \frac{5}{32} $

So, the probability $ P(X \textgreater{} 2) $ equals $ \frac{5}{32}
$.

    \begin{tcolorbox}[breakable, size=fbox, boxrule=1pt, pad at break*=1mm,colback=cellbackground, colframe=cellborder]
\prompt{In}{incolor}{ }{\boxspacing}
\begin{Verbatim}[commandchars=\\\{\}]

\end{Verbatim}
\end{tcolorbox}

    $\textbf{Question17}$

Let X be a continuous random variable with probability density function
(pdf): $f(x)=c,0 \leq x \leq 1 \&\& -cx+2c,1\textless x \leq 2 \&\& 0, otherwise$.

The value of constant c equals:

$\textbf{Answer}$

    To find the value of the constant $ c $, we need to ensure that the
probability density function (pdf) integrates to 1 over the entire range
of $ x $, which is from 0 to 2 in this case.

The integral of the pdf over its entire range must equal 1:

$ \int\_\{0\}^\{2\} f(x) , dx = 1 $

Given the piecewise function: $ f(x) =\begin{cases} c & 0 \leq x \leq 1 \\ -cx + 2c & 1 < x \leq 2 \\ 0 & \text{otherwise} \end{cases}$

We need to integrate $ f(x) $ over the intervals $[0,1]$ and
$[1,2]$, and set the sum of these integrals equal to 1.

$ \int_{0}^{1} c \, dx + \int_{1}^{2} (-cx + 2c) \, dx = 1 $

Now, let's compute each integral:

$ \int_{0}^{1} c \, dx = cx \Bigg|_{0}^{1} = c \cdot 1 - c \cdot 0 = c $

$ \int_{1}^{2} (-cx + 2c) \, dx = \left( -\frac{cx^2}{2} + 2cx \right) \Bigg|_{1}^{2} $
$ = \left( -\frac{c \cdot 2^2}{2} + 2c \cdot 2 \right) - \left( -\frac{c \cdot 1^2}{2} + 2c \cdot 1 \right) $
$ = \left( -2c + 4c \right) - \left( -\frac{c}{2} + 2c \right) $
$ = (2c) - (-\frac{c}{2} + 2c) $
$ = 2c + \frac{c}{2} - 2c $
$ = \frac{c}{2} $

Now, we set the sum of these integrals equal to 1:

$ c + \frac{c}{2} = 1 $

$ \frac{3c}{2} = 1 $

$ c = \frac{2}{3} $

So, the value of the constant $ c $ is $ \frac{2}{3} $.

    \begin{tcolorbox}[breakable, size=fbox, boxrule=1pt, pad at break*=1mm,colback=cellbackground, colframe=cellborder]
\prompt{In}{incolor}{ }{\boxspacing}
\begin{Verbatim}[commandchars=\\\{\}]

\end{Verbatim}
\end{tcolorbox}

    $\textbf{Question18}$

The random variable X follows a Normal distribution $N(μ,σ^2)$. If for some real number $ δ \textgreater 0 $, $P(X \geq μ+δ)=0.15865 $ then
$P(μ-δ\textless X\textless μ) approximately equals $

$\textbf{Answer}$

    Given that $ X $ follows a normal distribution $ N(\mu, \sigma^2)
$, we know that the standard normal distribution (with mean 0 and
standard deviation 1), denoted as $ Z $, can be obtained by
standardizing $ X $ as follows:

$ Z = \frac{X - \mu}{\sigma} $

Given: $ P(X \geq \mu + \delta) = 0.15865 $

To find $ P(\mu - \delta \textless{} X \textless{} \mu) $, we first
standardize the given probabilities using the standard normal
distribution.

For $ P(X \geq \mu + \delta) $, standardizing gives: $ P\left(Z
\geq \frac{\mu + \delta - \mu}{\sigma}\right) = P(Z
\geq \frac{\delta}{\sigma}) $

Using the properties of the standard normal distribution (which is
symmetric about 0), $ P(Z \geq \frac{\delta}{\sigma}) $ is equivalent
to $ 1 - P(Z \textless{} \frac{\delta}{\sigma}) $.

Given that $ P(X \geq \mu + \delta) = 0.15865 $, we have: $ 1 - P(Z
\textless{} \frac{\delta}{\sigma}) = 0.15865 $

This implies that: $ P(Z \textless{} \frac{\delta}{\sigma}) = 1 -
0.15865 = 0.84135 $

Now, we want to find $ P(\mu - \delta \textless{} X \textless{} \mu)
$. We can express this as: $ P(X \textless{} \mu) - P(X \textless{}
\mu - \delta) $

Standardizing both terms gives: $ P(Z \textless{} 0) - P(Z \textless{}
\frac{\mu - \mu + \delta}{\sigma}) $

$ = P(Z \textless{} 0) - P(Z \textless{} \frac{\delta}{\sigma}) $

$ = 0.5 - P(Z \textless{} \frac{\delta}{\sigma}) $

Given that $ P(Z \textless{} \frac{\delta}{\sigma}) = 0.84135 $, we
have: $ P(X \textless{} \mu) - P(X \textless{} \mu - \delta)
\approx 0.5 - 0.84135 $

$ \approx -0.34135 $

But probabilities cannot be negative, so we need to take the absolute
value:

$ \textbar0.5 - 0.84135\textbar{} \approx 0.34135 $

So, $ P(\mu - \delta \textless{} X \textless{} \mu) $ is approximately
$ 0.34135 $.

    \begin{tcolorbox}[breakable, size=fbox, boxrule=1pt, pad at break*=1mm,colback=cellbackground, colframe=cellborder]
\prompt{In}{incolor}{ }{\boxspacing}
\begin{Verbatim}[commandchars=\\\{\}]

\end{Verbatim}
\end{tcolorbox}

    $\textbf{Question19}$

Let X1,\ldots,X100 be a random sample of 100 independent and identically
distributed observations drawn from an unknown distribution with
expected value μ=120 and standard deviation σ=45.

The probability that the sample mean X¯¯¯¯=X1+\ldots+X100/100 lies in
the interval {[}112,125{]} is approximately equal to:

$\textbf{Answer}$

    To solve this problem, we can use the Central Limit Theorem (CLT), which
states that the distribution of the sample mean of a large sample drawn
from any population will be approximately normally distributed,
regardless of the shape of the population distribution.

Given that we have a sample size of 100, we can use the normal
distribution to approximate the distribution of the sample mean.

According to the CLT, the mean of the sample means ($ \mu_{\bar{X}} $) is equal to the population mean ($ \mu $), and the standard deviation of the sample means ($ \sigma_{\bar{X}} $) is equal to the population standard deviation divided by the square root of the sample size ($ \sigma / \sqrt{n} $).

Given: - $ \mu = 120 $ - $ \sigma = 45 $ - Sample size ($ n $) =
100

$ \mu\_\{\bar\{X\}\} = \mu = 120 $

$ \sigma\_\{\bar\{X\}\} = \frac{\sigma}{\sqrt{n}} =
\frac{45}{\sqrt{100}} = \frac{45}{10} = 4.5 $

Now, we standardize the interval {[}112, 125{]} using the sample mean
distribution:

$ Z\_1 = \frac{112 - \mu_{\bar{X}}}{\sigma_{\bar{X}}} =
\frac{112 - 120}{4.5} = \frac{-8}{4.5} = -1.7778 $

$ Z\_2 = \frac{125 - \mu_{\bar{X}}}{\sigma_{\bar{X}}} =
\frac{125 - 120}{4.5} = \frac{5}{4.5} = 1.1111 $

Now, we find the probabilities corresponding to these z-scores using a
standard normal distribution table or calculator.

$ P(Z\_1 \textless{} Z \textless{} Z\_2) $

$ P(-1.7778 \textless{} Z \textless{} 1.1111) $

Using a standard normal distribution table or calculator, we find:

$ P(Z \textless{} 1.1111) = 0.8665 $ $ P(Z \textless{} -1.7778) =
0.0384 $

$ P(-1.7778 \textless{} Z \textless{} 1.1111) = P(Z \textless{} 1.1111)
- P(Z \textless{} -1.7778) = 0.8665 - 0.0384 = 0.8281 $

So, the probability that the sample mean $ \bar\{X\} $ lies in the
interval {[}112, 125{]} is approximately 0.8281.

    \begin{tcolorbox}[breakable, size=fbox, boxrule=1pt, pad at break*=1mm,colback=cellbackground, colframe=cellborder]
\prompt{In}{incolor}{ }{\boxspacing}
\begin{Verbatim}[commandchars=\\\{\}]

\end{Verbatim}
\end{tcolorbox}

    $\textbf{Question20}$

Let X and Y be two independent random variables such that E(X)=109/50,
E(Y)=157/100, Var(X)×Var(Y)\textgreater0, and E(XY)=17113/5000. We
conclude that

$\textbf{Answer}$

    To find out what we can conclude from the given information, let's break
down the equations and analyze them step by step.

Given: - $ E(X) = \frac{109}{50} $ - $ E(Y) = \frac{157}{100} $ - $
\text{Var}(X) \times \text{Var}(Y) \textgreater{} 0 $ - $ E(XY) =
\frac{17113}{5000} $

We know that the covariance of two independent random variables is 0.
Therefore, $ E(XY) = E(X)E(Y) $ when X and Y are independent.

From the given information, we have:

$ E(X)E(Y) = \left(\frac{109}{50}\right) \left(\frac{157}{100}\right) =
\frac{17113}{5000} $

which matches $ E(XY) $.

So, given that the expected value of the product of the random variables
is equal to the product of their expected values, we can conclude that X
and Y are uncorrelated, which is a property of independent random
variables. Therefore, the conclusion is that X and Y are independent.

    \begin{tcolorbox}[breakable, size=fbox, boxrule=1pt, pad at break*=1mm,colback=cellbackground, colframe=cellborder]
\prompt{In}{incolor}{ }{\boxspacing}
\begin{Verbatim}[commandchars=\\\{\}]

\end{Verbatim}
\end{tcolorbox}

    $\textbf{Question21}$

Let X and Y be two random variables with E(X)=0.5, E(Y)=0.6, E(XY)=0.39
and standard deviations σX\textgreater0 and σY\textgreater0p. Which of
the following is correct: a. Cov(X,Y)=0. b. Variables X and Y are
independent. c.~Cov(X,Y)=0.09. d.~When X increases Y tends to decrease.
e. None of the above.

$\textbf{Answer}$

    To determine the correct statement, let's analyze each option:

\begin{enumerate}
\def\labelenumi{\alph{enumi}.}
\tightlist
\item
  $ \text{Cov}(X,Y) = 0 $:
\end{enumerate}

The covariance of two random variables is defined as: $ \text{Cov}(X,Y)
= E(XY) - E(X)E(Y) $

Given $ E(X) = 0.5 $, $ E(Y) = 0.6 $, and $ E(XY) = 0.39 $, we
have: $ \text{Cov}(X,Y) = 0.39 - (0.5)(0.6) = 0.39 - 0.3 = 0.09 $

So, statement a is incorrect.

\begin{enumerate}
\def\labelenumi{\alph{enumi}.}
\setcounter{enumi}{1}
\tightlist
\item
  Variables X and Y are independent:
\end{enumerate}

For two random variables to be independent, their covariance must be 0.
Since we found $ \text{Cov}(X,Y) = 0.09 $, this implies that X and Y
are not independent. Therefore, statement b is incorrect.

\begin{enumerate}
\def\labelenumi{\alph{enumi}.}
\setcounter{enumi}{2}
\tightlist
\item
  $ \text{Cov}(X,Y) = 0.09 $:
\end{enumerate}

As we calculated earlier, $ \text{Cov}(X,Y) = 0.09 $. So, statement c
is correct.

\begin{enumerate}
\def\labelenumi{\alph{enumi}.}
\setcounter{enumi}{3}
\tightlist
\item
  When X increases Y tends to decrease:
\end{enumerate}

The covariance does not directly indicate the direction of the
relationship between two variables. It only measures the strength and
direction of the linear relationship between them. Therefore, we cannot
directly conclude the relationship between X and Y based on the
covariance alone. Thus, statement d is incorrect.

Therefore, the correct answer is option c: $ \text{Cov}(X,Y) = 0.09 $.

    \begin{tcolorbox}[breakable, size=fbox, boxrule=1pt, pad at break*=1mm,colback=cellbackground, colframe=cellborder]
\prompt{In}{incolor}{ }{\boxspacing}
\begin{Verbatim}[commandchars=\\\{\}]

\end{Verbatim}
\end{tcolorbox}


    % Add a bibliography block to the postdoc
    
    
    
\end{document}

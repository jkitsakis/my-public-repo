\documentclass[11pt]{article}

    \usepackage[breakable]{tcolorbox}
    \usepackage{parskip} % Stop auto-indenting (to mimic markdown behaviour)
    

    % Basic figure setup, for now with no caption control since it's done
    % automatically by Pandoc (which extracts ![](path) syntax from Markdown).
    \usepackage{graphicx}
    % Maintain compatibility with old templates. Remove in nbconvert 6.0
    \let\Oldincludegraphics\includegraphics
    % Ensure that by default, figures have no caption (until we provide a
    % proper Figure object with a Caption API and a way to capture that
    % in the conversion process - todo).
    \usepackage{caption}
    \DeclareCaptionFormat{nocaption}{}
    \captionsetup{format=nocaption,aboveskip=0pt,belowskip=0pt}

    \usepackage{float}
    \floatplacement{figure}{H} % forces figures to be placed at the correct location
    \usepackage{xcolor} % Allow colors to be defined
    \usepackage{enumerate} % Needed for markdown enumerations to work
    \usepackage{geometry} % Used to adjust the document margins
    \usepackage{amsmath} % Equations
    \usepackage{amssymb} % Equations
    \usepackage{textcomp} % defines textquotesingle
    % Hack from http://tex.stackexchange.com/a/47451/13684:
    \AtBeginDocument{%
        \def\PYZsq{\textquotesingle}% Upright quotes in Pygmentized code
    }
    \usepackage{upquote} % Upright quotes for verbatim code
    \usepackage{eurosym} % defines \euro

    \usepackage{iftex}
    \ifPDFTeX
        \usepackage[T1]{fontenc}
        \IfFileExists{alphabeta.sty}{
              \usepackage{alphabeta}
          }{
              \usepackage[mathletters]{ucs}
              \usepackage[utf8x]{inputenc}
          }
    \else
        \usepackage{fontspec}
        \usepackage{unicode-math}
    \fi

    \usepackage{fancyvrb} % verbatim replacement that allows latex
    \usepackage[Export]{adjustbox} % Used to constrain images to a maximum size
    \adjustboxset{max size={0.9\linewidth}{0.9\paperheight}}

    % The hyperref package gives us a pdf with properly built
    % internal navigation ('pdf bookmarks' for the table of contents,
    % internal cross-reference links, web links for URLs, etc.)
    \usepackage{hyperref}
    % The default LaTeX title has an obnoxious amount of whitespace. By default,
    % titling removes some of it. It also provides customization options.
    \usepackage{titling}
    \usepackage{longtable} % longtable support required by pandoc >1.10
    \usepackage{booktabs}  % table support for pandoc > 1.12.2
    \usepackage{array}     % table support for pandoc >= 2.11.3
    \usepackage{calc}      % table minipage width calculation for pandoc >= 2.11.1
    \usepackage[inline]{enumitem} % IRkernel/repr support (it uses the enumerate* environment)
    \usepackage[normalem]{ulem} % ulem is needed to support strikethroughs (\sout)
                                % normalem makes italics be italics, not underlines
    \usepackage{mathrsfs}
    

    
    % Colors for the hyperref package
    \definecolor{urlcolor}{rgb}{0,.145,.698}
    \definecolor{linkcolor}{rgb}{.71,0.21,0.01}
    \definecolor{citecolor}{rgb}{.12,.54,.11}

    % ANSI colors
    \definecolor{ansi-black}{HTML}{3E424D}
    \definecolor{ansi-black-intense}{HTML}{282C36}
    \definecolor{ansi-red}{HTML}{E75C58}
    \definecolor{ansi-red-intense}{HTML}{B22B31}
    \definecolor{ansi-green}{HTML}{00A250}
    \definecolor{ansi-green-intense}{HTML}{007427}
    \definecolor{ansi-yellow}{HTML}{DDB62B}
    \definecolor{ansi-yellow-intense}{HTML}{B27D12}
    \definecolor{ansi-blue}{HTML}{208FFB}
    \definecolor{ansi-blue-intense}{HTML}{0065CA}
    \definecolor{ansi-magenta}{HTML}{D160C4}
    \definecolor{ansi-magenta-intense}{HTML}{A03196}
    \definecolor{ansi-cyan}{HTML}{60C6C8}
    \definecolor{ansi-cyan-intense}{HTML}{258F8F}
    \definecolor{ansi-white}{HTML}{C5C1B4}
    \definecolor{ansi-white-intense}{HTML}{A1A6B2}
    \definecolor{ansi-default-inverse-fg}{HTML}{FFFFFF}
    \definecolor{ansi-default-inverse-bg}{HTML}{000000}

    % common color for the border for error outputs.
    \definecolor{outerrorbackground}{HTML}{FFDFDF}

    % commands and environments needed by pandoc snippets
    % extracted from the output of `pandoc -s`
    \providecommand{\tightlist}{%
      \setlength{\itemsep}{0pt}\setlength{\parskip}{0pt}}
    \DefineVerbatimEnvironment{Highlighting}{Verbatim}{commandchars=\\\{\}}
    % Add ',fontsize=\small' for more characters per line
    \newenvironment{Shaded}{}{}
    \newcommand{\KeywordTok}[1]{\textcolor[rgb]{0.00,0.44,0.13}{\textbf{{#1}}}}
    \newcommand{\DataTypeTok}[1]{\textcolor[rgb]{0.56,0.13,0.00}{{#1}}}
    \newcommand{\DecValTok}[1]{\textcolor[rgb]{0.25,0.63,0.44}{{#1}}}
    \newcommand{\BaseNTok}[1]{\textcolor[rgb]{0.25,0.63,0.44}{{#1}}}
    \newcommand{\FloatTok}[1]{\textcolor[rgb]{0.25,0.63,0.44}{{#1}}}
    \newcommand{\CharTok}[1]{\textcolor[rgb]{0.25,0.44,0.63}{{#1}}}
    \newcommand{\StringTok}[1]{\textcolor[rgb]{0.25,0.44,0.63}{{#1}}}
    \newcommand{\CommentTok}[1]{\textcolor[rgb]{0.38,0.63,0.69}{\textit{{#1}}}}
    \newcommand{\OtherTok}[1]{\textcolor[rgb]{0.00,0.44,0.13}{{#1}}}
    \newcommand{\AlertTok}[1]{\textcolor[rgb]{1.00,0.00,0.00}{\textbf{{#1}}}}
    \newcommand{\FunctionTok}[1]{\textcolor[rgb]{0.02,0.16,0.49}{{#1}}}
    \newcommand{\RegionMarkerTok}[1]{{#1}}
    \newcommand{\ErrorTok}[1]{\textcolor[rgb]{1.00,0.00,0.00}{\textbf{{#1}}}}
    \newcommand{\NormalTok}[1]{{#1}}
    
    % Additional commands for more recent versions of Pandoc
    \newcommand{\ConstantTok}[1]{\textcolor[rgb]{0.53,0.00,0.00}{{#1}}}
    \newcommand{\SpecialCharTok}[1]{\textcolor[rgb]{0.25,0.44,0.63}{{#1}}}
    \newcommand{\VerbatimStringTok}[1]{\textcolor[rgb]{0.25,0.44,0.63}{{#1}}}
    \newcommand{\SpecialStringTok}[1]{\textcolor[rgb]{0.73,0.40,0.53}{{#1}}}
    \newcommand{\ImportTok}[1]{{#1}}
    \newcommand{\DocumentationTok}[1]{\textcolor[rgb]{0.73,0.13,0.13}{\textit{{#1}}}}
    \newcommand{\AnnotationTok}[1]{\textcolor[rgb]{0.38,0.63,0.69}{\textbf{\textit{{#1}}}}}
    \newcommand{\CommentVarTok}[1]{\textcolor[rgb]{0.38,0.63,0.69}{\textbf{\textit{{#1}}}}}
    \newcommand{\VariableTok}[1]{\textcolor[rgb]{0.10,0.09,0.49}{{#1}}}
    \newcommand{\ControlFlowTok}[1]{\textcolor[rgb]{0.00,0.44,0.13}{\textbf{{#1}}}}
    \newcommand{\OperatorTok}[1]{\textcolor[rgb]{0.40,0.40,0.40}{{#1}}}
    \newcommand{\BuiltInTok}[1]{{#1}}
    \newcommand{\ExtensionTok}[1]{{#1}}
    \newcommand{\PreprocessorTok}[1]{\textcolor[rgb]{0.74,0.48,0.00}{{#1}}}
    \newcommand{\AttributeTok}[1]{\textcolor[rgb]{0.49,0.56,0.16}{{#1}}}
    \newcommand{\InformationTok}[1]{\textcolor[rgb]{0.38,0.63,0.69}{\textbf{\textit{{#1}}}}}
    \newcommand{\WarningTok}[1]{\textcolor[rgb]{0.38,0.63,0.69}{\textbf{\textit{{#1}}}}}
    
    
    % Define a nice break command that doesn't care if a line doesn't already
    % exist.
    \def\br{\hspace*{\fill} \\* }
    % Math Jax compatibility definitions
    \def\gt{>}
    \def\lt{<}
    \let\Oldtex\TeX
    \let\Oldlatex\LaTeX
    \renewcommand{\TeX}{\textrm{\Oldtex}}
    \renewcommand{\LaTeX}{\textrm{\Oldlatex}}
    % Document parameters
    % Document title
    \title{Continuous\_Optimization\_Training\_Quiz}
    
    
    
    
    
% Pygments definitions
\makeatletter
\def\PY@reset{\let\PY@it=\relax \let\PY@bf=\relax%
    \let\PY@ul=\relax \let\PY@tc=\relax%
    \let\PY@bc=\relax \let\PY@ff=\relax}
\def\PY@tok#1{\csname PY@tok@#1\endcsname}
\def\PY@toks#1+{\ifx\relax#1\empty\else%
    \PY@tok{#1}\expandafter\PY@toks\fi}
\def\PY@do#1{\PY@bc{\PY@tc{\PY@ul{%
    \PY@it{\PY@bf{\PY@ff{#1}}}}}}}
\def\PY#1#2{\PY@reset\PY@toks#1+\relax+\PY@do{#2}}

\@namedef{PY@tok@w}{\def\PY@tc##1{\textcolor[rgb]{0.73,0.73,0.73}{##1}}}
\@namedef{PY@tok@c}{\let\PY@it=\textit\def\PY@tc##1{\textcolor[rgb]{0.24,0.48,0.48}{##1}}}
\@namedef{PY@tok@cp}{\def\PY@tc##1{\textcolor[rgb]{0.61,0.40,0.00}{##1}}}
\@namedef{PY@tok@k}{\let\PY@bf=\textbf\def\PY@tc##1{\textcolor[rgb]{0.00,0.50,0.00}{##1}}}
\@namedef{PY@tok@kp}{\def\PY@tc##1{\textcolor[rgb]{0.00,0.50,0.00}{##1}}}
\@namedef{PY@tok@kt}{\def\PY@tc##1{\textcolor[rgb]{0.69,0.00,0.25}{##1}}}
\@namedef{PY@tok@o}{\def\PY@tc##1{\textcolor[rgb]{0.40,0.40,0.40}{##1}}}
\@namedef{PY@tok@ow}{\let\PY@bf=\textbf\def\PY@tc##1{\textcolor[rgb]{0.67,0.13,1.00}{##1}}}
\@namedef{PY@tok@nb}{\def\PY@tc##1{\textcolor[rgb]{0.00,0.50,0.00}{##1}}}
\@namedef{PY@tok@nf}{\def\PY@tc##1{\textcolor[rgb]{0.00,0.00,1.00}{##1}}}
\@namedef{PY@tok@nc}{\let\PY@bf=\textbf\def\PY@tc##1{\textcolor[rgb]{0.00,0.00,1.00}{##1}}}
\@namedef{PY@tok@nn}{\let\PY@bf=\textbf\def\PY@tc##1{\textcolor[rgb]{0.00,0.00,1.00}{##1}}}
\@namedef{PY@tok@ne}{\let\PY@bf=\textbf\def\PY@tc##1{\textcolor[rgb]{0.80,0.25,0.22}{##1}}}
\@namedef{PY@tok@nv}{\def\PY@tc##1{\textcolor[rgb]{0.10,0.09,0.49}{##1}}}
\@namedef{PY@tok@no}{\def\PY@tc##1{\textcolor[rgb]{0.53,0.00,0.00}{##1}}}
\@namedef{PY@tok@nl}{\def\PY@tc##1{\textcolor[rgb]{0.46,0.46,0.00}{##1}}}
\@namedef{PY@tok@ni}{\let\PY@bf=\textbf\def\PY@tc##1{\textcolor[rgb]{0.44,0.44,0.44}{##1}}}
\@namedef{PY@tok@na}{\def\PY@tc##1{\textcolor[rgb]{0.41,0.47,0.13}{##1}}}
\@namedef{PY@tok@nt}{\let\PY@bf=\textbf\def\PY@tc##1{\textcolor[rgb]{0.00,0.50,0.00}{##1}}}
\@namedef{PY@tok@nd}{\def\PY@tc##1{\textcolor[rgb]{0.67,0.13,1.00}{##1}}}
\@namedef{PY@tok@s}{\def\PY@tc##1{\textcolor[rgb]{0.73,0.13,0.13}{##1}}}
\@namedef{PY@tok@sd}{\let\PY@it=\textit\def\PY@tc##1{\textcolor[rgb]{0.73,0.13,0.13}{##1}}}
\@namedef{PY@tok@si}{\let\PY@bf=\textbf\def\PY@tc##1{\textcolor[rgb]{0.64,0.35,0.47}{##1}}}
\@namedef{PY@tok@se}{\let\PY@bf=\textbf\def\PY@tc##1{\textcolor[rgb]{0.67,0.36,0.12}{##1}}}
\@namedef{PY@tok@sr}{\def\PY@tc##1{\textcolor[rgb]{0.64,0.35,0.47}{##1}}}
\@namedef{PY@tok@ss}{\def\PY@tc##1{\textcolor[rgb]{0.10,0.09,0.49}{##1}}}
\@namedef{PY@tok@sx}{\def\PY@tc##1{\textcolor[rgb]{0.00,0.50,0.00}{##1}}}
\@namedef{PY@tok@m}{\def\PY@tc##1{\textcolor[rgb]{0.40,0.40,0.40}{##1}}}
\@namedef{PY@tok@gh}{\let\PY@bf=\textbf\def\PY@tc##1{\textcolor[rgb]{0.00,0.00,0.50}{##1}}}
\@namedef{PY@tok@gu}{\let\PY@bf=\textbf\def\PY@tc##1{\textcolor[rgb]{0.50,0.00,0.50}{##1}}}
\@namedef{PY@tok@gd}{\def\PY@tc##1{\textcolor[rgb]{0.63,0.00,0.00}{##1}}}
\@namedef{PY@tok@gi}{\def\PY@tc##1{\textcolor[rgb]{0.00,0.52,0.00}{##1}}}
\@namedef{PY@tok@gr}{\def\PY@tc##1{\textcolor[rgb]{0.89,0.00,0.00}{##1}}}
\@namedef{PY@tok@ge}{\let\PY@it=\textit}
\@namedef{PY@tok@gs}{\let\PY@bf=\textbf}
\@namedef{PY@tok@gp}{\let\PY@bf=\textbf\def\PY@tc##1{\textcolor[rgb]{0.00,0.00,0.50}{##1}}}
\@namedef{PY@tok@go}{\def\PY@tc##1{\textcolor[rgb]{0.44,0.44,0.44}{##1}}}
\@namedef{PY@tok@gt}{\def\PY@tc##1{\textcolor[rgb]{0.00,0.27,0.87}{##1}}}
\@namedef{PY@tok@err}{\def\PY@bc##1{{\setlength{\fboxsep}{\string -\fboxrule}\fcolorbox[rgb]{1.00,0.00,0.00}{1,1,1}{\strut ##1}}}}
\@namedef{PY@tok@kc}{\let\PY@bf=\textbf\def\PY@tc##1{\textcolor[rgb]{0.00,0.50,0.00}{##1}}}
\@namedef{PY@tok@kd}{\let\PY@bf=\textbf\def\PY@tc##1{\textcolor[rgb]{0.00,0.50,0.00}{##1}}}
\@namedef{PY@tok@kn}{\let\PY@bf=\textbf\def\PY@tc##1{\textcolor[rgb]{0.00,0.50,0.00}{##1}}}
\@namedef{PY@tok@kr}{\let\PY@bf=\textbf\def\PY@tc##1{\textcolor[rgb]{0.00,0.50,0.00}{##1}}}
\@namedef{PY@tok@bp}{\def\PY@tc##1{\textcolor[rgb]{0.00,0.50,0.00}{##1}}}
\@namedef{PY@tok@fm}{\def\PY@tc##1{\textcolor[rgb]{0.00,0.00,1.00}{##1}}}
\@namedef{PY@tok@vc}{\def\PY@tc##1{\textcolor[rgb]{0.10,0.09,0.49}{##1}}}
\@namedef{PY@tok@vg}{\def\PY@tc##1{\textcolor[rgb]{0.10,0.09,0.49}{##1}}}
\@namedef{PY@tok@vi}{\def\PY@tc##1{\textcolor[rgb]{0.10,0.09,0.49}{##1}}}
\@namedef{PY@tok@vm}{\def\PY@tc##1{\textcolor[rgb]{0.10,0.09,0.49}{##1}}}
\@namedef{PY@tok@sa}{\def\PY@tc##1{\textcolor[rgb]{0.73,0.13,0.13}{##1}}}
\@namedef{PY@tok@sb}{\def\PY@tc##1{\textcolor[rgb]{0.73,0.13,0.13}{##1}}}
\@namedef{PY@tok@sc}{\def\PY@tc##1{\textcolor[rgb]{0.73,0.13,0.13}{##1}}}
\@namedef{PY@tok@dl}{\def\PY@tc##1{\textcolor[rgb]{0.73,0.13,0.13}{##1}}}
\@namedef{PY@tok@s2}{\def\PY@tc##1{\textcolor[rgb]{0.73,0.13,0.13}{##1}}}
\@namedef{PY@tok@sh}{\def\PY@tc##1{\textcolor[rgb]{0.73,0.13,0.13}{##1}}}
\@namedef{PY@tok@s1}{\def\PY@tc##1{\textcolor[rgb]{0.73,0.13,0.13}{##1}}}
\@namedef{PY@tok@mb}{\def\PY@tc##1{\textcolor[rgb]{0.40,0.40,0.40}{##1}}}
\@namedef{PY@tok@mf}{\def\PY@tc##1{\textcolor[rgb]{0.40,0.40,0.40}{##1}}}
\@namedef{PY@tok@mh}{\def\PY@tc##1{\textcolor[rgb]{0.40,0.40,0.40}{##1}}}
\@namedef{PY@tok@mi}{\def\PY@tc##1{\textcolor[rgb]{0.40,0.40,0.40}{##1}}}
\@namedef{PY@tok@il}{\def\PY@tc##1{\textcolor[rgb]{0.40,0.40,0.40}{##1}}}
\@namedef{PY@tok@mo}{\def\PY@tc##1{\textcolor[rgb]{0.40,0.40,0.40}{##1}}}
\@namedef{PY@tok@ch}{\let\PY@it=\textit\def\PY@tc##1{\textcolor[rgb]{0.24,0.48,0.48}{##1}}}
\@namedef{PY@tok@cm}{\let\PY@it=\textit\def\PY@tc##1{\textcolor[rgb]{0.24,0.48,0.48}{##1}}}
\@namedef{PY@tok@cpf}{\let\PY@it=\textit\def\PY@tc##1{\textcolor[rgb]{0.24,0.48,0.48}{##1}}}
\@namedef{PY@tok@c1}{\let\PY@it=\textit\def\PY@tc##1{\textcolor[rgb]{0.24,0.48,0.48}{##1}}}
\@namedef{PY@tok@cs}{\let\PY@it=\textit\def\PY@tc##1{\textcolor[rgb]{0.24,0.48,0.48}{##1}}}

\def\PYZbs{\char`\\}
\def\PYZus{\char`\_}
\def\PYZob{\char`\{}
\def\PYZcb{\char`\}}
\def\PYZca{\char`\^}
\def\PYZam{\char`\&}
\def\PYZlt{\char`\<}
\def\PYZgt{\char`\>}
\def\PYZsh{\char`\#}
\def\PYZpc{\char`\%}
\def\PYZdl{\char`$}
\def\PYZhy{\char`\-}
\def\PYZsq{\char`\'}
\def\PYZdq{\char`\"}
\def\PYZti{\char`\~}
% for compatibility with earlier versions
\def\PYZat{@}
\def\PYZlb{[}
\def\PYZrb{]}
\makeatother


    % For linebreaks inside Verbatim environment from package fancyvrb. 
    \makeatletter
        \newbox\Wrappedcontinuationbox 
        \newbox\Wrappedvisiblespacebox 
        \newcommand*\Wrappedvisiblespace {\textcolor{red}{\textvisiblespace}} 
        \newcommand*\Wrappedcontinuationsymbol {\textcolor{red}{\llap{\tiny$\m@th\hookrightarrow$}}} 
        \newcommand*\Wrappedcontinuationindent {3ex } 
        \newcommand*\Wrappedafterbreak {\kern\Wrappedcontinuationindent\copy\Wrappedcontinuationbox} 
        % Take advantage of the already applied Pygments mark-up to insert 
        % potential linebreaks for TeX processing. 
        %        {, <, #, %, $, ' and ": go to next line. 
        %        _, }, ^, &, >, - and ~: stay at end of broken line. 
        % Use of \textquotesingle for straight quote. 
        \newcommand*\Wrappedbreaksatspecials {% 
            \def\PYGZus{\discretionary{\char`\_}{\Wrappedafterbreak}{\char`\_}}% 
            \def\PYGZob{\discretionary{}{\Wrappedafterbreak\char`\{}{\char`\{}}% 
            \def\PYGZcb{\discretionary{\char`\}}{\Wrappedafterbreak}{\char`\}}}% 
            \def\PYGZca{\discretionary{\char`\^}{\Wrappedafterbreak}{\char`\^}}% 
            \def\PYGZam{\discretionary{\char`\&}{\Wrappedafterbreak}{\char`\&}}% 
            \def\PYGZlt{\discretionary{}{\Wrappedafterbreak\char`\<}{\char`\<}}% 
            \def\PYGZgt{\discretionary{\char`\>}{\Wrappedafterbreak}{\char`\>}}% 
            \def\PYGZsh{\discretionary{}{\Wrappedafterbreak\char`\#}{\char`\#}}% 
            \def\PYGZpc{\discretionary{}{\Wrappedafterbreak\char`\%}{\char`\%}}% 
            \def\PYGZdl{\discretionary{}{\Wrappedafterbreak\char`$}{\char`$}}% 
            \def\PYGZhy{\discretionary{\char`\-}{\Wrappedafterbreak}{\char`\-}}% 
            \def\PYGZsq{\discretionary{}{\Wrappedafterbreak\textquotesingle}{\textquotesingle}}% 
            \def\PYGZdq{\discretionary{}{\Wrappedafterbreak\char`\"}{\char`\"}}% 
            \def\PYGZti{\discretionary{\char`\~}{\Wrappedafterbreak}{\char`\~}}% 
        } 
        % Some characters . , ; ? ! / are not pygmentized. 
        % This macro makes them "active" and they will insert potential linebreaks 
        \newcommand*\Wrappedbreaksatpunct {% 
            \lccode`\~`\.\lowercase{\def~}{\discretionary{\hbox{\char`\.}}{\Wrappedafterbreak}{\hbox{\char`\.}}}% 
            \lccode`\~`\,\lowercase{\def~}{\discretionary{\hbox{\char`\,}}{\Wrappedafterbreak}{\hbox{\char`\,}}}% 
            \lccode`\~`\;\lowercase{\def~}{\discretionary{\hbox{\char`\;}}{\Wrappedafterbreak}{\hbox{\char`\;}}}% 
            \lccode`\~`\:\lowercase{\def~}{\discretionary{\hbox{\char`\:}}{\Wrappedafterbreak}{\hbox{\char`\:}}}% 
            \lccode`\~`\?\lowercase{\def~}{\discretionary{\hbox{\char`\?}}{\Wrappedafterbreak}{\hbox{\char`\?}}}% 
            \lccode`\~`\!\lowercase{\def~}{\discretionary{\hbox{\char`\!}}{\Wrappedafterbreak}{\hbox{\char`\!}}}% 
            \lccode`\~`\/\lowercase{\def~}{\discretionary{\hbox{\char`\/}}{\Wrappedafterbreak}{\hbox{\char`\/}}}% 
            \catcode`\.\active
            \catcode`\,\active 
            \catcode`\;\active
            \catcode`\:\active
            \catcode`\?\active
            \catcode`\!\active
            \catcode`\/\active 
            \lccode`\~`\~ 	
        }
    \makeatother

    \let\OriginalVerbatim=\Verbatim
    \makeatletter
    \renewcommand{\Verbatim}[1][1]{%
        %\parskip\z@skip
        \sbox\Wrappedcontinuationbox {\Wrappedcontinuationsymbol}%
        \sbox\Wrappedvisiblespacebox {\FV@SetupFont\Wrappedvisiblespace}%
        \def\FancyVerbFormatLine ##1{\hsize\linewidth
            \vtop{\raggedright\hyphenpenalty\z@\exhyphenpenalty\z@
                \doublehyphendemerits\z@\finalhyphendemerits\z@
                \strut ##1\strut}%
        }%
        % If the linebreak is at a space, the latter will be displayed as visible
        % space at end of first line, and a continuation symbol starts next line.
        % Stretch/shrink are however usually zero for typewriter font.
        \def\FV@Space {%
            \nobreak\hskip\z@ plus\fontdimen3\font minus\fontdimen4\font
            \discretionary{\copy\Wrappedvisiblespacebox}{\Wrappedafterbreak}
            {\kern\fontdimen2\font}%
        }%
        
        % Allow breaks at special characters using \PYG... macros.
        \Wrappedbreaksatspecials
        % Breaks at punctuation characters . , ; ? ! and / need catcode=\active 	
        \OriginalVerbatim[#1,codes*=\Wrappedbreaksatpunct]%
    }
    \makeatother

    % Exact colors from NB
    \definecolor{incolor}{HTML}{303F9F}
    \definecolor{outcolor}{HTML}{D84315}
    \definecolor{cellborder}{HTML}{CFCFCF}
    \definecolor{cellbackground}{HTML}{F7F7F7}
    
    % prompt
    \makeatletter
    \newcommand{\boxspacing}{\kern\kvtcb@left@rule\kern\kvtcb@boxsep}
    \makeatother
    \newcommand{\prompt}[4]{
        {\ttfamily\llap{{\color{#2}[#3]:\hspace{3pt}#4}}\vspace{-\baselineskip}}
    }
    

    
    % Prevent overflowing lines due to hard-to-break entities
    \sloppy 
    % Setup hyperref package
    \hypersetup{
      breaklinks=true,  % so long urls are correctly broken across lines
      colorlinks=true,
      urlcolor=urlcolor,
      linkcolor=linkcolor,
      citecolor=citecolor,
      }
    % Slightly bigger margins than the latex defaults
    
    \geometry{verbose,tmargin=1in,bmargin=1in,lmargin=1in,rmargin=1in}
    
    

\begin{document}
    
    \maketitle
	
    $\textbf{Question1}$

The univariate function $1+x^2-x^4/4$ exhibits stationary points at
$x=–\sqrt{2}$, x=0, and $x=\sqrt{2}$. These are, respectively.

\begin{enumerate}
\def\labelenumi{\alph{enumi}.}
\item
  a saddle point, a maximum and a minimum.
\item
  a maximum, a minimum, and a maximum.
\item
  three maxima.
\item
  a minimum, a saddle point and a maximum.
\item
  a minimum, a maximum, and a minimum.
\end{enumerate}

$\textbf{Answer}$

    To determine the nature of the stationary points of the function
$f(x) = 1 + x^2 - \frac{x^4}{4}$, we need to examine the sign of the
second derivative $f''(x)$ at each stationary point.

Given $f(x) = 1 + x^2 - \frac{x^4}{4}$, we can find the first and
second derivatives as follows:

$f'(x) = 2x - x^3$ $f''(x) = 2 - 3x^2$

Now let's evaluate $f''(x)$ at each stationary point:

\begin{enumerate}
\def\labelenumi{\arabic{enumi}.}
\tightlist
\item
  $x = -\sqrt{2}$:
  $f''(-\sqrt{2}) = 2 - 3(-\sqrt{2})^2 = 2 - 3(2) = 2 - 6 = -4$
\item
  $x = 0$: $f''(0) = 2 - 3(0)^2 = 2$
\item
  $x = \sqrt{2}$:
  $f''(\sqrt{2}) = 2 - 3(\sqrt{2})^2 = 2 - 3(2) = 2 - 6 = -4$
\end{enumerate}

From these calculations, we can determine the concavity at each point:

\begin{itemize}
\tightlist
\item
  $x = -\sqrt{2}$: Concave down (negative second derivative), so it's
  a local maximum.
\item
  $x = 0$: Concave up (positive second derivative), so it's a local
  minimum.
\item
  $x = \sqrt{2}$: Concave down (negative second derivative), so it's a
  local maximum.
\end{itemize}

So, the correct answer is:

\textbf{b. a maximum, a minimum, and a maximum.}

    \begin{tcolorbox}[breakable, size=fbox, boxrule=1pt, pad at break*=1mm,colback=cellbackground, colframe=cellborder]
\prompt{In}{incolor}{ }{\boxspacing}
\begin{Verbatim}[commandchars=\\\{\}]

\end{Verbatim}
\end{tcolorbox}

    $\textbf{Question2}$

The univariate function $x^3 + x^2 -x -1$ exhibits stationary points
at x=-1 and $x=\frac{1}{3}$. These are, respectively,

\begin{enumerate}
\def\labelenumi{\alph{enumi}.}
\item
  a maximum and a minimum.
\item
  a saddle point and a minimum.
\item
  a minimum and a maximum.
\item
  two saddle points.
\item
  None of these.
\end{enumerate}

$\textbf{Answer}$

    To determine the nature of the stationary points of the function 
$ f(x)= x^3 + x^2 - x - 1 $ at $ x = -1 $ and $ x = \frac{1}{3} $,
we need to analyze the behavior of the function around these points
using the first and second derivative tests.

\begin{enumerate}
\def\labelenumi{\arabic{enumi}.}
\tightlist
\item
  \textbf{First Derivative Test:}

  \begin{itemize}
  \tightlist
  \item
    At a stationary point, the first derivative is zero.
  \item
    $ f'(x) = 3x^2 + 2x - 1 $
  \item
    For $ x = -1 $, $ f'(-1) = 0 $.
  \item
    For $ x = \frac{1}{3} $, $ f'\left(\frac{1}{3}\right) = 0 $.
  \end{itemize}
\item
  \textbf{Second Derivative Test:}

  \begin{itemize}
  \tightlist
  \item
    At a stationary point where the first derivative is zero, we use the
    second derivative to determine concavity.
  \item
    $ f'\,'(x) = 6x + 2 $
  \item
    For $ x = -1 $, $ f'\,'(-1) = 6(-1) + 2 = -4 $, indicating a
    maximum.
  \item
    For $ x = \frac{1}{3} $, $ f'\,'\left(\frac{1}{3}\right) =
    6\left(\frac{1}{3}\right) + 2 = 4 $, indicating a minimum.
  \end{itemize}
\end{enumerate}

So, the correct answer is:

\begin{enumerate}
\def\labelenumi{\alph{enumi}.}
\setcounter{enumi}{2}
\tightlist
\item
  a minimum and a maximum.
\end{enumerate}

    \begin{tcolorbox}[breakable, size=fbox, boxrule=1pt, pad at break*=1mm,colback=cellbackground, colframe=cellborder]
\prompt{In}{incolor}{ }{\boxspacing}
\begin{Verbatim}[commandchars=\\\{\}]

\end{Verbatim}
\end{tcolorbox}

    $\textbf{Question3}$

The univariate function $3x^4-26x^3+78x^2-90x+2$ exhibits stationary
points at x=1, x=5/2, x=3. These are respectively:

\begin{enumerate}
\def\labelenumi{\alph{enumi}.}
\tightlist
\item
  a minimum, a maximum, and a minimum.
\item
  three minima.
\item
  two maxima and one minimum.
\item
  two minima and one maximum.
\item
  a maximum, a minimum, and a maximum.
\end{enumerate}

$\textbf{Answer}$

    To determine the nature of the stationary points, we need to analyze the
behavior of the function around each point. We can do this by examining
the sign of the derivative.

Given the function $ f(x) = 3x^4 - 26x^3 + 78x^2 - 90x + 2 $,
let's find its derivative $ f'(x) $ using calculus:

$ f'(x) = 12x^3 - 78x^2 + 156x - 90 $

Now, let's evaluate $ f'(x) $ at each stationary point:

\begin{enumerate}
\def\labelenumi{\arabic{enumi}.}
\item
  At $ x = 1 $: $ f'(1) = 12(1)^3 - 78(1)^2 + 156(1) - 90 = 12
  - 78 + 156 - 90 = 0 $ Since the derivative changes sign from negative
  to positive, $ x = 1 $ is a local minimum.
\item
  At $ x = \frac{5}{2} $: $ f'\left(\frac{5}{2}\right) =
  12\left(\frac{5}{2}\right)^3 - 78\left(\frac{5}{2}\right)^2 +
  156\left(\frac{5}{2}\right) - 90 $ $ = 12\left(\frac{125}{8}\right)
  - 78\left(\frac{25}{4}\right) + 156\left(\frac{5}{2}\right) - 90 $ $
  = 187.5 - 487.5 + 390 - 90 = 0 $ Since the derivative changes sign
  from positive to negative, $ x = \frac{5}{2} $ is a local maximum.
\item
  At $ x = 3 $: $ f'(3) = 12(3)^3 - 78(3)^2 + 156(3) - 90 $ $
  = 12(27) - 78(9) + 156(3) - 90 = 0 $ Since the derivative changes
  sign from negative to positive, $ x = 3 $ is a local minimum.
\end{enumerate}

So, the correct answer is:

\begin{enumerate}
\def\labelenumi{\alph{enumi}.}
\tightlist
\item
  a minimum, a maximum, and a minimum.
\end{enumerate}

    \begin{tcolorbox}[breakable, size=fbox, boxrule=1pt, pad at break*=1mm,colback=cellbackground, colframe=cellborder]
\prompt{In}{incolor}{ }{\boxspacing}
\begin{Verbatim}[commandchars=\\\{\}]

\end{Verbatim}
\end{tcolorbox}

    $\textbf{Question4}$

The function $f(x,y)=x^3+y^3-3xy$ exhibits two stationary points one
at (x,y)=(0,0) and one at (x,y)=(1,1). These are, respectively,

\begin{enumerate}
\def\labelenumi{\alph{enumi}.}
\item
  two minima.
\item
  two maxima
\item
  a maximum and a minimum.
\item
  a saddle point and a maximum.
\item
  a saddle point and a minimum.
\end{enumerate}

$\textbf{Answer}$

    To determine the nature of the stationary points of the function $ f(x,
y) = x^3 + y^3 - 3xy $, we can use the second partial derivative
test.

\begin{enumerate}
\def\labelenumi{\arabic{enumi}.}
\item
  First, find the first partial derivatives with respect to $ x $ and
  $ y $: $ \frac{\partial f}{\partial x} = 3x^2 - 3y $ $
  \frac{\partial f}{\partial y} = 3y^2 - 3x $
\item
  Then, find the second partial derivatives: $
  \frac{\partial^2 f}{\partial x^2} = 6x $ $
  \frac{\partial^2 f}{\partial y^2} = 6y $ $
  \frac{\partial^2 f}{\partial x \partial y} = -3 $
\item
  Evaluate the second partial derivatives at the critical points:

  \begin{itemize}
  \tightlist
  \item
    At (0, 0): $ \frac{\partial^2 f}{\partial x^2} = 0 $, $
    \frac{\partial^2 f}{\partial y^2} = 0 $, $
    \frac{\partial^2 f}{\partial x \partial y} = -3 $. It's
    inconclusive.
  \item
    At (1, 1): $ \frac{\partial^2 f}{\partial x^2} = 6 $, $
    \frac{\partial^2 f}{\partial y^2} = 6 $, $
    \frac{\partial^2 f}{\partial x \partial y} = -3 $. Here, $
    \frac{\partial^2 f}{\partial x^2}
    \cdot \frac{\partial^2 f}{\partial y^2} -
    (\frac{\partial^2 f}{\partial x \partial y})^2 \textgreater{} 0
    $ and $ \frac{\partial^2 f}{\partial x^2} \textgreater{} 0 $,
    thus it's a local minimum.
  \end{itemize}
\end{enumerate}

So, the stationary points are: - (0, 0): Indeterminate - (1, 1): Local
minimum

Hence, the correct answer is option: c.~a maximum and a minimum.

    \begin{tcolorbox}[breakable, size=fbox, boxrule=1pt, pad at break*=1mm,colback=cellbackground, colframe=cellborder]
\prompt{In}{incolor}{ }{\boxspacing}
\begin{Verbatim}[commandchars=\\\{\}]

\end{Verbatim}
\end{tcolorbox}

    $\textbf{Question5}$

At a certain point (x0,y0,z0) all the partial derivatives $θf/θx$, $θf/θy$, $θf/θz$ of a given three variable function $f(x,y,z)$ are zero. The point
is:

\begin{enumerate}
\def\labelenumi{\alph{enumi}.}
\item
  A local minimum.
\item
  A local maximum.
\item
  Saddle point.
\item
  Information provided is insufficient.
\item
  A global minimum.
\end{enumerate}

$\textbf{Answer}$

    The given information suggests that the point (x0, y0, z0) is a critical
point, where all partial derivatives of the function $ f(x, y, z) $
are zero. However, this information alone doesn't determine the nature
of the critical point. To determine whether it's a local minimum,
maximum, saddle point, or something else, we need more information,
specifically about the second partial derivatives or higher.

For example, if the second partial derivatives at the point satisfy the
second derivative test:

\begin{enumerate}
\def\labelenumi{\arabic{enumi}.}
\tightlist
\item
  If the second derivative test shows that the second derivative with
  respect to x, y, and z are all positive at the point, then it is a
  local minimum.
\item
  If the second derivative test shows that the second derivative with
  respect to x, y, and z are all negative at the point, then it is a
  local maximum.
\item
  If the second derivative test shows that the second derivative with
  respect to x and y have opposite signs, or the second derivative with
  respect to x, y, or z is zero, then it is a saddle point.
\item
  If the second derivative test is inconclusive, then further analysis
  is needed.
\end{enumerate}

Without the second derivative information, we cannot definitively
determine the nature of the critical point. So, the correct answer is:

\begin{enumerate}
\def\labelenumi{\alph{enumi}.}
\setcounter{enumi}{3}
\tightlist
\item
  Information provided is insufficient.
\end{enumerate}

    \begin{tcolorbox}[breakable, size=fbox, boxrule=1pt, pad at break*=1mm,colback=cellbackground, colframe=cellborder]
\prompt{In}{incolor}{ }{\boxspacing}
\begin{Verbatim}[commandchars=\\\{\}]

\end{Verbatim}
\end{tcolorbox}

    $\textbf{Question6}$

We consider the multivariate function $f(x,y)=1/3(x^3-y^3)-(x-y)$.

It exhibits:

\begin{enumerate}
\def\labelenumi{\alph{enumi}.}
\item
  three minima and one saddle point
\item
  three maxima and one saddle point
\item
  none of these
\item
  one minimum, one maximum and two saddle points
\item
  two minima and two maxima
\end{enumerate}

$\textbf{Answer}$

    To determine the critical points of the function $ f(x, y) =
\frac{1}{3}(x^3 - y^3) - (x - y) $, we first find the partial
derivatives with respect to $ x $ and $ y $:

$ \frac{\partial f}{\partial x} = x^2 - 1 $ $
\frac{\partial f}{\partial y} = -y^2 + 1 $

Setting these partial derivatives to zero to find the critical points:

$ x^2 - 1 = 0 \Rightarrow x = \pm 1 $ $ -y^2 + 1 = 0
\Rightarrow y = \pm 1 $

So, the critical points are $ (1, 1) $, $ (-1, -1) $, $ (1, -1) $,
and $ (-1, 1) $.

To determine the nature of these critical points, we can use the second
partial derivative test or the Hessian matrix. Calculating the second
partial derivatives:

$ \frac{\partial^2 f}{\partial x^2} = 2x $ $
\frac{\partial^2 f}{\partial y^2} = -2y $ $
\frac{\partial^2 f}{\partial x \partial y} = 0 $

Now, evaluating these second partial derivatives at the critical points:

At $ (1, 1) $: $ \frac{\partial^2 f}{\partial x^2} = 2(1) = 2 $
(positive) $ \frac{\partial^2 f}{\partial y^2} = -2(1) = -2 $
(negative) So, it's a saddle point.

At $ (-1, -1) $: $ \frac{\partial^2 f}{\partial x^2} = 2(-1) = -2 $
(negative) $ \frac{\partial^2 f}{\partial y^2} = -2(-1) = 2 $
(positive) So, it's a saddle point.

At $ (1, -1) $: $ \frac{\partial^2 f}{\partial x^2} = 2(1) = 2 $
(positive) $ \frac{\partial^2 f}{\partial y^2} = -2(-1) = 2 $
(positive) So, it's a local minimum.

At $ (-1, 1) $: $ \frac{\partial^2 f}{\partial x^2} = 2(-1) = -2 $
(negative) $ \frac{\partial^2 f}{\partial y^2} = -2(1) = -2 $
(negative) So, it's a local maximum.

Hence, the function exhibits one minimum, one maximum, and two saddle
points, so the correct answer is:

\textbf{d.~one minimum, one maximum, and two saddle points}

    \begin{tcolorbox}[breakable, size=fbox, boxrule=1pt, pad at break*=1mm,colback=cellbackground, colframe=cellborder]
\prompt{In}{incolor}{ }{\boxspacing}
\begin{Verbatim}[commandchars=\\\{\}]

\end{Verbatim}
\end{tcolorbox}

    $\textbf{Question7}$

A multivariate function $f(x1,…,xn)$ exhibits a stationary point at
(x\textsuperscript{01,\ldots,x}0n). Given the Hessian
H=H(x\textsuperscript{01,\ldots,x}0n), this point is always a minimum
when:

\begin{enumerate}
\def\labelenumi{\alph{enumi}.}
\item
  All eigenvalues of H are positive.
\item
  All eigenvalues of H are negative.
\item
  trace(H)\textgreater0
\item
  det(H)\textless0
\item
  None of these.
\end{enumerate}

$\textbf{Answer}$

    To determine whether the stationary point is a minimum, we can use the
second derivative test. The second derivative test states that if the
Hessian matrix $ H $ at the stationary point has all positive
eigenvalues, then the point is a local minimum.

So the correct option is:

\begin{enumerate}
\def\labelenumi{\alph{enumi}.}
\tightlist
\item
  All eigenvalues of $ H $ are positive.
\end{enumerate}

    \begin{tcolorbox}[breakable, size=fbox, boxrule=1pt, pad at break*=1mm,colback=cellbackground, colframe=cellborder]
\prompt{In}{incolor}{ }{\boxspacing}
\begin{Verbatim}[commandchars=\\\{\}]

\end{Verbatim}
\end{tcolorbox}

    $\textbf{Question8}$

Consider the convex function $f(x_1,x_2)=x^{2}_1+10x^{2}_2$. Which of
the following choices coincides with the first gradient update $x^(1)$
having selected learning rate equal to 0.01 and initial guess
$x^(0)=[0.5,0.5]^T$

$\textbf{Answer}$

    To perform gradient descent on the function $ f(x\_1, x\_2) = x\_1^2
+ 10x\_2^2 $ with a learning rate of 0.01 and an initial guess of $
x^\{(0)\} = {[}0.5, 0.5{]}^T $, we need to compute the gradient
of the function and then update the parameters iteratively using the
gradient descent update rule.

The gradient of $ f $ with respect to $ x\_1 $ and $ x\_2 $ is:

$\nabla f(x_1, x_2) = \begin{bmatrix} 2x_1 \\ 20x_2 \end{bmatrix}$

The gradient descent update rule is:

$x^{(k+1)} = x^{(k)} - \alpha \nabla f(x^{(k)})$

where $ \alpha $ is the learning rate.

Given the learning rate $ \alpha = 0.01 $, and the initial guess $
x^\{(0)\} = {[}0.5, 0.5{]}^T $, the first gradient update $
x^\{(1)\} $ can be computed as follows:

$ x_1^{(1)}  = x_1^{(0)} - \alpha \frac{\partial f}{\partial x_1} \bigg|_{x^{(0)}} \\
 = 0.5 - 0.01 \times 2 \times 0.5 \\
 = 0.5 - 0.01 \\
 = 0.49 $ 
$ x_2^{(1)}  = x_2^{(0)} - \alpha \frac{\partial f}{\partial x_2} \bigg|_{x^{(0)}} \\
 = 0.5 - 0.01 \times 20 \times 0.5 \\
 = 0.5 - 0.1 \\
 = 0.4  $

So, the first gradient update $ x^\{(1)\} $ is approximately $
{[}0.49, 0.4{]}^T $.

    \begin{tcolorbox}[breakable, size=fbox, boxrule=1pt, pad at break*=1mm,colback=cellbackground, colframe=cellborder]
\prompt{In}{incolor}{ }{\boxspacing}
\begin{Verbatim}[commandchars=\\\{\}]

\end{Verbatim}
\end{tcolorbox}

    $\textbf{Question9}$

Consider the function $f(x,y,z)=x^3y + e\textsuperscript{\{-z\}y}2 +z^3$. 
In the application of the gradient descent optimization method, we search for a minimum along the direction given by the vector:

$\textbf{Answer}$

    To apply the gradient descent optimization method to find the minimum of
the function $ f(x, y, z) = x^3y + e\textsuperscript{\{-z\}y}2 +
z^3 $, we need to compute the gradient of the function with respect
to the variables $ x $, $ y $, and $ z $, and then move in the
direction opposite to the gradient.

The gradient of a multivariable function is a vector that points in the
direction of the steepest increase of the function, and its components
are the partial derivatives of the function with respect to each
variable.

So, let's find the gradient $ \nabla f(x, y, z) $:

$\nabla f(x, y, z) = \left( \frac{\partial f}{\partial x}, \frac{\partial f}{\partial y}, \frac{\partial f}{\partial z} \right)$

Where:

$\frac{\partial f}{\partial x} = 3x^2 y$

$\frac{\partial f}{\partial y} = x^3 + 2e^{-z}y$

$\frac{\partial f}{\partial z} = -e^{-z}y^2 + 3z^2$

So, the gradient vector $ \nabla f(x, y, z) $ is:

$\nabla f(x, y, z) = \left( 3x^2 y, x^3 + 2e^{-z}y, -e^{-z}y^2 + 3z^2 \right)$

To find the direction in which the function decreases the fastest, we
move in the direction opposite to the gradient. Thus, the direction
vector for gradient descent is:

$-\nabla f(x, y, z) = \left( -3x^2 y, -x^3 - 2e^{-z}y, e^{-z}y^2 - 3z^2 \right)$

This is the direction along which we would move to minimize the function
$ f(x, y, z) $.

    \begin{tcolorbox}[breakable, size=fbox, boxrule=1pt, pad at break*=1mm,colback=cellbackground, colframe=cellborder]
\prompt{In}{incolor}{ }{\boxspacing}
\begin{Verbatim}[commandchars=\\\{\}]

\end{Verbatim}
\end{tcolorbox}

    $\textbf{Question10}$

Consider the linear optimization problem defined by the objective
function $f(x_1,x_2)=3x_1+2x_2$ subject to the constraints
$x_1+x_2<=4$ and $2x_1+x_2<=5$.

The goal is to minimize f subject to these constraints.

Which of the following expressions correctly represents the Lagrangian
L(x1,x2,λ1,λ2) for this problem, incorporating the Lagrange multipliers
λ1 and λ2, where λi\textgreater0,i=1,2, for the respective constraints?

$\textbf{Answer}$

    The Lagrangian for the given linear optimization problem can be
expressed as follows:

$ L(x\_1, x\_2, \lambda\_1, \lambda\_2) = f(x\_1, x\_2) - \lambda\_1
(x\_1 + x\_2 - 4) - \lambda\_2 (2x\_1 + x\_2 - 5) $

So, among the given choices, the correct expression for the Lagrangian
would be:

$ L(x\_1, x\_2, \lambda\_1, \lambda\_2) = 3x\_1 + 2x\_2 - \lambda\_1
(x\_1 + x\_2 - 4) - \lambda\_2 (2x\_1 + x\_2 - 5) $

    \begin{tcolorbox}[breakable, size=fbox, boxrule=1pt, pad at break*=1mm,colback=cellbackground, colframe=cellborder]
\prompt{In}{incolor}{ }{\boxspacing}
\begin{Verbatim}[commandchars=\\\{\}]

\end{Verbatim}
\end{tcolorbox}

    $\textbf{Question11}$

Which of the following functions is/are convex?

\begin{enumerate}
\def\labelenumi{(\alph{enumi})}
\item
  $f(x)=ln(1+e^x)$ on R,
\item
  $f(x)=sin(x)$ on R,
\item
  $f(x)=sqrt{1+x^2}$ on R
\end{enumerate}

$\textbf{Answer}$

    To determine whether a function is convex, we typically look at the
second derivative. A function is convex on an interval if its second
derivative is non-negative on that interval. Let's analyze each
function:

\begin{enumerate}
\def\labelenumi{(\alph{enumi})}
\tightlist
\item
  $ f(x) = \ln(1+e^x) $ on $ \mathbb{R} $:
\end{enumerate}

To find the second derivative, we first find the first derivative:

$ f'(x) = \frac{1}{1 + e^x} \cdot e^x = \frac{e^x}{1 + e^x} $

Then, we find the second derivative:

$ f'\,'(x) = \frac{e^x \cdot (1 + e^x) - e^x \cdot e^x}{(1 + e^x)^2} =
\frac{e^x}{(1 + e^x)^2} $

This second derivative is always positive for all $ x $ in $
\mathbb{R} $. Thus, $ f(x) = \ln(1+e^x) $ is convex on $
\mathbb{R} $.

\begin{enumerate}
\def\labelenumi{(\alph{enumi})}
\setcounter{enumi}{1}
\tightlist
\item
  $ f(x) = \sin(x) $ on $ \mathbb{R} $:
\end{enumerate}

The second derivative of $ \sin(x) $ is $ -\sin(x) $, which is not
always non-negative. Thus, $ f(x) = \sin(x) $ is not convex on $
\mathbb{R} $.

\begin{enumerate}
\def\labelenumi{(\alph{enumi})}
\setcounter{enumi}{2}
\tightlist
\item
  $ f(x) = \sqrt{1+x^2} $ on $ \mathbb{R} $:
\end{enumerate}

To find the second derivative, we first find the first derivative:

$ f'(x) = \frac{1}{2 \sqrt{1 + x^2}} \cdot 2x =
\frac{x}{\sqrt{1 + x^2}} $

Then, we find the second derivative:

$ f'\,'(x) = \frac{(1 + x^2) - x^2/(1 + x^2)}{(1 + x^2)^{3/2}} =
\frac{1}{(1 + x^2)^{3/2}} $

This second derivative is always positive for all $ x $ in $
\mathbb{R} $. Thus, $ f(x) = \sqrt{1+x^2} $ is convex on $
\mathbb{R} $.

So, the functions that are convex on $ \mathbb{R} $ are (a) $ f(x) =
\ln(1+e^x) $ and (c) $ f(x) = \sqrt{1+x^2} $.

    \begin{tcolorbox}[breakable, size=fbox, boxrule=1pt, pad at break*=1mm,colback=cellbackground, colframe=cellborder]
\prompt{In}{incolor}{ }{\boxspacing}
\begin{Verbatim}[commandchars=\\\{\}]

\end{Verbatim}
\end{tcolorbox}

    $\textbf{Question12}$

Which of the following functions is/are convex ?

\begin{enumerate}
\def\labelenumi{\alph{enumi}.}
\item
  $f(x)=x^2$ on $R$,
\item
  $f(x)=3x+2$ on $R$ ,
\item
  $f(x,y)=e^{-x-y}$ on $R^2$
\end{enumerate}

$\textbf{Answer}$

    To determine if a function is convex, we typically check if its second
derivative (or Hessian matrix, for multivariable functions) is
non-negative over its domain.

Let's evaluate each function:

\begin{enumerate}
\def\labelenumi{\alph{enumi}.}
\tightlist
\item
  $f(x) = x^2$ on $\mathbb{R}$:

  \begin{itemize}
  \tightlist
  \item
    Second derivative: $f''(x) = 2$, which is always positive. Hence,
    $f(x)$ is convex on $\mathbb{R}$.
  \end{itemize}
\item
  $f(x) = 3x + 2$ on $\mathbb{R}$:

  \begin{itemize}
  \tightlist
  \item
    Second derivative: $f''(x) = 0$, which is neither positive nor
    negative. Hence, $f(x)$ is neither convex nor concave on
    $\mathbb{R}$.
  \end{itemize}
\item
  $f(x,y) = e^{-x-y}$ on $\mathbb{R}^2$:

  \begin{itemize}
  \tightlist
  \item
    Second partial derivatives: $f_{xx} = e^{-x-y}$,
    $f_{yy} = e^{-x-y}$, $f_{xy} = e^{-x-y}$.
  \item
    The Hessian matrix is: 
	$ H = \begin{pmatrix} f_{xx} & f_{xy} \\ f_{xy} & f_{yy} \end{pmatrix} = \begin{pmatrix}e^{-x-y} & e^{-x-y} \\ e^{-x-y} & e^{-x-y}\end{pmatrix}= e^\{-x-y\}
    \begin{pmatrix}1 & 1 \\ 1 & 1\end{pmatrix}
    $
  \item
    The determinant of the Hessian is $|H| = 0$, and the trace of the
    Hessian is the sum of its eigenvalues, which is $2e^{-x-y}$.
  \item
    Since the determinant is zero and the trace is positive (for
    $e^{-x-y} > 0$), $f(x,y)$ is neither convex nor concave on
    $\mathbb{R}^2$.
  \end{itemize}
\end{enumerate}

So, only function $f(x) = x^2$ is convex on $\mathbb{R}$.

    \begin{tcolorbox}[breakable, size=fbox, boxrule=1pt, pad at break*=1mm,colback=cellbackground, colframe=cellborder]
\prompt{In}{incolor}{ }{\boxspacing}
\begin{Verbatim}[commandchars=\\\{\}]

\end{Verbatim}
\end{tcolorbox}

    $\textbf{Question13}$

Which of the functions

$f_1(overrightarrow{x})=x^2_1+4x^2_2+2x_1x_2+120$ over R2,

$f_2(overrightarrow{x})=x^2_1+x^2_2+x^2_3-x_1x_2-x_2x_3+120$ over R3,

is/are convex ?

$\textbf{Answer}$

    To determine whether a function is convex, we can examine its Hessian
matrix. A function is convex if and only if its Hessian matrix is
positive semidefinite everywhere in its domain.

Let's calculate the Hessian matrices for both functions:

\begin{enumerate}
\def\labelenumi{\arabic{enumi}.}
\tightlist
\item
  For $ f\_1(\vec{x}) = x\_1^2 + 4x\_2^2 + 2x\_1x\_2 + 120 $:
\end{enumerate}

$ \text{Hessian matrix} =\begin{bmatrix}\frac{\partial^2 f_1}{\partial x_1^2} & \frac{\partial^2 f_1}{\partial x_1 \partial x_2} \\\frac{\partial^2 f_1}{\partial x_1 \partial x_2} & \frac{\partial^2 f_1}{\partial x_2^2}\end{bmatrix}=\begin{bmatrix}2 & 2 \\2 & 8\end{bmatrix}$

\begin{enumerate}
\def\labelenumi{\arabic{enumi}.}
\setcounter{enumi}{1}
\tightlist
\item
  For $ f\_2(\vec{x}) = x\_1^2 + x\_2^2 + x\_3^2 - x\_1x\_2 -
  x\_2x\_3 + 120 $:
\end{enumerate}

$ \text{Hessian matrix} =
\begin{bmatrix}
\frac{\partial^2 f_2}{\partial x_1^2} & \frac{\partial^2 f_2}{\partial x_1 \partial x_2} & \frac{\partial^2 f_2}{\partial x_1 \partial x_3} \\
\frac{\partial^2 f_2}{\partial x_1 \partial x_2} & \frac{\partial^2 f_2}{\partial x_2^2} & \frac{\partial^2 f_2}{\partial x_2 \partial x_3} \\
\frac{\partial^2 f_2}{\partial x_1 \partial x_3} & \frac{\partial^2 f_2}{\partial x_2 \partial x_3} & \frac{\partial^2 f_2}{\partial x_3^2}
\end{bmatrix}
=
\begin{bmatrix}
2 & -1 & 0 \\
-1 & 2 & -1 \\
0 & -1 & 2
\end{bmatrix}
$

Now, we need to check the eigenvalues of these matrices. If all
eigenvalues are nonnegative, then the matrix is positive semidefinite
and the function is convex.

For $ f\_1 $, the eigenvalues are approximately $ 1.1716 $ and $
8.8284 $, both positive. So, $ f\_1 $ is convex.

For $ f\_2 $, the eigenvalues are approximately $ 0.1716 $, $
1.8284 $, and $ 3 $, all positive. So, $ f\_2 $ is also convex.

Therefore, both $ f\_1 $ and $ f\_2 $ are convex functions.

    \begin{tcolorbox}[breakable, size=fbox, boxrule=1pt, pad at break*=1mm,colback=cellbackground, colframe=cellborder]
\prompt{In}{incolor}{ }{\boxspacing}
\begin{Verbatim}[commandchars=\\\{\}]

\end{Verbatim}
\end{tcolorbox}

    $\textbf{Question14}$

We want to find the minimum of the univariate function
$f(x)=e^xcos(x)$ in the interval (0,π) using the Newton's method.

We start from the point $x_0=frac{π}{2}$. The third point $x_2$ is:

$\textbf{Answer}$

    To find the minimum of the function $ f(x) = e^x \cos(x) $ in the
interval $ (0, \pi) $ using Newton's method, we need to iterate using
the formula:

$ x\_\{n+1\} = x\_n - \frac{f'(x_n)}{f''(x_n)} $

Given that $ f(x) = e^x \cos(x) $, we need to find $ f'(x) $ and
$ f'\,'(x) $:

$ f'(x) = e^x \cos(x) - e^x \sin(x) $ $ f'\,'(x) = 2e^x
\sin(x) $

Starting from $ x\_0 = \frac{\pi}{2} $, we can iterate using Newton's
method:

$ x\_\{n+1\} = x\_n - \frac{f'(x_n)}{f''(x_n)} $

Let's find the first iteration:

$ x\_1 = \frac{\pi}{2} - \frac{f'(\frac{\pi}{2})}{f''(\frac{\pi}{2})}
$

$ f'(\frac{\pi}{2}) = e^\{\frac{\pi}{2}\} \cos(\frac{\pi}{2}) -
e^\{\frac{\pi}{2}\} \sin(\frac{\pi}{2}) = 0 - e^\{\frac{\pi}{2}\}
= -\frac{1}{2} $

$ f'\,'(\frac{\pi}{2}) = 2e^\{\frac{\pi}{2}\} \sin(\frac{\pi}{2}) =
2e^\{\frac{\pi}{2}\} $

So,

$ x\_1 = \frac{\pi}{2} - \frac{-\frac{1}{2}}{2e^{\frac{\pi}{2}}} $ $
x\_1 = \frac{\pi}{2} + \frac{1}{4e^{\frac{\pi}{2}}} $

Now, for the second iteration, $ x\_1 $ will be our new $ x\_0 $:

$ x\_2 = x\_1 - \frac{f'(x_1)}{f''(x_1)} $

$ f'(x\_1) = e^\{x\_1\} \cos(x\_1) - e^\{x\_1\} \sin(x\_1) $

$ f'\,'(x\_1) = 2e^\{x\_1\} \sin(x\_1) $

Now, substitute these values and compute $ x\_2 $.

    \begin{tcolorbox}[breakable, size=fbox, boxrule=1pt, pad at break*=1mm,colback=cellbackground, colframe=cellborder]
\prompt{In}{incolor}{ }{\boxspacing}
\begin{Verbatim}[commandchars=\\\{\}]

\end{Verbatim}
\end{tcolorbox}

    $\textbf{Question15}$

The dual of the LP problem: max $5x_1+6x_2-10x_3$ subject to

$6x_1+3x_2+10x_3 <= 4$,

$7x_1+x_2+8x_3<=100$,

$4x_1+x_3<=50,x_1,x_2,x_3>=0$ is

$\textbf{Answer}$

    The dual of a linear programming (LP) problem involves flipping the
roles of the variables and constraints. In the primal problem, we have
variables $ x\_1, x\_2, x\_3 $ and constraints in the form of linear
inequalities. To form the dual, we introduce a new variable for each
constraint in the primal problem and a constraint for each variable.

The primal problem is:

Maximize: $ 5x\_1 + 6x\_2 - 10x\_3 $

Subject to:

\begin{enumerate}
\def\labelenumi{\arabic{enumi}.}
\tightlist
\item
  $ 6x\_1 + 3x\_2 + 10x\_3 \leq 4 $
\item
  $ 7x\_1 + x\_2 + 8x\_3 \leq 100 $
\item
  $ 4x\_1 + x\_3 \leq 50 $
\item
  $ x\_1, x\_2, x\_3 \geq 0 $
\end{enumerate}

The dual problem involves creating a new variable for each constraint
and a constraint for each variable:

Minimize: $ 4y\_1 + 100y\_2 + 50y\_3 $

Subject to:

\begin{enumerate}
\def\labelenumi{\arabic{enumi}.}
\tightlist
\item
  $ 6y\_1 + 7y\_2 + 4y\_3 \geq 5 $
\item
  $ 3y\_1 + y\_2 + y\_3 \geq 6 $
\item
  $ 10y\_1 + 8y\_2 \geq -10 $
\item
  $ y\_1, y\_2, y\_3 \geq 0 $
\end{enumerate}

This is the dual LP problem corresponding to the given primal LP
problem.

    \begin{tcolorbox}[breakable, size=fbox, boxrule=1pt, pad at break*=1mm,colback=cellbackground, colframe=cellborder]
\prompt{In}{incolor}{ }{\boxspacing}
\begin{Verbatim}[commandchars=\\\{\}]

\end{Verbatim}
\end{tcolorbox}

    $\textbf{Question16}$

The function $F(x)=X^TSX/X^TX$ with $X^T=(x,y)\in R^2-(0,0)$ and\\
$S=\begin{bmatrix} 1 & 0 \\ 0 & 2 \end{bmatrix}$ is not convex. It
attains a maximum as well as a minimum value.\\
At a minimum point (x0,y0) what is the second derivative matrix H ?
provide final result


$\textbf{Answer}$

    Let's follow the steps provided to solve the problem:

\begin{enumerate}
\def\labelenumi{\arabic{enumi}.}
\item
  \textbf{Compute the function after substituting X and S}:

  Given: $ F(x) = \frac{X^T S X}{X^T X} $

  Substituting $ X^T = (x, y) $ and $ S =  \begin{bmatrix} 1 & 0 \\ 0 & 2 \end{bmatrix} $, we get:

  $ F(x) =  \frac{\begin{bmatrix} x & y \end{bmatrix} \begin{bmatrix} 1 & 0 \\ 0 & 2 \end{bmatrix} \begin{bmatrix} x \\ y \end{bmatrix}}{\begin{bmatrix} x & y \end{bmatrix} \begin{bmatrix} x \\ y \end{bmatrix}}  $

  $ = \frac{\begin{bmatrix} x & y \end{bmatrix} \begin{bmatrix} x \\ 2y \end{bmatrix}}{\begin{bmatrix} x & y \end{bmatrix} \begin{bmatrix} x \\ y \end{bmatrix}} $

  $ = \frac{x^2 + 2y^2}{x^2 + y^2} $\item  \textbf{Locate the stationary points by solving $ \nabla \vec{F} = 0  $}:

  The gradient of $ F(x) $ is:

  $ \nabla \vec{F} =
  \begin{bmatrix} \frac{\partial F}{\partial x} \\ \frac{\partial F}{\partial y} \end{bmatrix}
  $

  To find stationary points, we solve $ \nabla \vec{F} = \vec{0} $:

  $ \frac{\partial F}{\partial x} =
  \frac{2x(x^2+y^2) - 2x(x^2+2y^2)}{(x^2+y^2)^2} = 0 $

  $ \frac{\partial F}{\partial y} =
  \frac{4y(x^2+y^2) - 2y(x^2+2y^2)}{(x^2+y^2)^2} = 0 $

  Simplifying these expressions:

  $ 2x(x\textsuperscript{2+y}2) - 2x(x\textsuperscript{2+2y}2) = 0 $

  $ 4y(x\textsuperscript{2+y}2) - 2y(x\textsuperscript{2+2y}2) = 0 $

  After some algebraic manipulations, we find that $ x = 0 $ and $ y
  = 0 $ are the only stationary points.
\item
  \textbf{Determine whether the stationary point leads to a minimum or
  maximum}:

  To determine whether the stationary point leads to a minimum or
  maximum, let's consider the function value at this point:

  $ F(0,0) = \frac{0^2 + 2(0)^2}{0^2 + 0^2} = \frac{0}{0} $

  At this point, the function is indeterminate. We need to further
  analyze to determine the nature of the stationary point.
\item
  \textbf{Compute the Hessian at this point}:

  The Hessian matrix is given by:

  $ H =
  \begin{bmatrix} \frac{\partial^2 F}{\partial x^2} & \frac{\partial^2 F}{\partial x \partial y} \\ \frac{\partial^2 F}{\partial y \partial x} & \frac{\partial^2 F}{\partial y^2} \end{bmatrix}
  $

  To compute the second derivatives, let's find $
  \frac{\partial^2 F}{\partial x^2} $, $
  \frac{\partial^2 F}{\partial y^2} $, and $
  \frac{\partial^2 F}{\partial x \partial y} $:

  $ \frac{\partial^2 F}{\partial x^2} =
  \frac{2(x^2+y^2)^2 - 4x^2(x^2+2y^2) - 2(x^2+y^2)2x^2}{(x^2+y^2)^3} $

  $ \frac{\partial^2 F}{\partial y^2} =
  \frac{8y^2(x^2+y^2) - 4y^2(x^2+2y^2) - 2(x^2+y^2)4y^2}{(x^2+y^2)^3} $

  $ \frac{\partial^2 F}{\partial x \partial y} =
  \frac{4xy(x^2+y^2) - 4xy(x^2+2y^2) - 2(x^2+y^2)2xy}{(x^2+y^2)^3} $

  Substituting $ x = 0 $ and $ y = 0 $ into these expressions gives
  us the elements of the Hessian matrix. After evaluation, we get:

  $ H = \begin{bmatrix} 0 & 0 \\ 0 & \frac{2}{x^2} \end{bmatrix} $

  Thus, the second derivative matrix $ H $ at the stationary point $
  (0, 0) $ is $ \begin{bmatrix} 0 & 0 \\ 0 & \frac{2}{x^2} \end{bmatrix} $.
\end{enumerate}

Therefore, the final result is $\begin{bmatrix} 0 & 0 \\ 0 & \frac{2}{x^2} \end{bmatrix}$.


    % Add a bibliography block to the postdoc
    
    
    
\end{document}
